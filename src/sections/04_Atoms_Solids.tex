\section{Multi-Particle Systems and Atoms}
\subsection{Two-Particle Systems}
\subsubsection{Properties Two-Particle Systems}
Again, we assume that $V$ is time-independent.

\newpar{}
\ptitle{Similarities to One-Particle Systems}

\begin{itemize}
    \item $\Psi$ describes system
    \item time dependence: TDSE
    \item total energy: TISE
    \item $\Psi_{gen}=\sum_{n} c_{n} \Psi_{n} \exp\left[\frac{-i E_{n} t}{\hbar}\right]$
\end{itemize}

\newpar{}
\ptitle{Differences to One-Particle Systems}
\begin{itemize}
    \item $\Psi(\mathbf{r_1},\mathbf{r_2},t)$ is a function of position of both particles
    \item $\widehat{H}$ is given by
          \begin{equation*}
              \widehat{H}=-\frac{\hbar^{2}}{2m_{1}}\nabla_{1}^{2}-\frac{\hbar^{2}}{2m_{2}}\nabla_{2}^{2}+V(\mathbf{r}_{1},\mathbf{r}_{2},t)
          \end{equation*}
          where $V$ includes any interactions between particles i.e.:
          \begin{enumerate}
              \item interactions between  electrons
              \item interactions of electrons with the nucleus
          \end{enumerate}
    \item The term
          \begin{equation*}
              |\Psi(\mathbf{r}_{1},\mathbf{r}_{2},t)|^{2} d^{3}\mathbf{r}_{1} d^{3}\mathbf{r}_{2}
          \end{equation*}
          describes the probability of finding particle 1 in volume $d^{3}\mathbf{r}_{1}$, particle 2 in volume $d^{3}\mathbf{r}_{2}$
\end{itemize}

\subsubsection{Z-Particle Systems}
In a general multi-particle system (e.g.\ atom with $Z$ protons) the Hamiltonian is given by
\begin{align*}
    \widehat{H} = & \underbrace{\sum_{j=1}^{Z}\left\{-\frac{\hbar^{2}}{2m}\nabla_{j}^{2}-\left(\frac{1}{4\pi \epsilon_{0}}\right)\frac{Ze^{2}}{r_{j}}\right\}}_{T\text{ and interaction with nucleus}} \\
    +             & \underbrace{\frac{1}{2}\left(\frac{1}{4\pi \epsilon_{0}}\right)\sum_{j\neq k}^{Z}\frac{e^{2}}{\left|\mathbf{r}_{j}-\mathbf{r}_{k}\right|}}_{\text{repulsive int.\ of electrons}}
\end{align*}
where in the ``repulsion'' term
\begin{itemize}
    \item the interaction between electrons $j,k$ is described
    \item $\frac{1}{2}$ avoids double counting due to permuting $j,k$
\end{itemize}
The ``repulsion'' term makes exact solving of the TISE, in general, impossible. An approximation and an exception are presented hereafter.

\subsubsection{Reduceable Two-Particle Problems}
In two special cases the TISE of two-particle systems can be solved analytically.

\newpar{}
\ptitle{1. Noninteracting Particles}

\begin{itemize}
    \item Corresponds to ignoring the ``repulsion'' term in $\widehat{H}$ of the Z-particle system
    \item Assume e.g.\ electrons only interact with nucleus but not each other (simplest but most inaccurate approximation)
    \item Each electron in single particle hydrogenic state: $\psi_{n,l,m_l,m_s}$
    \item See~\ref{ssec:NIP} for details
\end{itemize}

\newpar{}
\ptitle{2. Central Potential Systems}

Given a system where the particles interact only with each other via a potential that depends on their separation
\begin{equation*}
    V(\mathbf{r}_1,\mathbf{r}_2)\to V(|\mathbf{r}_1-\mathbf{r}_2|)
\end{equation*}
we can solve the two-particle TISE analytically by splitting the problem into center-of-mass motion
\begin{align*}
    \mathbf{R}\equiv\  & \frac{m_1\mathbf{r_1}+m_2\mathbf{r_2}}{m_1+m_2} &  & \text{ center of mass} \\
    M\equiv\           & m_1 + m_2                                       &  & \text{ total mass}
\end{align*}
and relative motion of the two particles
\begin{align*}
    \mathbf{r}\equiv\  & \mathbf{r_1}-\mathbf{r_2} &                      \\
    m_r\equiv\         & \frac{m_1 m_2}{m_1+m_2}   & \text{ reduced mass}
\end{align*}
In this case, the TISE separates to
\begin{equation*}
    -\frac{\hbar^{2}}{2M}\nabla_{\mathbf{R}}^{2}\psi-\frac{\hbar^{2}}{2m_r}\nabla_{\mathbf{r}}^{2}\psi+V(\mathbf{r})\psi=E\psi
\end{equation*}
and we can find a solution
\begin{equation*}
    \psi = \psi_\mathbf{R}(\mathbf{R})\cdot\psi_\mathbf{r}(\mathbf{r})
\end{equation*}
Remarks:
\begin{itemize}
    \item Works only for 2-particle system
    \item For the H-atom we calculated $\psi_\mathbf{r}(\mathbf{r})$ and assumed $m_r\approx m_e$ (plug it into $m_r$ to see why)
\end{itemize}

\subsubsection{Noninteracting Particles}\label{ssec:NIP}
As a coarse approximation we \textbf{ignore the ``repulsion'' term} from the n-particle TISE to neglect all electron-electron interactions. Then, for two given single-particle solutions $\psi_a$ and $\psi_b$ one differenciates between the following two cases.

\newpar{}
\ptitle{Distinguishable Particles}

The overall solutions are products
\begin{equation*}
    \psi(\mathbf{r}_{1},\mathbf{r}_{2})=\psi_{a}(\mathbf{r}_{1})\psi_{b}(\mathbf{r}_{2})
\end{equation*}
or
\begin{equation*}
    \psi(\mathbf{r}_{1},\mathbf{r}_{2})=\psi_{b}(\mathbf{r}_{1})\psi_{a}(\mathbf{r}_{2})
\end{equation*}
where
\begin{itemize}
    \item $\psi(\mathbf{r}_{1},\mathbf{r}_{2})$ is the overall, 2-particle wave function
    \item $\psi_{a}(\mathbf{r}_{1})$ means particle $1$ is in state $\psi_{a}$ etc.
\end{itemize}

\newpar{}
\ptitle{Indistinguishable Particles}

For the case of e.g.\ two electrons (which are indistinguishable) the overall 2-particle wave function is a \textbf{linear combination} of the aforementioned products:

\begin{equation*}
    \psi_{+}(\mathbf{r}_{1},\mathbf{r}_{2})=C\left[\psi_{a}(\mathbf{r}_{1})\psi_{b}(\mathbf{r}_{2})+\psi_{b}(\mathbf{r}_{1})\psi_{a}(\mathbf{r}_{2})\right]\quad \mathrm{symmetric}
\end{equation*}
or
\begin{equation*}
    \psi_{-}(\mathbf{r}_{1},\mathbf{r}_{2})=C\left[\psi_{a}(\mathbf{r}_{1})\psi_{b}(\mathbf{r}_{2})-\psi_{b}(\mathbf{r}_{1})\psi_{a}(\mathbf{r}_{2})\right]\quad \mathrm{antisymm.}
\end{equation*}
Remarks
\begin{itemize}
    \item Indistinguishable means that there is no way to  tell the two particles apart
    \item In practice, one can somewhat tell the particles apart mathematically if the distance between them is large i.e.\ their wave functions ``don't'' overlap. If the particles are closer together their states ``merge'' and become the (anti)symmetric overall wave function.
    \item $\psi_{+}$ stays exactly the same when swapping $\mathbf{r_1},\mathbf{r_2}$
    \item $\psi_{-}$ changes sign when swapping $\mathbf{r_1},\mathbf{r_2}$
    \item Implications of these two solutions are presented subsequently
\end{itemize}

\paragraph[Exchange Symmetry of psi]{Exchange Symmetry of Indistinguishable Particles $\psi_{\pm}$}

\ptitle{Exchange Operator}

The exchange operator swaps two particles and it's eigenvalues describe the (anti)symmetry
\begin{equation*}
    \widehat{P}\psi(\mathbf{r_1},\mathbf{r_2})=\psi(\mathbf{r_2},\mathbf{r_1})
\end{equation*}
$P$ has eigenfunctions
\begin{equation*}
    \psi_{\pm}
\end{equation*}
and eigenvalues
\begin{equation*}
    \pm 1
\end{equation*}
For two identical particles one has
\begin{align*}
    m_1=        & m_2        \\
    V(r_1,r_2)= & V(r_2,r_1)
\end{align*}
which means
\begin{equation*}
    \left[\widehat{P},\widehat{H}\right]=0
\end{equation*}
which by the generalized Ehrenfest theorem implies

\begin{equation*}
    \frac{d}{dt}\langle\widehat{P}\rangle=0
\end{equation*}
and means that $P$ is a conserved quantity and $\widehat{P}$, $\widehat{H}$ share the eigenfunctions $\psi_{\pm}$.

\ptitle{Exchange Symmetry of Noninteracting Particles}

One has that
\begin{itemize}
    \item $\psi_{+}$ is symmetric w.r.t.\ exchange (eigenvalue $+1$ for $P$)
    \item $\psi_{-}$ is antisymmetric w.r.t.\ exchange (eigenvalue $-1$ for $P$)
    \item $\psi_{\pm}$ are stationary states
    \item As $P$ is a conserved quantity a particle,\newline once (anti)symmetric w.r.t.\ exchange,\newline stays (anti)symmetric
\end{itemize}
For two \textbf{distinguishable} particles one finds that
\begin{equation*}
    \left\langle{\left(x_{1}-x_{2}\right)}^{2}\right\rangle_{d}=\left\langle x^{2}\right\rangle_{a}+\left\langle x^{2}\right\rangle_{b}-2\left\langle x\right\rangle_{a}\left\langle x\right\rangle_{b}
\end{equation*}
For two \textbf{undistinguishable} particles one has
\begin{align*}
    \left\langle{\left(x_{1}-x_{2}\right)}^{2}\right\rangle_{\pm}= & \left\langle x^{2}\right\rangle_{a}+\left\langle x^{2}\right\rangle_{b}-2\langle x\rangle_{a}\langle x\rangle_{b}\mp2|\langle x\rangle_{ab}|^{2} \\
                                                                   & =\left\langle{\left(\Delta x\right)}^{2}\right\rangle_{d}\mp2\left|\left\langle x\right\rangle_{ab}\right|^{2}
\end{align*}
Therefore, particles that are in state $\psi_{+}$ (symmetric w.r.t.\ exchange) tend to be closer together than particles that are in state $\psi_{-}$.\\
\textbf{Remarks}
\begin{itemize}
    \item $\left\langle\cdot\right\rangle_{d}/\left\langle\cdot\right\rangle_{\pm}$ is the expectation value in a distinguishable/indistinguishable state
    \item This phenomenon is often called ``exchange interaction'' or ``exchange force''
    \item Exchange symmetry or the overall 2-particle wave function influences location of particles w.r.t.\ each other
\end{itemize}

\paragraph{Bosons and Fermions}
\begin{itemize}
    \item \textbf{Bosons} (e.g.\ photons (s=1), gravitons (s=2)) are particles with \textbf{integer} spin
    \item \textbf{Fermions} (e.g.\ electrons, protons, neutrons (s=$\frac{1}{2}$)) are particles with \textbf{half-integer} spin
\end{itemize}
As bosons are indistinguishable (same holds for fermions) their overall wave functions behave according to the previously stated rules.

\ptitle{Exchange Symmetry of Hydrogen}

Because electrons are fermions one
\begin{itemize}
    \item would intuitively choose $\psi_{-}$ as an overall wave function.
    \item must choose $\psi_{+}$ because $\psi_{-}$ would mean antibonding in $\mathrm{H}_2$.
\end{itemize}
One finds that spin also influences the overall wave function according to
\begin{equation*}
    \psi(\mathbf{r_1},\mathbf{r_2})=\psi(\mathbf{r})\chi(s)
\end{equation*}
and one has for $\chi(s)$ and two identical spins $s=\frac{1}{2}$ the spin states
\begin{align*}
     & \begin{rcases}
           \uparrow \uparrow                                                        \\
           \downarrow \downarrow                                                    \\
           \frac{1}{\sqrt{2}}\left(\uparrow \downarrow + \downarrow \uparrow\right) \\
       \end{rcases} \chi_{\mathrm{triplet}}(s)\text{: symmetric w.r.t.\ ex.} \\
     & \begin{rcases}
           \frac{1}{\sqrt{2}}\left(\uparrow \downarrow - \downarrow \uparrow\right)
       \end{rcases} \chi_{\mathrm{singlet}}(s) \text{: antisymmetric w.r.t.\ ex.}
\end{align*}
which implies that the overall wave function $\psi(\mathbf{r_1},\mathbf{r_2})=\psi(\mathbf{r})\chi(s)$ of hydrogen becomes
\begin{equation*}
    \psi_{+}\cdot \chi_{\mathrm{singlet}}(s)
\end{equation*}
to achieve a stable chemical bond.

\newpar{}
\textbf{Remarks}
\begin{itemize}
    \item $\psi_{-}\cdot \left(\text{triplet}\right)$ would also lead to antisymmetric $\psi(\mathbf{r_1},\mathbf{r_2})$ but is higher in energy because of antibonding.
\end{itemize}

\newpar{}
\ptitle{Exchange Symmetry Axiom}

The exchange symmetry of the overall wave function of
\begin{itemize}
    \item identical \textbf{fermions is symmetric} w.r.t.\ exchange with
          \begin{align*}
              \psi & = \psi_{+}\cdot \chi_{\mathrm{singlet}}(s) \\
              \psi & = \psi_{-}\cdot \chi_{\mathrm{triplet}}(s)
          \end{align*}
    \item identical \textbf{bosons is antisymmetric} w.r.t.\ exchange with
          \begin{align*}
              \psi & = \psi_{+}\cdot \chi_{\mathrm{triplet}}(s) \\
              \psi & = \psi_{-}\cdot \chi_{\mathrm{singlet}}(s)
          \end{align*}
\end{itemize}

\newpar{}
\ptitle{Pauli Exclusion Principle}

One cannot have two electrons in exactly the same state (same quantum numbers) and still have the system antisymmetric w.r.t.\ exchange because if both were in the same state then
\begin{align*}
    a        & =b           \\
    \uparrow & = \downarrow % TBD: Stimmt das? (Aus Übungsstunde. Sieht etwas wild aus, macht aber schon Sinn (gleiches Prinzip wie bei \psi{-}))
\end{align*}
i.e.\ either $\chi_{\mathrm{singlet}}(s)$ or $\psi_{-}$ would become $0$ in the overall fermion wave function i.e.\ there is no such state.\\
Another explanation is that the only function that changes sign (required for fermions) \textit{and} stays the same (because both electrons are in the same state) when swapping the particles is the zero function.

\subsection{Atoms}
\subsubsection{Hydrogen Orbitals}
\paragraph{Orbital Description by Quantum Numbers (QN)}
\renewcommand{\arraystretch}{1.1}
\setlength{\oldtabcolsep}{\tabcolsep}\setlength\tabcolsep{2pt}
{\small     % Smaller font
    \begin{tabularx}{\linewidth}{@{}lllX@{}}
              & \textbf{Name} & \textbf{Range|Letter}                             & \textbf{Meaning}    \\
        $n$   & principal QN  & $0|K,\;1|L,\;2|M,\dots$                           & shell (radius)      \\
        $l$   & azimuthal QN  & $0|s,\;1|p,\;2|d,\;3|f,\dots$                     & subshell            \\
              &               & $n-1|\dots$                                       & (orbital shape)     \\
        $m_l$ & magnetic QN   & $-l,-l+1,\dots,l-1,l$                             & orbital orientation \\
              &               &                                                   & \# of orientations  \\
        $m_s$ & spin QN       & $+\frac{1}{2}/\uparrow,\:-\frac{1}{2}/\downarrow$ & spin orientation    \\
              &               &                                                   & (w.r.t.\ z-axis)    \\
        $s$   & spin          & fermions: half-int                                &                     \\
              &               & bosons: int                                       &
    \end{tabularx}
}           % End smaller font
\renewcommand{\arraystretch}{1}
\setlength{\tabcolsep}{\oldtabcolsep}

\ptitle{Visualization}

\begin{center}
    \includegraphics[width = 0.75\linewidth]{shell_subshell_orbital.png}
\end{center}
\textit{source: chemistry stack exchange}
\begin{center}
    \includegraphics[width = 0.99\linewidth]{orbitals.png}
\end{center}

\ptitle{Remarks}

\begin{itemize}
    \item One sometimes also calls orbitals
          \begin{itemize}
              \item by their azimuthal QN, e.g.\ for $l=1$: ``p-orbitals''
              \item by their principal QN, e.g.\ for $n=2$: ``2p-orbitals''
          \end{itemize}
    \item Each orbital can contain \textbf{two electrons}, one with $m_s=+\frac{1}{2}$, one with $m_s=-\frac{1}{2}$ (Pauli: unique set of QN)
    \item An s-orbital with QN $n$ has $n-1$ radial nodes
\end{itemize}

\paragraph{Orbital Table}
\renewcommand{\arraystretch}{1.1}
\setlength{\oldtabcolsep}{\tabcolsep}\setlength\tabcolsep{3pt}
{\small     % Smaller font
    \begin{tabularx}{\linewidth}{@{}llllll@{}}
            &                & \textbf{Subshell(s)} & \# \textbf{Orbitals} & \multicolumn{2}{c}{\# \textbf{of electrons}\dots}                      \\
        $n$ & \textbf{Shell} & $l=0\dots n-1$       & $m_{l}=-l,\dots l$   & \textbf{per Subshell}                             & \textbf{per Shell} \\
        1   & K              & s                    & 1                    & 2                                                 & 2                  \\
        \cmidrule{1-6}
        2   & L              & s                    & 1                    & 2                                                 & 8                  \\
            &                & p                    & 3                    & 6                                                 &                    \\
        \cmidrule{1-6}
        3   & M              & s                    & 1                    & 2                                                 & 18                 \\
            &                & p                    & 3                    & 6                                                 &                    \\
            &                & d                    & 5                    & 10                                                &                    \\
        \cmidrule{1-6}
        4   & N              & s                    & 1                    & 2                                                 & 32                 \\
            &                & p                    & 3                    & 6                                                 &                    \\
            &                & d                    & 5                    & 10                                                &                    \\
            &                & f                    & 7                    & 14                                                &
    \end{tabularx}
}           % End smaller font
\renewcommand{\arraystretch}{1}
\setlength{\tabcolsep}{\oldtabcolsep}

\ptitle{Remarks}

\begin{itemize}
    \item The total number of orbitals is the number of possible states ($n,l,m_l$) that one electron can occupy.
    \item Including spin one can have two electrons per orbital.
\end{itemize}

\subsubsection{Filling Orbitals with Electrons}
\paragraph{Hydrogenic (One-Electron) Atoms}
The hydrogenic atom model is an atom with $Z$ protons and $1$ single electron.

\ptitle{Energy Levels}

\begin{center}
    \includegraphics[width = 0.75\linewidth]{hydr_energy_levels.png}
\end{center}
\begin{itemize}
    \item Reminder: $E_n = -\left[\frac{m}{2\hbar^2}{\left(\frac{e^2}{4\pi\epsilon_0}\right)}^2\right]\frac{1}{n^2} = \frac{E_1}{n^2}$
    \item States are degenerate: the energy only depends on $n$, so the 2s and 2p orbitals are on same level given $n$ for an 1-electron atom!
\end{itemize}

\paragraph{Multielectron (Non-Hydrogenic) Atoms}

\ptitle{Energy Levels}

\begin{center}
    \includegraphics[width = 0.6\linewidth]{multielectron_energy_levels.png}
\end{center}
\begin{itemize}
    \item Example of screening: 1s and 2s electrons neutralize a part of the influence of the nucleus on the p orbitals
          \begin{itemize}
              \item The energy states become non-degenerate due to electron-electron interactions
              \item s electrons are closer to nucleus: higher attraction to nucleus $\leftrightarrow$ lower in energy
              \item p electrons are further away due to screening: lower attraction to nucleus $\leftrightarrow$ higher in energy
          \end{itemize}
    \item Screening results in different energies depending on location of electron i.e.\ shape of subshell
\end{itemize}

\ptitle{Filling Energy Levels with Electrons}

As the shells overlap, one can not just assume that a larger $n$ yields a larger energy level (e.g.\ $E_{4s}<E_{3d}$). Therefore, one needs to follow a specific order when filling orbitals with electrons. One can either use the graphical method:
\begin{center}
    \includegraphics[width = 0.4\linewidth]{fill_energies_graphical.png}
\end{center}
\textit{source: http://hyperphysics.phy-astr.gsu.edu/}\\
or use the periodic table. Both yield the filling order
\begin{equation*}
    1s^2 2s^2 2p^6 3s^2 3p^6 4s^2 3d^{10}\dots
\end{equation*}

\ptitle{Noble Gas Notation}

Instead of writing all base orbitals down, even for atoms with large $Z$, one can initiate the electron configuration formula with the noble gas preceding the element in brackets.

\newpar{}
\textbf{Example}:
\begin{equation*}
    Fe(Z=26): 1s^2 2s^2 2p^6 3s^2 3p^6 4s^2 3d^6
\end{equation*}
can be rewritten as
\begin{equation*}
    Fe(Z=26): \left[Ar\right]4s^2 3d^6
\end{equation*}

\ptitle{Exceptions}

Some elements don't follow the general rules for filling orbitals. Two important examples are
\begin{equation*}
    Cr: \left[Ar\right]4s^1 3d^5
\end{equation*}
\begin{equation*}
    Cu: \left[Ar\right]4s^1 3d^{10}
\end{equation*}
which yields an energetically more optimal state.

\ptitle{Valence Electrons}

\begin{itemize}
    \item The chemical reactivity strongly depends on the \# of valence electrons (outermost shell)
    \item Noble (inert) gases have a full valence shell: don't react chemically
\end{itemize}

\subsubsection{Full-State Description of Atoms}

\begin{itemize}
    \item Full subshells give fixed contributions to the state of an atom.
    \item Non-full subshells have freedom in how to fill their orbitals with electrons. Therefore, one needs to establish a method to describe these states and answer the question ``in what orbital and spin angular momentum states are these electrons''?
    \item This method is especially useful when searching for the ground state of a certain atom.
\end{itemize}

\ptitle{Important Rules}

\begin{enumerate}
    \item \textbf{Filled subshells} never contribute to $L$, $S$, $J$.
    \item One can use the \textbf{addition rule} to find all possible combinations of momentum cones.
    \item All the angular momentum quantum numbers ($l,s,\dots,L,S,\dots$) must be \textbf{multiplied by} $\boldsymbol{\hbar}$ to find the actual angular momenta.
\end{enumerate}

\paragraph{Total Momentum Quantum Numbers}

\ptitle{Angular Momentum (AM) in Multiparticle Systems}

One can establish a full-state description of an atom using total (orbital and spin) angular momentum. As these observables commute with $\hat{H}$ they are conserved quantities and useful for state description. For orbital angular momentum one defines
\begin{align*}
    L                   & \text{: total orbital AM}                \\
    M_L=\sum_{i}m_{l,i} & \text{: total orb. AM projection on $z$} \\
\end{align*}
For spin one has
\begin{align*}
    S                   & \text{: total spin AM}                    \\
    M_S=\sum_{i}m_{s,i} & \text{: total spin. AM projection on $z$} \\
\end{align*}
and in total one has
\begin{align*}
    J           & \text{: total AM}                   \\
    M_J=M_L+M_S & \text{: total AM projection on $z$} \\
\end{align*}

Remarks
\begin{itemize}
    \item For $M_L=0$ one also has $L=0$
    \item For $M_S=0$ one also has $S=0$
    \item $M_L$ relates to $L$ as $m_l$ relates to $l$ (i.e.\ $M_L$ is a projection of total angular momentum along $z$)
    \item $M_S$ relates to $S$ as $m_s$ relates to $s$
    \item One finds the possible values for $L,S,J$ using the addition rule.
\end{itemize}

\ptitle{Addition Rule}

Given two orbital or spin angular momenta $j_1, j_2$ one has
\begin{equation*}
    j_{\mathrm{total}}=(j_1+j_2),\dots,|j_1-j_2|
\end{equation*}
with integer steps in between.\\ 
Remarks:
\begin{itemize}
    \item Describes combinations to fill partially-filled subshells (many possible combinations).
    \item Arises because AM are directional and quantized.
    \item Depending on the specific values for the $j_i$ one has half-integer, integer or multi-integer steps between possible $J$ (Caution: often not all in-between steps are taken)!
    \item Given ranges for $L$ and $S$ for an atom, one can choose a specific pair $(L,S)$ and apply the addition rule to get $J$ i.e.
          \begin{equation*}
              J=(L+S),\dots,|L-S|
          \end{equation*}
    \item Remember that for electrons one always has $s=+\frac{1}{2}$ but $m_s=\pm \frac{1}{2}$
\end{itemize}

\ptitle{Term Symbol}

As partially-filled subshells impose a range of possible atom states (ways to fill electrons in subshells) given by the combinations of the $L,S,J$ one defines a simplified notation given by
\begin{equation*}
    \prescript{2S+1}{}{X_J}
\end{equation*}
where one uses for $X$ the same letters as for the subshells
\begin{equation*}
    X=
    \begin{cases}
        S,     & L=0 \\
        P,     & L=1 \\
        D,     & L=2 \\
        \vdots &
    \end{cases}
\end{equation*}

Remarks
\begin{itemize}
    \item The superscript $2S+1$ stands for the \# of projections
    \item The special case of $2S+1=1$ is called singlet
    \item The special case of $2S+1=3$ is called triplet
\end{itemize}

\ptitle{Excange Symmetry}

\begin{itemize}
    \item The addition rule does not consider exchange symmetry (Pauli exclusion).
    \item Therefore, not all states $\prescript{2S+1}{}{X_J}$ found for a certain atom are antisymmetric w.r.t.\ exchange but one needs the latter for electrons!
    \item Some states would require 2 electrons with exactly the same quantum numbers.
    \item There are specific rules to find these valid states (not part of this course).
\end{itemize}

\ptitle{Hund's Rules for Finding Ground States}

After considering exchange symmetry, one has reduced the number of possible states. One can the apply Hund's rule to find the lowest-energy state called ``ground state''
\begin{enumerate}
    \item State with largest $S$ is most stable
    \item For states with same $S$, largest $L$ is most stable
    \item For states with same $S$ and $L$ (but different $J$)
          \begin{itemize}
              \item Smallest $J$ most stable for subshells that are no more than half full
              \item Larqest $J$ most stable for subshells that are more than half full
          \end{itemize}
\end{enumerate}


\section{Solids}
\subsection{Free-Electron Model}
Neglecting spin and electron-electron interactions, an electrons wavefunction in a box with length $l_x,l_y,l_z$ is given by
\noindent\begin{align*}
    E_{n_x,n_y,n_z}    & = \frac{\hbar^2\pi^2}{2m}\left(\frac{n_x^2}{l_x^2}+\frac{n_y^2}{l_y^2}+\frac{n_z^2}{l_z^2}\right),\quad n_x,n_y,n_z \in \mathbb{N}\backslash 0 \\
    \psi_{n_x,n_y,n_z} & = \sqrt{\frac{8}{l_x l_y l_z}} \sin\left(\frac{n_x\pi}{l_x}x\right) \sin\left(\frac{n_y\pi}{l_y}y\right) \sin\left(\frac{n_z\pi}{l_z}z\right)
\end{align*}
Based on this solution of the SE, the distance between the discrete energy levels in a 1 cm box is at most
\noindent\begin{equation*}
    \Delta E \leq 10^{-6} ~\mathrm{eV}
\end{equation*}
and therefore the states form continuous \textit{electron bands}.


\subsubsection{k-Space}
By transforming the energy into the k-space (wave numbers), the energy becomes
\noindent\begin{equation*}
    E = \frac{\hbar^2k^2}{2m}, \quad \begin{cases}
        k^2 = k_x^2+k_y^2+k_z^2 \\
        k_x = \frac{n_x\pi}{l_x}, k_y = \frac{n_y\pi}{l_y}, k_z = \frac{n_z\pi}{l_z}
    \end{cases}
\end{equation*}
and each solution occupies
\noindent\begin{equation*}
    \frac{\pi^3}{l_x l_y l_z}=\frac{\pi^3}{V}
\end{equation*}
of k-space.
% TBD: where did the n_{x}, n_{y} ... go?

\subsubsection{Fermi Energy}
For electrons (fermions), the energy levels are filled with 2 $e^-$ at each level, and the highest occupied level is called the \textbf{Fermi level}.

\newpar{}
For a solid with $N$ atoms with $q$ valence electrons each the total volume
\noindent\begin{equation*}
    \frac{1}{8}\cdot\underbrace{\frac{4}{3}\pi k_F^3}_{\textsf{vol. sphere}} = \underbrace{\frac{Nq}{2}}_{\frac{\textsf{\# electrons}}{\mathsf{spin}}}\cdot \underbrace{\frac{\pi^3}{V}}_{\textsf{vol. per state}}
\end{equation*}
is occupied in k-space until all electron states are occupied.\\
The Fermi wave number and its energy can then be found as:
\noindent\begin{align*}
    k_F & = {\left(\frac{3\pi^2 N q}{V}\right)}^{\frac{1}{3}}                   \\
    E_F & = \frac{\hbar^2}{2m}{\left(\frac{3\pi^2 N q}{V}\right)}^{\frac{2}{3}}
\end{align*}

The \textit{density of states} - the number of one-electron levels per unit energy is given by
\noindent\begin{equation*}
    D(E)=\frac{\partial N(E)}{\partial E} = \frac{V}{2\pi^2}{\left(\frac{2m}{\hbar^2}\right)}^{\frac{3}{2}} E^{\frac{1}{2}}
\end{equation*}
and from this, the spacing between energy levels results:
\noindent\begin{equation*}
    \Delta E = \frac{1}{D(E)}
\end{equation*}

\subsection{Potential Models}
\subsubsection{Bloch's Theorem}
Combining a \textbf{periodic} potential
\noindent\begin{equation*}
    V(X) = V(x+a)
\end{equation*}
with the 1D TISE, \textit{Bloch's Theorem} states that
\noindent\begin{equation*}
    \psi(x+a) = e^{iKa} \psi(x), \quad K\in \mathbb{R}
\end{equation*}
In words: The wave function only changes in space by a phase factor.

\newpar{}
As a result the probability density is also periodic:
\noindent\begin{equation*}
    {|\psi(x+a)|}^2={|\psi(x)|}^2
\end{equation*}

\subsubsection{Periodic Boundary Conditions}
With periodic boundary conditions
\noindent\begin{equation*}
    \psi(x+Na) = \psi(x)
\end{equation*}
and Bloch's theorem
\noindent\begin{align*}
    \psi(x) = \psi(x+Na) & = e^{iKNa}\psi(x) \\
    1                    & =e^{iKNa}
\end{align*}

the quantization of $K$ can be found:
\noindent\begin{equation*}
    K=\frac{2\pi j}{Na}, \qquad j\in \mathbb{Z}
\end{equation*}

\ptitle{Potential}

In its simples form, the potential can be modeled with a \textit{Dirac-comb}:
\noindent\begin{equation*}
    V(x)=\alpha\sum_{j=0}^{N-1}\delta(x-ja)
\end{equation*}
Solving the 1D TISE for this potential results in
\noindent\begin{equation*}
    \underbrace{\cos(Ka)}_{\in [-1,1]} = \underbrace{\cos(ka) + \frac{m\alpha}{\hbar^2 k}\sin(ka)}_{\textsf{exceeds }[-1,1]}, \quad k=\sqrt{\frac{2mE}{\hbar^2}}
\end{equation*}

As a consequence, only certain ranges/bands of energy are allowed.
\begin{center}
    \includegraphics[width=.5\linewidth]{solids_gaps.png}
\end{center}

\subsubsection{Energy Bands and Band Gaps}

\begin{center}
    \includegraphics[width=\linewidth]{solids_gaps_materials.png}
\end{center}