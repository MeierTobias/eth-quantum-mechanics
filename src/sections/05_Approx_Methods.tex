\section{Approximation Methods}

\ptitle{Perturbation theory}
\begin{itemize}
    \item Attempts to solve new problems given the solution of an existing, simpler one.
    \item Provides corrections to \textbf{all} energy levels of a certain \textbf{order}.
\end{itemize}
\ptitle{Variational principle}
\begin{itemize}
    \item Doesn't need previous knowledge of energies or states but, in turn, can also not use it to yield a broad range of results.
    \item Provides an upper bound for the ground state energy (and some information on the 1st excited state) using trial functions.
\end{itemize}

\subsection{Time-Independent Perturbation Theory}
The perturbed Hamiltonian is given by
\begin{equation*}
    H=H^{(0)}+\lambda H^{\prime}
\end{equation*}
with an adjustable dummy parameter $\lambda$ (finally set to $1$). The perturbed TISE is therefore
\begin{equation*}
    H\psi_n = E_n \psi_n
\end{equation*}

\begin{examplesection}[Matrix example]
    \noindent\begin{equation*}
        \widehat{\mathbf{H}} = \underbrace{V_0 
        \begin{bmatrix}
            1&0&0\\
            0&1&0\\
            0&0&2
        \end{bmatrix}}_{\mathbf{H}^{(0)}}
        + \underbrace{V_0 \varepsilon
        \begin{bmatrix}
            -1&0&0\\
            0&0&1\\
            0&1&0
        \end{bmatrix}}_{\lambda\mathbf{H}^{\prime}}
    \end{equation*}

    since $\mathbf{H}^{(0)}$ is already diagonalized
    \noindent\begin{gather*}
        \underbrace{\lambda_1 = \lambda_2 = V_0}_{\textsf{degenerate}},\quad \underbrace{\lambda_3 = 2V_0}_{\textsf{non-degenerate}}\\
        \psi_1^{(0)} = \begin{bmatrix}
            1&0&0
        \end{bmatrix}^{\mathsf{T}},
        \psi_2^{(0)} = \begin{bmatrix}
            0&1&0
        \end{bmatrix}^{\mathsf{T}},
        \psi_3^{(0)} = \begin{bmatrix}
            0&0&1
        \end{bmatrix}^{\mathsf{T}}
    \end{gather*}

    \newpar{}
    \ptitle{Non-Degenerate (3)}
    \noindent\begin{equation*}
        E_3 = E_3^{(0)}+E_3^{(1)}+E_3^{(2)} = V_0(2+\varepsilon^2)
    \end{equation*}

    \ptitle{Degenerate (1\&2)}
    \noindent\begin{align*}
            E_1 &= E_1^{(0)} + E_{-}^{(1)} = \lambda_1+ E_{-}^{(1)} \\
            E_2 &= E_2^{(0)} + E_{+}^{(1)} = \lambda_1+ E_{+}^{(1)}
    \end{align*}
\end{examplesection}

\subsubsection{Non-Degenerate Perturbation Theory}
Approximating perturbation of non-degenerate states and energies can be achieved by non-degenerate perturbation theory.
\newpar{}
\ptitle{Solving by Power Series}

The perturbed total energy and state can then be approximated by a power series (e.g.\ Taylor series expansion) in $\lambda$
\begin{align*}
    E_{n}    & =E_{n}^{(0)}+\lambda E_{n}^{(1)}+\lambda^{2}E_{n}^{(2)}+\cdots         \\
    \psi_{n} & =\psi_{n}^{(0)}+\lambda\psi_{n}^{(1)}+\lambda^{2}\psi_{n}^{(2)}+\cdots
\end{align*}
Where $E_{n}^{(k)}$, $\psi_{n}^{(k)}$ are \textit{kth-order  corrections}. Rearranging the perturbed TISE and \textit{testing} with $\psi_{n}^{(0)}$ yields the

\newpar{}
\ptitle{1st and 2nd Order Approximations}
\begin{equation*}
    E_{n}^{(1)}=\left\langle\psi_{n}^{(0)}\right| H^{\prime} \left.\psi_{n}^{(0)}\right\rangle
\end{equation*}

\begin{equation*}
    \psi_{n}^{(1)}=\sum_{m\neq n}\frac{\left\langle\psi_{m}^{(0)}\right|H^{\prime}\left.\psi_{n}^{(0)}\right\rangle}{\left(E_{n}^{(0)}-E_{m}^{(0)}\right)}\psi_{m}^{(0)}
\end{equation*}

\begin{equation*}
    E_{n}^{(2)}=\sum_{m\neq n}\frac{\left|\left\langle\psi_{m}^{(0)}\right|H^{\prime}\left.\psi_{n}^{(0)}\right\rangle\right|^{2}}{E_{n}^{(0)}-E_{m}^{(0)}}
\end{equation*}

\newpar{}
\ptitle{Remarks}
\begin{itemize}
    \item The 1st order energy and state corrections are given combining exising knowledge (of the energies and states) with the new, perturbed Hamiltonian.
    \item The denominators in the given formulas blow up for degenerate states. Therefore, one needs to use degenerate perturbation theory for approximation of these states/energies.
\end{itemize}

\subsubsection{Degenerate Perturbation Theory}
Given \textbf{two} degenerate unperturbed states
\begin{align*}
    H^{(0)}\psi_a^{(0)}                                            & =E^{(0)}\psi_a^{(0)} \\
    H^{(0)}\psi_b^{(0)}                                            & =E^{(0)}\psi_b^{(0)} \\
    \left\langle\psi_a^{(0)}\right|\left.\psi_b^{(0)}\right\rangle & =0
\end{align*}
then, any linear combination of the states
\begin{equation*}
    \psi^{(0)}=\alpha\psi_a^{(0)}+\beta\psi_b^{(0)}
\end{equation*}
is still an eigenstate of $H^{(0)}$, with the same eigenvalue $E^{(0)}$:
\begin{equation*}
    H^{(0)}\psi^{(0)}=E^{(0)}\psi^{(0)}
\end{equation*}
Therefore, the approximation formulas from the previous section become ambiguous. Following a similar approach as for non-degenerate perturbation theory but using $\psi^{(0)}$ yields the

\newpar{}
\ptitle{1st Order Approximation}
\begin{gather*}
    E_1 = E_1^{(0)} + E_{-}^{(1)} \quad < \quad E_2 = E_2^{(0)} + E_{+}^{(1)}\\
    E_{\pm}^{(1)}=\frac{1}{2}\left[W_{aa}+W_{bb}\pm\sqrt{{(W_{aa}-W_{bb})}^{2}+4 |W_{ab}|^{2}}\right]\\
    W_{ij}=\left\langle\psi_{i}^{(0)} \right| H^{\prime} \left.\psi_{j}^{(0)}\right\rangle
\end{gather*}
This solution arises from finding a suitable basis for the degenerate states, i.e.\ the right linear combination $\psi^{(0)}=\alpha\psi_a^{(0)}+\beta\psi_b^{(0)}$ so that we can use $\psi^{(0)}$ in the formulas from the previous section. Then, the two $\psi$ are linearly independent basis vectors, such as the standard basis vectors.

\newpar{}
\ptitle{Remarks}
\begin{itemize}
    \item As we have degenerate states, we need two corrections (the $\pm$): one for each state's energy (eigenstate).
\end{itemize}

\paragraph{n-Fold Degeneracies}
Finding $E_{\pm}^{(1)}$ is equivalent to diagonalizing a degenerate block $\mathbf{W}$ in the Hamiltonian matrix $\mathbf{H}$ i.e.\ solving
\begin{equation*}
    \underbrace{
        \left(\begin{array}{cc}
            W_{aa} & W_{ab} \\
            W_{ba} & W_{bb}
        \end{array}\right)}_{\mathbf{W}}
    \left(\begin{array}{c}
            \alpha \\
            \beta
        \end{array}\right)
    =E^{(1)}\left(\begin{array}{c}
            \alpha \\
            \beta
        \end{array}\right)
\end{equation*}
This method can be generalized to n-degenerate states where one has an $n\times n$ $\mathbf{W}$ matrix.\\
If $\mathbf{W}$ is diagonal
\begin{itemize}
    \item The $\psi_a^{(0)},\psi_b^{(0)},\psi_c^{(0)} \dots $ are already proper eigenstates
    \item One has for the 2-fold degenerate case
          \begin{align*}
              E_{+}^{(1)} & =W_{aa}=\left\langle\psi_{a}^{(0)}\right|\left.\widehat{H}^{\prime}\psi_{a}^{(0)}\right\rangle \\
              E_{-}^{(1)} & =W_{bb}=\left\langle\psi_{b}^{(0)}\right|\left.\widehat{H}^{\prime}\psi_{b}^{(0)}\right\rangle
          \end{align*}
          which is what we found for non-degenerate perturbation theory.
\end{itemize}

\subsection{Variational Principle}
In contrast to perturbation theory, using the variational principle we can
\begin{itemize}
    \item Only find an upper bound for the \textbf{ground state} energy
    \item Not solve the TISE given $\widehat{H}$
\end{itemize}
The variational principle theorem then states:
\begin{equation*}
    E_{gs}\leq\left\langle\psi_{\mathrm{trial}}\right|\widehat{H} \psi_{\mathrm{trial}}\left.\right\rangle \equiv \left<H\right>
\end{equation*}
The estimate for the upper bound can be optimized by inserting (possibly many) parameters into $ \psi_{\mathrm{trial}}$ and minimizing the resulting bound.


\newpar{}
\ptitle{Example}

In a \textit{quartic oscillator} one has
\begin{equation*}
    \widehat{H}=\frac{-\hbar^{2}}{2m}\frac{d^{2}}{dx^{2}}+Cx^{4}
\end{equation*}
A common trial function is a Gaussian curve
\begin{equation*}
    \Psi_{\mathrm{trial}}={\left(\frac{\alpha}{\pi}\right)}^{1/4}\exp\left[-\alpha x^{2}/2\right]
\end{equation*}
resulting in
\begin{equation*}
    \left\langle H\right\rangle=\frac{\hbar^2\alpha}{4m}+\frac{3c}{4\alpha^2}
\end{equation*}

Minimizing the expecation value
\noindent\begin{equation*}
    \frac{d}{d \alpha} \left\langle H\right\rangle=0
\end{equation*}
yields 
\begin{equation*}
    E_{gs}\leq \frac{3}{8}{\left(\frac{6c\hbar^4}{m^2}\right)}^{\frac{1}{3}}, \quad\mathrm{with}\; \alpha_{\min} = {\left(\frac{6mc}{\hbar^2}\right)}^{\frac{1}{3}}
\end{equation*}
