\section{Schroedinger Equation}

\subsection{Stationary States}
The Schrödinger equation can be solved by separation of variables:
\noindent\begin{align*}
    \Psi_n(x,t)                                                        & = \psi_n(x)\varphi_n(t)                                                                                         \\
    \underbrace{i\hbar\frac1\varphi\frac{d\varphi}{dt}}_{\varphi_n(t)} & =\underbrace{-\frac{\hbar^2}{2m}\frac1\psi\frac{d^2\psi}{dx^2}+V}_{\psi_n (x)} = \underbrace{E}_{\text{const.}}
\end{align*}

\ptitle{Left hand side:}

The \textbf{time-dependent} part is a first order ODE that can be solved by:
\noindent\begin{align*}
    \frac{d\varphi}{dt} & =-\frac{iE}{\hbar}\varphi \quad\Leftrightarrow                                 \\
    \varphi_n(t)        & =\exp\left[\frac{-iE_n t}{\hbar}\right], \qquad E_n = \frac{\hbar^2 k_n^2}{2m}
\end{align*}

Where the solution $\varphi_n(t)$ is called the \textbf{phase factor}.
\newpar{}

\ptitle{Right hand side:}

The \textbf{time-independent} part is a second order ODE, whoose solution depends on the potential $V(x)$ (i.e.\ a measure of the environment of the particle).
\noindent\begin{align*}
    \Bigl[\overbrace{-\frac{\hbar^2}{2m}\frac{d^2}{dx^2}}^{E_{kin}}+ \overbrace{V(x)}^{E_{pot}}\Bigr]\psi & = E\psi \\
    \widehat{H}\psi                                                                                       & = E\psi
\end{align*}
Which is also called the \textbf{time independent} SE (\textbf{TISE}).

\newpar{}
Assuming that the particle has only kinetic energy ($V(x) = 0$), two general solutions are possible:
\noindent\begin{align*}
    \psi(x) & =A\sin(kx)+B\cos(kx)                                               &  & \text{standing wave (ISW)} \\
    \psi(x) & =Ce^{ikx}+De^{-ikx}                                                &  & \text{free particle}       \\
    k       & =\sqrt{\frac{2mE}{\hbar^{2}}}=\frac{p}{\hbar}=\frac{2\pi}{\lambda} &  & \text{wave number}
\end{align*}

\ptitle{Hamiltonian}

$\widehat{H}$ is the Hamiltonian operator representing \textbf{total energy}:
\noindent\begin{equation*}
    \langle H\rangle = E,\quad{\sigma_H}^2 = 0, \quad \langle p\rangle = 0
\end{equation*}

\newpar{}

\textbf{Remarks:}
\begin{itemize}
    \item $|\Psi(x,t)|^2 = |\psi(x)|^2$ (prob.\ density is time-independent):
          \noindent\begin{equation*}
              \langle Q\rangle=\int_{-\infty}^\infty\psi^*(x)\widehat{Q}\psi(x)dx
          \end{equation*}
    \item $\widehat{H}\psi = E\psi$ is an \textit{eigenvalue equation}
    \item every expectation value is \textbf{constant in time}
\end{itemize}

\subsubsection{Combining the Separable Solutions}
The \textbf{separable solutions}
\noindent\begin{equation*}
    \Psi_n(x,t)=\psi_n(x)\underbrace{e^{-iE_n t/\hbar}}_{\varphi_n(t)}
\end{equation*}
are \textbf{stationary states} that can be superpositioned to obtain $\Psi(x,t)$ (\textit{Fourier series}):
\noindent\begin{equation*}
    \Psi(x,t) =\sum_{n=1}^\infty c_n\psi_n(x) \underbrace{\exp\left[\frac{-iE_n t}{\hbar}\right]}_{\varphi_n(t)}=\sum_{n=1}^\infty c_n\Psi_n(x,t)
\end{equation*}
where $|c_n|^2$ is the \textit{probability that an energy measurement would result in} $E_n$. Therefore $|c_n|^2$ is normalized and energy is conserved:
\noindent\begin{equation*}
    \sum_{n=1}^\infty|c_n|^2 =1
\end{equation*}
The expectation value of the energy of a particle in a general state is given by:
\noindent\begin{equation*}
    \langle H\rangle=\sum_{n=1}^\infty|c_n|^2E_n
\end{equation*}

\ptitle{Remarks:}
\begin{itemize}
    \item In other words, $\Psi_n$ \textit{says} where a particle most likely \textit{is} when it has the energy $E_n$. So $\Psi$ tells us that a particle is in serval $\Psi_n$'s with several $E_n$'s at the same time. A Measurement will \textbf{collapse} $\Psi$ into one of the $\Psi_n$'s.
    \item In contrast to the case of stationary states, probabilities and expectations of the general solution (a linear combination of stationary states) are in general \textbf{not} time-independent.
    \item As $c_n$ are independent on time, the probability to get a certain energy and $\langle H\rangle$ are constant in time (energy conservation).
\end{itemize}
\subsubsection{Free Particle}
A \textit{free particle} propagates in space as a \textbf{wave packet} and its wavefunction is determined by
\noindent\begin{equation*}
    \Psi_{wp}(x,t)=\frac{1}{\sqrt{2\pi}}\int_{-\infty}^{\infty}g(k)\exp\left[i\left(kx- \underbrace{\frac{\hbar k^{2}}{2m}}_{\omega}t\right)\right]dk
\end{equation*}
the corresponding \textbf{shape function} can be determined from the initial condition $\Psi(x,0)$ i.e.\ setting $t=0$ ($g(k)$ is the Fourier transform of the initial state):
\noindent\begin{equation*}
    g(k)=\frac{1}{\sqrt{2\pi}}\int_{-\infty}^{\infty}\Psi_{wp}(x,0)e^{-ikx}dx
\end{equation*}

with:

\begin{align*}
    k & = \frac{2\pi}{\lambda} \\
    p & = \hbar k
\end{align*}

\textbf{Remark:}

\begin{itemize}
    \item A free particle can have any positive energy $E$.
    \item There is no free particle with a definite energy.
    \item $\frac{1}{\sqrt{2\pi}}$ plays the role of $c_n$ from the discrete superposition case.
    \item Separable solutions do not represent physical states, therefore $\notin$ Hilbert space.
    \item The particle carries a range of $k$
\end{itemize}

\ptitle{Group Velocity}

A wavepacket consisting of infinitely many waves that have a different \textit{group velocity} than the \textit{phase velocity} of the individual waves:
\noindent\begin{equation*}
    v_{group} = v_{classical} = 2v_{phase}
\end{equation*}

\begin{center}
    \includegraphics[width = 0.4\linewidth]{group_vel.png}
\end{center}


\subsubsection{Infinite Square Well (ISW)}\label{ssec:ISW}
A particle constrained in ``walls'' behaves differently from a free particle. Given the potential i.e.\ constraint
\noindent\begin{equation*}
    V(x)=\begin{cases}0,&0\le x\le a\\\infty,&\text{otherwise}\end{cases}
\end{equation*}
the particle will manifest as infinitely many standing waves
\noindent\begin{align*}
    \psi_{n}(x)=\sqrt{\frac{2}{a}}\sin\left(\frac{n\pi}{a}x\right)
\end{align*}
each with a corresponding energy (\textbf{quantized})
\noindent\begin{equation*}
    E_{n}=\frac{\hbar^{2}k_{n}^{2}}{2m}=\frac{n^{2}\pi^{2}\hbar^{2}}{2ma^{2}} > 0
\end{equation*}

\begin{center}
    \includegraphics[width = 0.4\linewidth]{ISW.png}
\end{center}

\textbf{Remarks:}
\begin{itemize}
    \item Solutions alternate between \textit{odd} and \textit{even} starting from $\psi_1$: \textit{odd}
    \item $\psi_{n+1}$ has one node (zero-crossing) more than $\psi_n$
    \item $\psi_n$ are eigenfunctions/vectors and $E_n$ are the corresponding eigenvalues
    \item $\psi_n$ are orthonormal base functions (complete). This implies that any function can be  described by this \textit{Fourier series} (\textbf{Dirichlet's theorem}):
          \noindent\begin{equation*}
              \int_{-\infty}^{\infty} \psi_m^*\psi_n\; dx = \delta_{mn}= \begin{cases}
                  1 & m=n     \\
                  0 & m\neq n
              \end{cases}
          \end{equation*}
\end{itemize}

\ptitle{Stationary states}
\begin{equation*}
    \Psi_n(x,t)=\underbrace{\sqrt{\frac{2}{a}}\sin\left(\frac{n\pi}{a}x\right)}_{\psi_n} \underbrace{\exp\left[-i\frac{n^{2}\pi^{2}\hbar}{2ma^{2}}t\right]}_{\varphi_n}.
\end{equation*}

\ptitle{General Solution}

These stationary states can be combined to form
\noindent\begin{equation*}
    \Psi(x,t)=\sum_{n=1}^{\infty}c_{n} \underbrace{\sqrt{\frac{2}{a}}\sin\left(\frac{n\pi}{a}x\right)}_{\psi_n} \underbrace{\exp\left[-i\frac{n^{2}\pi^{2}\hbar}{2ma^{2}}t\right]}_{\varphi_n}.
\end{equation*}
With the initial condition $\Psi(x,0)$ the weights $c_n$ can be determined:
\noindent\begin{equation*}
    c_n=\sqrt{\frac{2}{a}}\int_0^a\sin\left(\frac{n\pi}{a}x\right)\Psi(x,0)dx
\end{equation*}

\ptitle{Expectation Values}
\noindent\begin{align*}
    \langle \varphi_n|x^2\varphi_m\rangle & = \frac{2a^2}{\pi^2}\left(\frac{{(-1)}^{n-m}}{{(n-m)}^2}-\frac{{(-1)}^{n+m}}{{(n+m)}^2}\right) \\
    \langle \varphi_n|x^2\varphi_n\rangle & = a^2\left(\frac{1}{3} -\frac{1}{2{(n\pi)}^2}\right)
\end{align*}


\subsection{Harmonic Oscillator}

The Harmonic Oscillator differs from the Infinite Square Well by the Potential Energy function $V(x,t)$ which is given by the following quadratic function:

\begin{equation*}
    V(x) = \frac{1}{2}k x^2 = \frac{1}{2}m \omega^2 x^2 \qquad \text{with} \quad \omega = \sqrt{\frac{k}{m}}
\end{equation*}
The mechanical equivalent would be a frictionless spring and mass system with spring constant $k_s$.

\newpar{}

The TISE becomes
\noindent\begin{align*}
    \widehat{H}\psi                                                          & = E\psi \qquad \Leftrightarrow \\
    \Bigl[-\frac{\hbar^2}{2m}\frac{d^2}{dx^2}+\widehat{V}(x)\Bigr]\psi       & = E\psi \qquad \Leftrightarrow \\
    \frac{-\hbar^2}{2m}\frac{d^2\psi}{dx^2} + \frac{1}{2}m \omega^2 x^2 \psi & = E\psi
\end{align*}

\ptitle{Ladder Operators}

To solve this TISE one can use the \textbf{rising and lowering operators}: (for details on operators see~\ref{comm})
\begin{align*}
    \widehat{a}_{+} & = \frac{1}{\sqrt{2\hbar m \omega}}\left(-i\widehat{p}+m\omega\widehat{x}\right) &  & \text{raising operator}  \\
    \widehat{a}_{-} & = \frac{1}{\sqrt{2\hbar m \omega}}\left(+i\widehat{p}+m\omega\widehat{x}\right) &  & \text{lowering operator}
\end{align*}

\ptitle{Hamiltonian}

The Hamiltonian Operator can now be written as
\noindent\begin{align*}
    \widehat{H} & =\hbar\omega\left(\widehat{a}_{-}\widehat{a}_{+}-\frac{1}{2}\right) \\
                & =\hbar\omega\left(\widehat{a}_{+}\widehat{a}_{-}+\frac{1}{2}\right)
\end{align*}
from which
\noindent\begin{gather*}
    \widehat{a}_{-}\widehat{a}_{+}=\frac{1}{\hbar\omega}\widehat{H}+\frac{1}{2} \quad\quad \widehat{a}_{+}\widehat{a}_{-}=\frac{1}{\hbar\omega}\widehat{H}-\frac{1}{2}\\
    \left[\widehat{a}_{-},\widehat{a}_{+}\right] = 1
\end{gather*}

\ptitle{TISE}

Applied to the TISE we get
\begin{align*}
    \widehat{H}(\widehat{a}_{+}\psi_n) & = (E_n+\hbar\omega)\widehat{a}_{+}\psi_n \\
    \widehat{H}(\widehat{a}_{-}\psi_n) & = (E_n-\hbar\omega)\widehat{a}_{-}\psi_n
\end{align*}
This implies, if $\psi_n$ are solutions to the TISE with energy $E_n$, then $\widehat{a}_{+}\psi_n$/$\widehat{a}_{-}\psi_n$ are also solutions but with one more/less quantum of energy.
\newpar{}
At the bottom of the energy potential function the steady state is given by
\begin{equation*}
    \widehat{a}_{-}\psi_0 = 0\psi_0 = 0
\end{equation*}
Which is a simple ODE that results in the solution for the fist state (one level above having no energy at all).
\begin{equation*}
    \psi_0 = {\left(\frac{m\omega}{\pi\hbar}\right)}^{\frac{1}{4}}\exp\left(\frac{-m\omega}{2\hbar}x^2\right)
\end{equation*}
To get $\psi_n$, $\widehat{a}_{+}$ can be applied $n$-times.
\begin{equation*}
    \psi_n = \frac{1}{\sqrt{n!}}{\left(\widehat{a}_{+}\right)}^n \psi_0
\end{equation*}
with
\begin{align*}
    E_0 & = \frac{1}{2}\hbar\omega     \\
    E_n & = (n+\frac{1}{2})\hbar\omega
\end{align*}

For an arbitrary state $n$ the next state is given by
\begin{align*}
    \widehat{a}_{+}\psi_n & = \sqrt{n+1}\psi_{n+1} \\
    \widehat{a}_{-}\psi_n & = \sqrt{n}\psi_{n-1}
\end{align*}

\includegraphics[width=\linewidth]{harmonic_oscillator.png}

\ptitle{Hermite Polynomial Form}

\noindent\begin{align*}
    \psi_n(x) & ={\left(\frac{m\omega}{\pi\hbar}\right)}^{\frac{1}{4}}\frac{1}{\sqrt{2^n n!}}H_n(\xi)\exp\left(-\frac{\xi^2}{2}\right) \\
    \xi       & =\sqrt{\frac{m\omega}{\hbar}}x
\end{align*}
{\tiny\noindent\begin{align*}
    H_1 & = 2x                                         &  & \text{odd}  \\
    H_2 & = 4x^2-2                                     &  & \text{even} \\
    H_3 & = 8x^3-12x                                   &  & \text{odd}  \\
    H_4 & = 16x^4-48x^2+12                             &  & \text{even} \\
    H_n & = {(-1)}^n e^{x^2} \frac{d^n}{dx^n} e^{-x^2}
\end{align*}
}

\ptitle{Remarks:}
\begin{itemize}
    \item The energy is quantized.
    \item $n$ is the number of energy quanta in the oscillator
    \item $\widehat{N} \equiv \widehat{a}_{+}\widehat{a}_{-} = $ number operator because $\left<N\right> = n$ for $\psi_n$
    \item The lowest state $\psi_0$ has the \textbf{zero point energy}. Even if all the energy is removed from the system (zero Kelvin) the oscillator still has this zero point energy.
    \item Solutions alter between even and odd because the potential is \textbf{symmetric}.
    \item $\psi_{n+1}$ has one node more than $\psi_n$
    \item $\psi_n$ are mutually orthogonal + complete (form a basis).
    \item Soft boundary conditions - finite potential at the boundary/barrier.
    \item Outside the potential the particle has a potential energy higher than its potential energy for zero velocity. Hence, it must have negative kinetic energy outside the potential parabola.
    \item The spacing between two energy levels spreads out for large $k$ as $\omega=\sqrt{\frac{k}{m}}$.
\end{itemize}

\ptitle{Specific Expectation Values}
\noindent\begin{align*}
    \left\langle x \right\rangle _n   & = 0                                                                                                    & \forall n \in \mathbb{N} \\
    \left\langle x^2 \right\rangle _n & = \left(n+\frac{1}{2}\right)\frac{\hbar}{m\omega}                                                      & \forall n \in \mathbb{N} \\
    \left\langle p \right\rangle _n   & = 0                                                                                                    & \forall n \in \mathbb{N} \\
    \left\langle p^2 \right\rangle _n & = \left(n+\frac{1}{2}\right)\hbar m\omega                                                              & \forall n \in \mathbb{N} \\
    \left\langle T \right\rangle _n   & = \frac{\left\langle p^2 \right\rangle}{2m} =  \frac{\hbar \omega}{2}\left(n+\frac{1}{2}\right)                                   \\
    \left\langle V \right\rangle _n   & = \frac{m\omega^2}{2}\left\langle x^2 \right\rangle = \frac{\hbar \omega}{2}\left(n+\frac{1}{2}\right)
\end{align*}
Note that the energy $E_n$ is split up equally into kinetic and potential energy $T,V$

\subsection{Finite Potential}
Another way of modeling the boundary conditions through the finite potentials $V(x) < \infty$.

For any potential, the solution to the SE will either be a combination of \textit{scattering} and \textit{bound states} depending on the energy of the particle:
\noindent\begin{equation*}
    \begin{cases}
        E > V(\pm \infty) & \text{scattering state} \\
        E < V(\pm \infty) & \text{bound state}
    \end{cases}
\end{equation*}
\textbf{Remarks}

\begin{itemize}
    \item The total energy of the particle is conserved i.e.\ if the potential $V$ increases, the wavelength $\lambda$ (and so the velocity!) decreases and vice versa.
          \noindent\begin{equation*}
              k=\sqrt{\frac{2m(E-V_0)}{\hbar^2}} = \frac{2\pi}{\lambda}
          \end{equation*}
    \item If a particle penetrates into a potential higher than its own energy level it will take negative kinetic energy to conserve energy.
\end{itemize}

\ptitle{Solving TISE}
\begin{enumerate}
    \item Split up into regions
    \item Solve TISE for regions
          \begin{itemize}
              \item $V=V_i$:
                    \noindent\begin{equation*}
                        -\frac{\hbar^2}{2m}\frac{d^2}{dx^2} \psi + V_i\psi=E\psi
                    \end{equation*}
              \item $V=\infty$: no solution
          \end{itemize}
    \item match solutions at interfaces between regions using boundary conditions:
          \begin{itemize}
              \item $\psi(x)$ continous and finite
              \item $\frac{d\psi}{dx}$ continous
          \end{itemize}
\end{enumerate}

\subsubsection{Finite Potential Step}
\begin{center}
    \includegraphics[width = .5\linewidth]{fin_pot_step.png}
\end{center}
\begin{itemize}
    \item $\textcolor{red}{E}>V_0$: transmission and reflection (increased $\lambda$)
    \item $0<\textcolor{blue}{E}<V_0$: reflection and penetration (tunneling) into barrier with exponential decay
    \item $E<0$: no physical solution
\end{itemize}

\subsubsection{Finite Potential Well}
\begin{center}
    \includegraphics[width = .5\linewidth]{fin_pot_well.png}
\end{center}
\begin{itemize}
    \item $\textcolor{red}{E}>V_0$: transmission and reflection. Constructive/destructive interference, decreased $\lambda$ over well.
    \item $0<\textcolor{blue}{E}<V_0$: at least one bound state with penetration into barrier
    \item $E<0$: no physical solution
\end{itemize}

\subsubsection{Finite Potential Barrier}
\begin{center}
    \includegraphics[width = .5\linewidth]{fin_pot_barr.png}
\end{center}
\begin{itemize}
    \item $\textcolor{red}{E}>V_0$: transmission and reflection (increased $\lambda$ over step)
    \item $0<\textcolor{blue}{E}<V_0$: reflection and penetration (tunneling) into barrier with exponentially decaying probability of transmission
    \item $E<0$: no physical solution
\end{itemize}

\subsubsection{Tunneling}
The probability that a particle with energy $0<E<V_0$ is transmitted through a tall and wide barrier is exponentially decreasing with the thickness $a$ (which is why we can't run through walls) and the difference in energy $\sqrt{V_0-E}$. The transmission coefficient $T$ is given by:
\noindent\begin{equation*}
    T\approx\frac{16E(V_0-E)}{V_0^2}\exp\Biggl[-4 \underbrace{\frac{\sqrt{2m(V_0-E)}}{\hbar}}_{k} a\Biggr]
\end{equation*}

\subsection{3D Schrödinger Equation}
The TDSE for a three dimensional system is given by
\begin{equation*}
    i\hbar\frac{\partial\Psi}{\partial t} = \frac{-\hbar^2}{2m}\nabla^2\Psi + \widehat{V}\Psi
\end{equation*}
which is derived from
\begin{align*}
    \widehat{H} & = \frac{-\hbar^2}{2m}\nabla^2 + \widehat{V}                                                  \\
    \widehat{H} & = \frac{1}{2m}\left(\widehat{p}_x^2 + \widehat{p}_y^2 +\widehat{p}_z^2 \right) + \widehat{V}
\end{align*}
with
\begin{gather*}
    \nabla^2=\frac{\partial^2}{\partial x^2}+\frac{\partial^2}{\partial y^2}+\frac{\partial^2}{\partial z^2} \\
    \widehat{p}_d = \frac{\hbar}{i}\frac{\partial}{\partial d}
\end{gather*}

If $\widehat{V}$ is \textbf{time-independent} the stationary states are given by
\begin{equation*}
    \Psi_n(\mathbf{r},t)=\psi_n(\mathbf{r})\:e^{-iE_n t/\hbar}
\end{equation*}
where $\psi_n(\mathbf{r})$ are solutions to the TISE
\begin{equation*}
    -\frac{\hbar^2}{2m}\nabla^2\psi_n + \widehat{V}\psi_n = E_n \psi_n
\end{equation*}
and $\Psi$ is normalized
\noindent\begin{equation*}
    \int|\Psi|^2 d^3\mathbf{r}=1
\end{equation*}

\subsubsection{Spherical Symmetry}\label{3dSE_spherical_symmetry}
For spherical symmetric problems (like the hydrogen atom~\ref{H-atom}) it is best to transform the 3D TDSE from cartesian to polar coordinates. Then one can substitute $\nabla^2(r, \theta, \phi)$ in $\widehat{H}$ and separate the partial differential equation $\widehat{H}\psi = E\psi$ by variables.
\renewcommand{\arraystretch}{0.7}
\setlength{\oldtabcolsep}{\tabcolsep}\setlength\tabcolsep{0pt}
\begin{equation*}
    \begin{matrix}
                            &          & R(r)            &          &                \\
                            & \nearrow &                 &          &                \\
        \psi(r,\theta,\phi) &          &                 &          & \Theta(\theta) \\
                            & \searrow &                 & \nearrow &                \\
                            &          & Y(\theta, \phi) &          &                \\
                            &          &                 & \searrow &                \\
                            &          &                 &          & \Phi(\phi)
    \end{matrix}
\end{equation*}
\renewcommand{\arraystretch}{1}
\setlength\tabcolsep{\oldtabcolsep}

This results in three separate ODEs for each variable $R(r)$, $\Theta(\theta)$ and $\Phi(\phi)$.
\begin{equation*}
    \psi(r,\theta,\phi) = R(r)\:Y(\theta, \phi) = R(r)\:\Theta(\theta)\:\Phi(\phi)
\end{equation*}

\newpar{}
\ptitle{Azimuthal Angle Solution} $\Phi$

The solution to the azimuthal angle ODE is given by
\begin{equation*}
    \Phi(\phi)=e^{i m_\ell \phi}
\end{equation*}
where $m_\ell$ is the \textbf{magnetic quantum number}
\begin{equation*}
    m_\ell \in \mathbb{Z}
\end{equation*}

\newpar{}
\ptitle{Polar Angle Solution} $\Theta$

The solution to the polar angle ODE is given by
\begin{equation*}
    \Theta(\theta) = A P_{\ell}^{m_\ell}(\cos(\theta))
\end{equation*}
where $A$ is a constant, $\ell$ is the \textbf{azimuthal quantum number} and $P_{\ell}^{m_\ell}(x)$ are the \textbf{associated Legendre functions}.
\begin{equation*}
    \ell \in \mathbb{N}_0 \quad \text{and} \quad |m_\ell| \leq \ell
\end{equation*}

\begin{tikzpicture}

    \begin{axis}[%
            name=plot1,
            width=0.4\linewidth,
            ymin=-0.5, 
            ymax=0]
    \end{axis}

    \begin{axis}[%
            name=plot2,
            width=0.4\linewidth,
            at=(plot1.right of south east), anchor=left of south west,
            ymin=-6.5, ymax=0.2]
    \end{axis}

    \begin{axis}[%
            name=plot3,
            width=0.4\linewidth,
            at=(plot2.right of south east), anchor=left of south west,
            ymin=-6.5, ymax=0.2]
    \end{axis}

\end{tikzpicture}

\newpar{}
\ptitle{Angular Solutions} $\mathbf{Y}$

The two angular solutions combined are called \textbf{spherical harmonics}.
\begin{equation*}
    Y_{\ell}^{m_\ell}(\theta, \phi) = \epsilon\sqrt{\frac{2\ell+1}{4\pi}\frac{(\ell-|m_\ell|)!}{(\ell+|m_\ell|)!}}\:e^{im_\ell\phi}\:P_{\ell}^{m_\ell}(\cos(\theta))
\end{equation*}
where
\begin{equation*}
    \epsilon = \begin{cases}
        {(-1)}^{m_\ell} & m_\ell > 0    \\
        1               & m_\ell \leq 0
    \end{cases}
\end{equation*}

The spherical harmonics are normalized and orthogonal.

\newpar{}
\ptitle{Radial Equation} $\mathbf{R}$

Due to the spherical symmetry the radial equation becomes
\begin{equation*}
    \frac{-\hbar^2}{2m}\:\frac{d^2u}{dr^2}+\underbrace{\left[V(r)+\frac{-\hbar^2}{2m}\frac{\ell(\ell+1)}{r^2}\right]}_{V_{\text{eff}}}u = Eu
\end{equation*}
with
\begin{equation*}
    u(r) = rR(r)
\end{equation*}
Where $V(r)$ is the potential energy function which depends on the system. A solution of the radial equation of the hydrogen atom is given in Section~\ref{H-atom}.

\newpar{}
\ptitle{Remark on Spherical Coordinates}
\begin{align*}
    \theta & = \text{polar angle}     \\
    \phi   & = \text{azimuthal angle} \\
    r      & = \text{radius}
\end{align*}
\begin{align*}
    \mathbf{\nabla} & =\mathbf{u}_{r}\frac{\partial}{\partial r}+\frac{\mathbf{u}_{\theta}}{r}\frac{\partial}{\partial\theta}+\frac{\mathbf{u}_{\phi}}{r\sin\theta}\frac{\partial}{\partial\phi}                                      \\
    \nabla^2        & =\frac{1}{r^2}\frac{\partial}{\partial r}\left(r^2\frac{\partial}{\partial r}\right) + \frac{1}{r^2\sin(\theta)}\frac{\partial}{\partial\theta}\left(\sin(\theta)\frac{\partial}{\partial \theta}\right) \ldots \\
                    & + \frac{1}{r^2\sin^2(\theta)}\left(\frac{\partial^2}{\partial \phi^2}\right)
\end{align*}
\begin{equation*}
    d^3\mathbf{r} = dx\:dy\:dz = r^2\sin(\theta)dr\:d\theta\:d\phi
\end{equation*}

\textbf{Normalization}

\noindent\begin{equation*}
    \int_0^{2\pi}\int_0^{\pi} |Y_\ell^{m_\ell}|^2 \sin\theta\;d\theta d\phi \overset{!}{=} 1
\end{equation*}

\textbf{Orthogonality}

\noindent\begin{equation*}
    \int_0^{2\pi}\int_0^{\pi} {(Y_\ell^{m_\ell})}^{(a)}{(Y_\ell^{m_\ell})}^{(b)} \sin\theta\;d\theta d\phi \overset{!}{=} \begin{cases}
        1 & a=b     \\
        0 & a\neq b
    \end{cases}
\end{equation*}

\textbf{Remark}

Explicit solutions for the first few solutions of $Y_l^{m_l}$ can be found in the \textit{Useful Info Sheet}.

\subsection{Hydrogen Atom}\label{H-atom}
The potential energy function, i.e. $\widehat{V}(\mathbf{r})$, is \textbf{spherically symmetric} with respect to the nucleus. Therefor the SE in polar coordinates, stated in Section~\ref{3dSE_spherical_symmetry} can be used.
\newpar{}
The Coulombic attraction between the electron and the proton gives the potential energy function
\begin{equation*}
    V(r) = \frac{-e^2}{4\pi\epsilon_0}\frac{1}{r}
\end{equation*}

\ptitle{Radial Solution}

The solution to the radial part of the 3D SE becomes
\begin{align*}
    R_{n\ell}(r) = & \sqrt{{\left(\frac{2}{na}\right)}^3\frac{(n-\ell-1)!}{2n{[(n+\ell)!]}^3}} \ldots                    \\
                   & \ldots\exp\left(\frac{-r}{na}\right){\left(\frac{2r}{na}\right)}^\ell L_{n-\ell-1}^{2\ell+1}(2r/na)
\end{align*}
where $L_q^p (x)$ are the \textbf{associated Laguerre functions}, $n$ is the \textbf{principal quantum number} and $a$ is the Bohr radius.
\begin{equation*}
    a = \frac{4\pi\epsilon_0\hbar^2}{m_e e^2}=0.529 \cdot 10^{-10}m
\end{equation*}
where $m_e\simeq m$

\ptitle{Final Solution}

Combining the radial and the angular solution we get
\begin{equation*}
    \psi_{n\ell m_\ell}(r,\theta,\phi) = R_{n\ell}(r)\:Y_\ell^{m_\ell}(\theta, \phi)
\end{equation*}

where the quantum numbers are in the ranges of

\newpar{}
\textbf{principal quantum number} $n$
\noindent\begin{equation*}
    n = 1, 2, 3, \ldots
\end{equation*}
\begin{itemize}
    \item $n$ defines the energy level of an electron and thus the size of the electron cloud and the energy associated with the electrons orbit.
\end{itemize}

\newpar{}
\textbf{azimuthal quantum number} $\ell$
\noindent\begin{equation*}
    \ell = 0, 1, 2, \ldots , n-1
\end{equation*}
\begin{itemize}
    \item $\ell$ determines the shape of the electrons orbit ($\ell$ is the number of angular nodes).
    \item $\ell$ is directly related to the orbital angular momentum.
    \item $n-\ell-1$ is the number of radial nodes
\end{itemize}

\newpar{}
\textbf{magnetic quantum number} $m_\ell$
\noindent\begin{equation*}
    m_\ell =-\ell, -\ell+1, \ldots , \ell-1, \ell
\end{equation*}
\begin{itemize}
    \item $m_l$ describes the orientation of the angular momentum relative to an external magnetic field.
    \item $m_l$ is responsible for the magnetic splitting of spectral lines i.e.\ the \textit{Zeeman effect}
\end{itemize}

\begin{center}
    \includegraphics[width=\linewidth]{orbit_plots/hydrogen_orbit.jpg}
\end{center}

\newpar{}
\ptitle{Energies}

The Bohr formula gives the orbit energies of an H-atom.
\begin{equation*}
    E_n = -\left[\frac{m}{2\hbar^2}{\left(\frac{e^2}{4\pi\epsilon_0}\right)}^2\right]\frac{1}{n^2} = \frac{E_1}{n^2}
\end{equation*}
with $n \in \mathcal{N}$ and $E_1 = -13.6eV$

This is true for all \textit{hydrogenic} atoms. Note that the energy of the electronic level only depends on $n$.

\newpar{}
\ptitle{Degeneracy}

States are called \textit{degenerate} if they share their energy levels.
Considering the hydrogen atom with a given $n$ (except the ground state $n=1$), all states with $\ell=0,1,\ldots, n-1$ and $m_\ell=-\ell, \ldots, \ell$ are degenerate because they share their energy $E_n$.
\begin{center}
    \includegraphics[width=0.5\linewidth]{hydrogen_degeneracy.png}
\end{center}

\subsection{Cubical Infinite Square Well}
For the potential
\noindent\begin{equation*}
    V(x,y,z)=\begin{cases}0,&\text{if }x,y,z\text{ are all between }0\text{ and }a\\\infty,&\text{otherwise}\end{cases}
\end{equation*}
the stationary states $\Psi_n$ are given by

\noindent\begin{equation*}
    \Psi \left(x,y,z\right)={\left(\frac{2}{a}\right)}^{\frac{3}{2}} \sin\left(\frac{n_{x}\pi}{a}x\right)\sin\left(\frac{n_{y}\pi}{a}y\right)\sin\left(\frac{n_{z}\pi}{a}z\right)
\end{equation*}
and the corresponding energies are
\noindent\begin{equation*}
    E_n = \frac{\hbar^{2}}{2m} \frac{\pi^{2}}{a^{2}} \underbrace{\left(n_{x}^{2}+n_{y}^{2}+n_{z}^{2}\right)}_{n}
\end{equation*}

\section{Angular Momentum and Spin}
\subsection{Angular Momentum}

\ptitle{Operators}

From classical mechanics we derive
\begin{equation*}
    \widehat{\mathbf{L}}=
    \begin{pmatrix}
        \widehat{L}_x \\
        \widehat{L}_y \\
        \widehat{L}_z
    \end{pmatrix}
    =\widehat{\mathbf{r}}\times\widehat{\mathbf{p}}
    =
    \begin{pmatrix}
        y\widehat{p}_z-z\widehat{p}_y \\
        z\widehat{p}_x-x\widehat{p}_z \\
        x\widehat{p}_y-y\widehat{p}_x
    \end{pmatrix}
    =\frac{\hbar}{i}(\mathbf{\widehat{r}}\times\widehat{\mathbf{\nabla}})
\end{equation*}
with
\begin{equation*}
    |\mathbf{L}|=\sqrt{L_{x}^{2}+L_{y}^{2}+L_{z}^{2}}
\end{equation*}

\ptitle{Remarks}

\begin{itemize}
    \item $\widehat{L}_x, \widehat{L}_y, \widehat{L}_z$ are incompatible
    \item $\widehat{L}^2$ and $\widehat{L}_x, \widehat{L}_y, \widehat{L}_z$ are compatible
    \item Various angular momentum commutators can be found in the appendix:\ \ref{comm_am_sp}
\end{itemize}

\subsubsection{Ladder Operators for Angular Momentum}

Similar to the harmonic oscillator one can find the \textbf{ladder operators}
\begin{equation*}
    \widehat{L}_{\pm}=\widehat{L}_x\pm i \widehat{L}_y
\end{equation*}
which increase or lower the eigenvalue of $\widehat{L}_z$ by $\hbar$. These operators can be used to find the eigenvalues of angular momentum.

\subsubsection{Eigenfunctions}

As they are compatible (share eigenfunctions), we choose to find the eigenfunctions for $\widehat{L^2}$ and $\widehat{L_z}$
\begin{align*}
    \widehat{L}^2 f_{\ell}^{m_l} & =\hbar^{2}\ell (\ell+1) f_{\ell}^{m_l} \\
    \widehat{L}_z f_{\ell}^{m_l} & =\hbar mf_{\ell}^{m_l}
\end{align*}
with quantum numbers
\begin{align*}
    \ell & =0, 1/2, 1, 3/2,\ldots                \\
    m    & =-\ell, -\ell+1, \ldots, \ell-1, \ell
\end{align*}
and the orthogonal \textbf{eigenfunctions}
\begin{equation*}
    f_{\ell}^{m_\ell}=Y_{\ell}^{m_\ell}
\end{equation*}
i.e.\ the spherical harmonics.

\newpar{}
\ptitle{Remarks}
\begin{itemize}
    \item For each $\ell$ we have $2\ell+1$ values for $m_\ell$
    \item The eigenvalues are found by using the ladder method
    \item $Y_{\ell}^{m_\ell}$ are eigenfunctions of hermitian operators ($\widehat{L}^2$, $\widehat{L}_{z}$) and have \textbf{different eigenvalues}. Therefore, they are \textbf{orthogonal}.
\end{itemize}

\ptitle{Derivation}

Using $\nabla$ in spherical coordinates we get
\begin{equation*}
    \widehat{\mathbf{L}}=\frac{\hbar}{i}\Bigg(\mathbf{u}_{\phi}\frac{\partial}{\partial\theta}-\mathbf{u}_{\theta}\frac{1}{\sin\theta}\frac{\partial}{\partial\phi}\Bigg)
\end{equation*}
and
\begin{align*}
    \widehat{L}_{x} & =-i\hbar\left(-\sin\phi\frac{\partial}{\partial\theta}-\cos\phi\cot\theta\frac{\partial}{\partial\phi}\right)                                                                        \\
    \widehat{L}_{y} & =-i\hbar\left(+\cos\phi\frac{\partial}{\partial\theta}-\sin\phi\cot\theta\frac{\partial}{\partial\phi}\right)                                                                        \\
    \widehat{L}_{z} & =\frac{\hbar}{i}\frac{\partial}{\partial\phi}                                                                                                                                        \\
    \widehat{L}^{2} & =-\hbar^{2}\left[\frac{1}{\sin\theta}\frac{\partial}{\partial\theta}(\sin\theta\frac{\partial}{\partial\theta})+\frac{1}{\sin^{2}\theta}\frac{\partial^{2}}{\partial\phi^{2}}\right]
\end{align*}
where combining the $\widehat{L}_{i}$ yields $\widehat{L}^{2}$. Applying $\widehat{L}_{z}$ and $\widehat{L}^{2}$ to the eigenvalue equations we find that the equations become equivalent to the \textbf{angular equation} from the hydrogen atom and hence, have the same eigenfunctions.

\subsubsection{Physical Interpretation}
% As derived for the hydrogen atom, a single electron orbiting around a nucleus is described by $\psi_{n\ell m_\ell}(r,\theta,\phi)$ (having three quantum numbers $n,l,m_l$). 
The electron's angular momentum is described by the quantum numbers $l,m_l$. Given the eigenvalues of $\widehat{L}^2$ and $\widehat{L}_z$ we find the expectations
\begin{align*}
    |\mathbf{L}| & =\hbar\sqrt{\ell(\ell+1)} \\
    L_{z}        & =m_{\ell}\hbar
\end{align*}
as $m_{\ell}=-\ell,-\ell+1,\dots,\ell-1,\ell$ we conclude that $L_{z}$
\begin{itemize}
    % \item Can only take discrete values
    \item $L_z < |\mathbf{L}|$: $\mathbf{L}$ can never point directly along the $z$ axis
\end{itemize}

\newpar{}
As $L_x, L_y, L_z$ are incompatible we conclude that
\begin{itemize}
    \item We can determine $|\mathbf{L}|$ and $L_{z}$ simultaneously.
    \item But then have uncertainty in $L_x, L_y$.
\end{itemize}
Given a certain $\ell$, angular momentum can be described by a cone for each $m_\ell$
\begin{center}
    \includegraphics[width = 0.6\linewidth]{angular_momentum.png}
\end{center}

\subsection{Spin}
Similar to the earth having an angular momentum w.r.t.\ the sun but also itself, an electron has an angular momentum w.r.t.\ the nucleus and itself.

\ptitle{Operators}

$\widehat{\mathbf{S}}$ follows the same rules as $\widehat{\mathbf{L}}$ (see appendix\ \ref{comm_am_sp})

\ptitle{Remarks}

\begin{itemize}
    \item Electrons have an intrinsic spin.
    \item An electron \textbf{acts as if} it was rotating on an axis but it is not as it is zero-dimensional.
    \item This can easily be imagined as a spinning ball that is not a ball and does not spin.
\end{itemize}


\subsubsection{Eigenvalue Equation for Spin}
For a QM particle (not necessarily an electron) with spin one has
\begin{align*}
    \widehat{S}^2f_{s}^{m_s}   & =\hbar^{2}s (s+1) f_{s}^{m_s} \\
    \widehat{S}_{z}f_{s}^{m_s} & =\hbar m_s f_{s}^{m_s}
\end{align*}
where
\begin{align*}
    s   & =0,\frac{1}{2},1,\frac{3}{2},\dots \\
    m_s & =-s, -s+1,\dots, s-1, s
\end{align*}
i.e.\ angular momenta are restricted to integer and half-integer values.

\ptitle{Set Value for $\mathbf{s}$}

In contrast to angular momentum of a QM particle where $l$ could take arbitrary integer values, $s$ takes \textbf{a set value} for each particle.

\ptitle{Remarks}

\begin{itemize}
    \item For electrons we have $s=\frac{1}{2}$, for photons $s=1$
    \item Electrons have
          \begin{itemize}
              \item $m_s=\frac{1}{2}$ (spin up)
              \item $m_s=-\frac{1}{2}$ (spin down)
              \item Therefore, only two eigenstates $f_{\frac{1}{2}}^{\pm \frac{1}{2}}$
          \end{itemize}
\end{itemize}

\subsubsection{The State of Spin of an Electron}
As eigenstates for spin are not functions of $r,\theta, \phi$ (particle is zero-dimensional) one cannot write them as mathematical formulae for $f_{s}^{m_s}$. Instead, one uses Dirac notation:
\begin{equation*}
    f_{s}^{m_{s}}\rightarrow|s,m_{s}\rangle
\end{equation*}

\ptitle{Spin of an Electron}

For an electron with $s=\frac{1}{2}$, the eigenfunctions can be described by ket and bra as:
\begin{align*}
    f_{\frac{1}{2}}^{+\frac{1}{2}} & \rightarrow\left|\frac{1}{2},\quad +\frac{1}{2}\right>= \left|\uparrow\right>   \\
    f_{\frac{1}{2}}^{-\frac{1}{2}} & \rightarrow\left|\frac{1}{2},\quad-\frac{1}{2}\right> = \left|\downarrow\right>
\end{align*}
and one has
\begin{align*}
    |\mathbf{S}| & =\sqrt{\frac{1}{2}\left(\frac{1}{2}+1\right)}\hbar=\sqrt{\frac{3}{4}}\hbar \\
    S_z          & =\pm \frac{1}{2}\hbar
\end{align*}

\ptitle{General Spin State}

Spin can be in a general state

\begin{equation*}
    \left|\lambda\right>=a\left|\frac{1}{2},\quad +\frac{1}{2}\right>+b\left|\frac{1}{2},\quad -\frac{1}{2}\right>
\end{equation*}
where the probability of measuring a certain spin (assuming $\left|\lambda\right>$ \textbf{normalized}!) is given by $|a|^2$ for $\uparrow$ and $|b|^2$ for $\downarrow$ respectively.

\ptitle{Spin Operators as Matrices}

Setting the two eigenstates of the electron as basis vectors of a \textbf{two-dimensional} vector space we get

\begin{align*}
    \left|\frac{1}{2},\quad +\frac{1}{2}\right> & \rightarrow \left(\begin{matrix}
                                                                            1 \\
                                                                            0
                                                                        \end{matrix}\right) \\
    \left|\frac{1}{2},\quad -\frac{1}{2}\right> & \rightarrow \left(\begin{matrix}
                                                                            0 \\
                                                                            1
                                                                        \end{matrix}\right)
\end{align*}
Defining spin up and spin down \textbf{relative to z-axis} we get (by solving eigenvalue equation in Dirac notation):
\begin{align*}
    \widehat{S}^2   & \rightarrow \frac{3}{4}{\hbar}^2\begin{bmatrix}
                                                          1 & 0 \\
                                                          0 & 1
                                                      \end{bmatrix} &
    \widehat{S}_{z} & \rightarrow \frac{\hbar}{2}\begin{bmatrix}
                                                     1 & 0  \\
                                                     0 & -1
                                                 \end{bmatrix}       \\
    \widehat{S}_{x} & \rightarrow \frac{\hbar}{2}\begin{bmatrix}
                                                     0 & 1 \\
                                                     1 & 0
                                                 \end{bmatrix}      &
    \widehat{S}_{y} & \rightarrow \frac{\hbar}{2}\begin{bmatrix}
                                                     0 & -i \\
                                                     i & 0
                                                 \end{bmatrix}
\end{align*}

Note that
\begin{itemize}
    \item $\widehat{S}_{x}$, $\widehat{S}_{y}$ are non-diagonal i.e.\ the basis vectors are not eigenvectors of these operators.
    \item If they were diagonal in z-basis, $\widehat{S}_{x},\widehat{S}_{y},\widehat{S}_{z}$ would have the same eigenvectors (but is not the case as they are incompatible).
\end{itemize}

\ptitle{Ladder Operators for Spin}

The ladder operators have the effect
\begin{align*}
    \widehat{S}_{+}\left|\frac{1}{2},-\frac{1}{2}\right> & =\hbar\left|\frac{1}{2},+\frac{1}{2}\right> & \text{ (raises $S_z$ by $\hbar$)} \\
    \widehat{S}_{-}\left|\frac{1}{2},+\frac{1}{2}\right> & =\hbar\left|\frac{1}{2},-\frac{1}{2}\right> & \text{ (lowers $S_z$ by $\hbar$)}
\end{align*}
and are in z-axis basis given by the \textbf{non-hermitian} matrices
\begin{align*}
    \widehat{S}_{+} & \rightarrow \hbar\begin{bmatrix}
                                           0 & 1 \\
                                           0 & 0
                                       \end{bmatrix} \\
    \widehat{S}_{-} & \rightarrow \hbar\begin{bmatrix}
                                           0 & 0 \\
                                           1 & 0
                                       \end{bmatrix}
\end{align*}

