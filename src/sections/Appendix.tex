% TODO: check if new additions to appendix are in final summary
% TODO: add power series / taylor series

\section{Appendix}
\subsection{Notation}
The following notation is based on the book \textit{Introduction to Quantum Mechanics} by David J. Darrel
\begin{align*}
     & \lvert \alpha \rangle                                             &  & \text{Vector in Dirac notation}           \\
     & \alpha_i^{b_j}                                                    &  & \text{Vector element $i$ wrt.\ base } b_j \\
     & \widehat{T}                                                       &  & \text{Linear transformation/ Operator}    \\
     & \mathbf{T}                                                        &  & \text{Matrix}                             \\
     & \widetilde{\mathbf{T}}                                            &  & \text{Transpose}                          \\
     & \mathbf{T}^*                                                      &  & \text{Complex conjugate}                  \\
     & \mathbf{T}^\dagger = {\widetilde{\mathbf{T}}}^*                   &  & \text{Hermitian}                          \\
     & \mathbf{T}^{-1}                                                   &  & \text{Inverse}                            \\
     & \lbrack \mathbf{S},\mathbf{T} \rbrack = \mathbf{ST} - \mathbf{TS} &  & \text{Commutator}
\end{align*}

\subsection{Statistics}
Given a probability density $\rho$ the probability of obtaining $x$ in the interval $x+dx$ is given by $\rho dx$ and hence the probability of finding it in [$a,b$] is given by

\begin{equation*}
    P_{ab}=\int_{a}^{b} \rho(x)\, dx
\end{equation*}

The probability density $\rho$ is called \textbf{normalized} if

\begin{equation*}
    \int_{-\infty}^{\infty}\rho(x)dx = 1
\end{equation*}

The \textbf{expectation value} $\langle x \rangle$ can be calculated with

\begin{equation*}
    \langle x \rangle = \int_{-\infty}^{\infty} x\rho(x) dx \underbrace{{\langle \Psi|x\Psi\rangle} }_{\textsf{Dirac Notation}}
\end{equation*}

The \textbf{variance} ${\sigma_x}^2$ and \textbf{standard deviation} $\sigma_x$ is given by

\begin{equation*}
    {\sigma_x}^2 = \langle x^2 \rangle - {\langle x \rangle }^2
\end{equation*}

\ptitle{Standard Deviation of Observables}
\begin{align*}
    \sigma_{\widehat{Q}}^{2} & =\left\langle{\left(Q-\langle Q\rangle\right)}^{2}\right\rangle                                                                                                                                    \\
                             & =\left\langle\Psi\left|{\left(\widehat{Q}-q\right)}^{2}\Psi\right\rangle\stackrel{\text{hermitian}}{=}\left\langle\left(\widehat{Q}-q\right)\Psi\right|\left(\widehat{Q}-q\right)\Psi\right\rangle
\end{align*}

\subsection{Commutators}\label{comm}

\noindent\begin{equation*}
    \left[\widehat{A},\widehat{B}\right] = \widehat{A}\widehat{B} - \widehat{B}\widehat{A} = -\left[\widehat{B},\widehat{A}\right]
\end{equation*}
If $\widehat{A}$ and $\widehat{B}$ do not commute, $\left[\widehat{A},\widehat{B}\right] \neq 0$ and therefore the order of the operations matters.

\newpar{}
\ptitle{Rules}

\noindent\begin{align*}
    \left[\widehat{A}\widehat{B},\widehat{C}\right]  & =\widehat{A}\left[\widehat{B},\widehat{C}\right]+\left[\widehat{A},\widehat{C}\right]\widehat{B}     \\
    \left[\widehat{A}+\widehat{B},\widehat{C}\right] & = \left[\widehat{A},\widehat{C}\right]+\left[\widehat{B},\widehat{C}\right]                          \\
    \left[\widehat{A},\widehat{B}^2\right]           & = \left[\widehat{A},\widehat{B}\right] \widehat{B} - \widehat{B}\left[\widehat{B},\widehat{A}\right] \\
                                                     & = \left[\widehat{A},\widehat{B}\right] \widehat{B} + \widehat{B}\left[\widehat{A},\widehat{B}\right]
\end{align*}

\ptitle{Evaluate Commutators}

For some operators it can be useful to evaluate the commutator on a test function first, and then remove the test function in the end, e.g.:
\noindent\begin{align*}
    \left[\frac{d^2}{dx^2}, \widehat{x}\right] f(x) & = \frac{d^2}{dx^2} xf(x) - x\frac{d^2}{dx^2}f(x) \\
                                                    & = 2\frac{d}{dx} f(x)                             \\
    \left[\frac{d^2}{dx^2}, \widehat{x}\right]      & = 2\frac{d}{dx}
\end{align*}

\textbf{Remark}:

It is important to calculate $\widehat{A}\widehat{B}f - \widehat{B}\widehat{A}f$ and not $(\widehat{A}\widehat{B} - \widehat{B}\widehat{A})f$ directly.

\subsubsection{Useful Commutators}

\noindent\begin{align*}
    \left[c,\hat B\right]                        & =0  \\
    \left[\widehat{a}_{-},\widehat{a}_{+}\right] & = 1
\end{align*}

\newpar{}
\ptitle{Canonical Commutation Relations}

\noindent\begin{align*}
    \left[\widehat{x},\widehat{p}_x\right]    & = i\hbar  \\
    \left[\widehat{p}_x, \widehat{x}\right]   & = -i\hbar \\
    \left[\widehat{x}, \widehat{y}\right]     & = 0       \\
    \left[\widehat{x}, \widehat{p}_y\right]   & = 0       \\
    \left[\widehat{p}_x, \widehat{p}_y\right] & = 0
\end{align*}

\textbf{Remark} Works for all spatial dimensions.

\newpar{}
\ptitle{Other Momentum Commutators}
\noindent\begin{align*}
    \left[\widehat{x}^n,\widehat{p}\right]  & = i \hbar n x^{n-1}    \\
    \left[\widehat{x},\widehat{p}^2\right]  & = 2i\hbar\widehat{p}   \\
    \left[f(\widehat{x}),\widehat{p}\right] & = i \hbar\frac{df}{dx}
\end{align*}

\ptitle{Energy Commutators}
\noindent\begin{align*}
    \left[\widehat{V},\widehat{x}\right]  & =0                            &  & \text{Potential energy} \\
    \left[\widehat{H},\widehat{x}\right]  & =-\frac{i\hbar}{m}\widehat{p} &  & \text{Total energy}     \\
    \left[\widehat{H}, \widehat{V}\right] & \neq 0
\end{align*}

\ptitle{Angular Momentum Commutators}\label{comm_am_sp}
\noindent\begin{gather*}
    \left[\widehat{L}_{x},\widehat{L}_{y}\right] =i\hbar \widehat{L}_{z}, \qquad \left[\widehat{L}_{y},\widehat{L}_{z}\right]  =i\hbar \widehat{L}_{x}, \qquad \left[\widehat{L}_{z},\widehat{L}_{x}\right]  =i\hbar \widehat{L}_{y}      \\
    \left[\widehat{L}^{2},\widehat{L}_{x}\right] = \left[\widehat{L}^{2},\widehat{L}_{y}\right] = \left[\widehat{L}^{2},\widehat{L}_{z}\right] = 0\\
    \left[\widehat{L}^2, \mathbf{L}\right]=0
\end{gather*}

\newpar{}
\ptitle{Remarks}
\begin{itemize}
    \item For \textbf{spin}, the same rules hold as for angular momentum.
    \item Note: One can ``permute through'' the operators in the first row.
\end{itemize}

\subsection{Inner Product}\label{ssec:InnerProd}
\ptitle{Properties}
\noindent\begin{align*}
    \langle f|g \rangle         & := \int_{-\infty}^{\infty} f^* g\; d \mathbf{r} \in L_2 &  &                                   \\
    \langle \lambda f|g \rangle & =\lambda^* \langle f|g \rangle                          &  & \text{semilinear for scalar mul.} \\
    \langle f|\lambda g \rangle & =\lambda \langle f|g \rangle                                                                   \\
    \langle f+g|h \rangle       & =\langle f|h \rangle + \langle g|h \rangle              &  & \text{linear for addition}        \\
    \langle f|g+h \rangle       & =\langle f|g \rangle + \langle f|h \rangle                                                     \\
    \langle g|f \rangle         & = {\langle f|g \rangle}^*                               &  & \text{``hermitian''}              \\
    \langle f|f \rangle         & \ge 0                                                   &  & \text{positive definite}          \\
    \langle f|f \rangle         & = 0 \leftrightarrow f=0
\end{align*}

\ptitle{Schwarz-Inequality}
\begin{align*}
    |\langle f|g \rangle |^2                  & \leq \langle f|f \rangle \langle g|g \rangle                 \\
    \left|\int_{a}^{b}{f(x)}^{*}g(x)dx\right| & \leq \sqrt{\int_{a}^{b}|f(x)|^{2}dx\int_{a}^{b}|g(x)|^{2}dx}
\end{align*}

\subsection{Constants}
\noindent\begin{align*}
    h = 6.63 \cdot 10^{-34} \;\mathrm{J\cdot s}      &  &  & \text{Planck constant}    \\
    k_b =8.617 \cdot 10^{-5} \;\mathrm{\frac{eV}{K}} &  &  & \text{Boltzmann constant}
\end{align*}

\subsection{Important Laws from Classical Physics}
\textbf{Ideal Gas}\\
Ideal gas law:
\noindent\begin{align*}
    PV & =nRT=n k_B T                                              \\
    \text{with}                                                    \\
    n  & =\frac{N_{particles}}{N_A} \text{ ($n$: number of Moles)} \\
    R  & =N_A k_B \text{ universal gas constant}
\end{align*}
Distance between atoms:
\noindent\begin{align*}
    d & ={\left(\frac{V_{gas}}{N}\right)}^{1/3}
\end{align*}

\subsection{Useful Integrals}
\subsubsection{Exponentials}
\begin{footnotesize}
    \noindent\begin{align*}
         & \int e^{\lambda x}dx           &  & =\frac{1}{\lambda }e^{\lambda x}+C                                                 \\
         & \int a^{\lambda x}dx           &  & =\frac{1}{\lambda \cdot \ln(a)}a^{\lambda x}+C                                     \\
         & \int e^{\lambda x}\sin(ax+b)dx &  & =\frac{e^{\lambda x}}{a^2+\lambda ^2}\left(\lambda \sin(ax+b)-a\cos(ax+b)\right)+C \\
         & \int e^{\lambda x}\cos(ax+b)dx &  & =\frac{e^{\lambda x}}{a^2+\lambda ^2}\left(\lambda \cos(ax+b)+a\sin(ax+b)\right)+C \\
         & \int x \cdot e^{\lambda x}dx   &  & =(\frac{\lambda x-1}{\lambda ^2})\cdot e^{\lambda x}+C                             \\
         & \int x^2 \cdot e^{\lambda x}dx &  & =(\frac{\lambda ^2x^2-2\lambda x+2}{\lambda ^3})\cdot e^{\lambda x}                \\
         & \int x\cdot e^{x^2}dx          &  & =\frac{1}{2}\cdot e^{x^2}+C
    \end{align*}
\end{footnotesize}

\paragraph{Specific Intervals}
\begin{footnotesize}
    \noindent\begin{align*}
         & \int_0^{\infty} e^{-\lambda x}x^n dx                                               &  & =\frac{n!}{\lambda ^{n+1}},\quad \lambda >0                                       \\
         & \int_0^{\infty} e^{-\lambda x^2} dx                                                &  & =\frac{1}{2}\sqrt{\frac{\pi}{\lambda }},\quad \lambda >0                          \\
         & \int_{-\infty}^{\infty} e^{-\lambda x^2} dx                                        &  & =\sqrt{\frac{\pi}{\lambda }},\quad \lambda >0                                     \\
         & \int_{-\infty}^{\infty}e^{-ax^2}e^{-iwx} dx                                        &  & = \sqrt{\frac{\pi}{a}}e^{\frac{w^2}{4a}}                                          \\
         & \int_{-\infty}^{\infty}e^{-(ax^2+bx+c)}dx                                          &  & = \sqrt{\frac{\pi}{a}}e^{\frac{b^2}{4a}-c}                                        \\
         & \int_{-\infty}^{\infty}x^2e^{-\lambda x^2}dx                                       &  & = \frac{1}{2\lambda }\sqrt{\frac{\pi}{\lambda }},\quad \mathrm{Re}\{\lambda \} >0 \\
         & \int_0^\infty x^n e^{-ax}dx                                                        &  & =\frac{n!}{a^{n+1}},\quad n\in\mathbb{N}^+                                        \\
         & \int_{0}^{a} x\sin^2\left(\frac{n\pi}{x}x\right)\; dx                              &  & =\frac{a^2}{4}                                                                    \\
         & \int_0^a x^2\cdot\sin^2\left(\frac{n\pi}{a}x\right)dx                              &  & ={\left(\frac{a}{2\pi n}\right)}^3\left(\frac{4\pi^3n^3}{3}-2n\pi\right)          \\
         & \int_{0}^{a} x\sin\left(\frac{\pi}{x}x\right)\sin\left(\frac{2\pi}{x}x\right)\; dx &  & = -\frac{8a^2}{9\pi^2}
    \end{align*}
\end{footnotesize}

\subsection{Solution Table}

\subsubsection{Laguerre Polynomials}
\noindent\begin{equation*}
    L_q(x)\equiv\frac{e^x}{q!}{\left(\frac d{dx}\right)}^q\left(e^{-x}x^q\right)
\end{equation*}
\noindent\begin{align*}
    L_{0} & =1                                          \\
    L_{1} & =-x+1                                       \\
    L_{2} & =x^2-4x+2                                   \\
    L_{3} & =-x^3+9x^2-18x+6                            \\
    L_{4} & =x^4-16x^3+72x^2-96x+24                     \\
    L_{5} & =-x^5+25x^4-200x^3+600x^2-600x+120          \\
    L_{6} & =x^6-36x^5+450x^4-2400x^3+5400x^2-4320x+720
\end{align*}

\subsubsection{Associated Laguerre Polynomials}
\noindent\begin{equation*}
    L_{q}^{p}(x)\equiv{(-1)}^{p}{\left(\frac{d}{dx}\right)}^{p}L_{p+q}(x)
\end{equation*}
\noindent\begin{align*}
    L_0^0 & =1           & L_{0}^{2} & =2               \\
    L_1^0 & =-x+1        & L_1^2     & =-6x+18          \\
    L_2^0 & =x^2-4x+2    & L_2^2     & =12x^2-96x+144   \\
    L_0^1 & =1           & L_0^3     & =6               \\
    L_1^1 & =-2x+4       & L_1^3     & =-24x+96         \\
    L_2^1 & =3x^2-18x+18 & L_2^3     & =60x^2-600x+1200
\end{align*}

\subsubsection{Legendre Polynomials}
\noindent\begin{equation*}
    P_l(x)=\frac1{2^l\cdot l!}{\left(\frac d{dx}\right)}^l{\left(x^2-1\right)}^l
\end{equation*}
\noindent\begin{align*}
    P_{0} & =1                                \\
    P_{1} & =x                                \\
    P_{2} & =\frac{1}{2}(3x^{2}-1)            \\
    P_{3} & =\frac{1}{2}(5x^{3}-3x)           \\
    P_{4} & =\frac{1}{8}(35x^{4}-30x^{2}+3)   \\
    P_{5} & =\frac{1}{8}(63x^{5}-70x^{3}+15x)
\end{align*}

\subsubsection[Associated Legendre Functions for cos]{Associated Legendre Functions for $x=\cos\theta$}
\noindent\begin{equation*}
    P_l^m(x)={(1-x^2)}^{\frac{|m|}2}{\left(\frac d{dx}\right)}^{|m|}\cdot P_l(x)
\end{equation*}
\noindent\begin{align*}
    P_{0}^{0} & =1                       & P_{2}^{0} & =\frac{1}{2}(3 \cos^{2}\theta-1)             \\
    P_{1}^{1} & =-\sin\theta             & P_{3}^{3} & =-15 \sin\theta(1-\cos^{2}\theta)            \\
    P_{1}^{0} & =\cos\theta              & P_{3}^{2} & =15 \sin^{2}\theta \cos\theta                \\
    P_{2}^{2} & =3 \sin^{2}\theta        & P_{3}^{1} & =-\frac{3}{2}\sin\theta (5 \cos^{2}\theta-1) \\
    P_{2}^{1} & =-3 \sin\theta\cos\theta & P_{3}^{0} & =\frac{1}{2} (5 \cos^{3}\theta-3 \cos\theta)
\end{align*}

\subsubsection{Spherical Harmonics}
\noindent\begin{equation*}
    Y_{\ell}^{m_\ell}(\theta, \phi) = \epsilon\sqrt{\frac{2\ell+1}{4\pi}\frac{(\ell-|m_\ell|)!}{(\ell+|m_\ell|)!}}\:e^{im_\ell\phi}\:P_{\ell}^{m_\ell}(\cos(\theta))
\end{equation*}
\noindent\begin{align*}
    Y_0^0      & = {\left(\frac1{4\pi}\right)}^{\frac12}                                                   \\
    Y_1^0      & = {\left(\frac{3}{4\pi}\right)}^{\frac{1}{2}}\cos\theta                                   \\
    Y_1^{\pm1} & = \mp{\left(\frac{3}{8\pi}\right)}^{\frac{1}{2}}\sin\theta\cdot e^{\pm i\varphi}          \\
    Y_2^0      & = {\left(\frac{5}{16\pi}\right)}^{\frac{1}{2}}(3\cos^2\theta-1)                           \\
    Y_2^{\pm1} & = \mp{\left(\frac{15}{8\pi}\right)}^{\frac12}\sin\theta\cos\theta\cdot e^{\pm i\varphi}   \\
    Y_2^{\pm2} & = {\left(\frac{15}{32\pi}\right)}^{\frac12}\sin^2\theta\cdot e^{\pm2i\varphi}             \\
    Y_3^0      & = {(\frac7{16\pi})}^{\frac12}(5\cos^3\theta-3\cos\theta)                                  \\
    Y_3^{\pm1} & = \mp{\left(\frac{21}{64\pi}\right)}^{\frac12}\sin\theta(5\cos^2\theta-1)e^{\pm i\varphi} \\
    Y_3^{\pm2} & = {(\frac{105}{32\pi})}^{\frac12}\sin^2\theta\cos^2\theta\cdot e^{\pm2i\varphi}           \\
    Y_3^{\pm3} & = \mp{\left(\frac{35}{64\pi}\right)}^{\frac12}\sin^3\theta\cdot e^{\pm3i\varphi}
\end{align*}

\subsubsection{Radial Wave Functions for Hydrogen}
\noindent\begin{align*}
    R_{n\ell}(r) = & \sqrt{{\left(\frac{2}{na}\right)}^3\frac{(n-\ell-1)!}{2n{[(n+\ell)!]}^3}} \ldots                    \\
                   & \ldots\exp\left(\frac{-r}{na}\right){\left(\frac{2r}{na}\right)}^\ell L_{n-\ell-1}^{2\ell+1}(2r/na)
\end{align*}

\begin{align*}
     & R_{10} = 2a^{-3/2}\exp(-r/a)                                                                                                                              \\[1.5em]
     & R_{20} = \frac{1}{\sqrt{2}}a^{-3/2}\left(1-\frac12\frac ra\right)\exp(-r/2a)                                                                              \\
     & R_{21} = \frac{1}{2\sqrt{6}}a^{-3/2}\left(\frac ra\right)\exp(-r/2a)                                                                                      \\[1.5em]
     & R_{30} = \frac{2}{3\sqrt{3}}a^{-3/2}\left(1-\frac23\frac ra+\frac2{27}{\left(\frac ra\right)}^2\right)\exp(-r/3a)                                         \\
     & R_{31} = \frac{8}{27\sqrt{6}}a^{-3/2}\left(1-\frac16\frac ra\right)\left(\frac ra\right)\exp(-r/3a)                                                       \\
     & R_{32} = \frac{4}{81\sqrt{30}}a^{-3/2}{\left(\frac ra\right)}^2\exp(-r/3a)                                                                                \\[1.5em]
     & R_{40} = \frac{1}{4} a^{-3/2}\left(1-\frac34\frac ra+\frac18{\left(\frac ra\right)}^2-\frac1{192}{\left(\frac ra\right)}^3\right)\exp(-r/4a)              \\
     & R_{41} = \frac{5}{16\sqrt{15}}a^{-3/2}\left(1-\frac{1}{4}\frac{r}{a}+\frac{1}{80}{\left(\frac{r}{a}\right)}^{2}\right)\left(\frac{r}{a}\right)\exp(-r/4a) \\
     & R_{42} = \frac{1}{64\sqrt{5}}a^{-3/2}\left(1-\frac1{12}\frac ra\right){\left(\frac ra\right)}^2\exp(-r/4a)                                                \\
     & R_{43} = \frac{1}{768\sqrt{35}}a^{-3/2}{\left(\frac ra\right)}^3\exp(-r/4a)
\end{align*}