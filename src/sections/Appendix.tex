\section{Appendix}
\subsection{Notation}
The following notation is based on the book \textit{Introduction to Quantum Mechanics} by David J. Darrel
\begin{align*}
     & \lvert \alpha \rangle                                             &  & \text{Vector in Dirac notation}           \\
     & \alpha_i^{b_j}                                                    &  & \text{Vector element $i$ wrt.\ base } b_j \\
     & \widehat{T}                                                       &  & \text{Linear transformation/ Operator}    \\
     & \mathbf{T}                                                        &  & \text{Matrix}                             \\
     & \widetilde{\mathbf{T}}                                            &  & \text{Transpose}                          \\
     & \mathbf{T}^*                                                      &  & \text{Complex conjugate}                  \\
     & \mathbf{T}^\dagger = {\widetilde{\mathbf{T}}}^*                   &  & \text{Hermitian}                          \\
     & \mathbf{T}^{-1}                                                   &  & \text{Inverse}                            \\
     & \lbrack \mathbf{S},\mathbf{T} \rbrack = \mathbf{ST} - \mathbf{TS} &  & \text{Commutator}
\end{align*}

\subsection{Statistics}
Given a probability density $\rho$ the probability of obtaining $x$ in the interval $x+dx$ is given by $\rho dx$ and hence the probability of finding it in [$a,b$] is given by

\begin{equation*}
    P_{ab}=\int_{a}^{b} \rho(x)\, dx
\end{equation*}

The probability density $\rho$ is called \textbf{normalized} if

\begin{equation*}
    \int_{-\infty}^{\infty}\rho(x)dx = 1
\end{equation*}

The \textbf{expectation value} $\langle x \rangle$ can be calculated with

\begin{equation*}
    \langle x \rangle = \int_{-\infty}^{\infty} x\rho(x) dx \underbrace{{\langle \Psi|x\Psi\rangle} }_{\textsf{Dirac Notation}}
\end{equation*}

The \textbf{variance} ${\sigma_x}^2$ and \textbf{standard deviation} $\sigma_x$ is given by

\begin{equation*}
    {\sigma_x}^2 = \langle x^2 \rangle - {\langle x \rangle }^2
\end{equation*}

\ptitle{Standard Deviation of Observables}
\begin{align*}
    \sigma_{\widehat{Q}}^{2} & =\left\langle{\left(Q-\langle Q\rangle\right)}^{2}\right\rangle                                                                                                                                    \\
                             & =\left\langle\Psi\left|{\left(\widehat{Q}-q\right)}^{2}\Psi\right\rangle\stackrel{\text{hermitian}}{=}\left\langle\left(\widehat{Q}-q\right)\Psi\right|\left(\widehat{Q}-q\right)\Psi\right\rangle
\end{align*}

\subsection{Commutators}\label{comm}

\noindent\begin{equation*}
    \left[\widehat{A},\widehat{B}\right] = \widehat{A}\widehat{B} - \widehat{B}\widehat{A} = -\left[\widehat{B},\widehat{A}\right]
\end{equation*}
If $\widehat{A}$ and $\widehat{B}$ do not commute, $\left[\widehat{A},\widehat{B}\right] \neq 0$ and therefore the order of the operations matters.

\newpar{}

\ptitle{Canonical Commutation Relation}

\begin{align*}
    \left[\widehat{x},\widehat{p}_x\right]    & = i\hbar  \\
    \left[\widehat{p}_x, \widehat{x}\right]   & = -i\hbar \\
    \left[\widehat{x}, \widehat{y}\right]     & = 0       \\
    \left[\widehat{p}_x, \widehat{p}_y\right] & = 0
\end{align*}

\textbf{Remark} Works for all spatial dimensions.

\newpar{}
\ptitle{Rules}

\noindent\begin{align*}
    \left[\widehat{A}\widehat{B},\widehat{C}\right] & =\widehat{A}\left[\widehat{B},\widehat{C}\right]+\left[\widehat{A},\widehat{C}\right]\widehat{B}
\end{align*}


\ptitle{Useful Commutators}

\noindent\begin{align*}
    \left[c,\hat B\right]                        & =0                                                         \\
    \left[\widehat{a}_{-},\widehat{a}_{+}\right] & = 1                                                        \\
    \left[\widehat{x}^n,\widehat{p}\right]       & = i \hbar n x^{n-1}                                        \\
    \left[\widehat{x},\widehat{p}^2\right]       & = 2i\hbar\widehat{p}                                       \\
    \left[\widehat{V},\widehat{x}\right]         & =0                            &  & \text{Potential energy} \\
    \left[\widehat{H},\widehat{x}\right]         & =-\frac{i\hbar}{m}\widehat{p} &  & \text{Total energy}     \\
    \left[\widehat{H}, \widehat{V}\right]        & \neq 0                                                     \\
    \left[f(\widehat{x}),\widehat{p}\right]      & = i \hbar\frac{df}{dx}                                     \\
    \left[\widehat{L}_x,\widehat{L}_y\right]     & =i\hbar\widehat{L}_z          &  & \text{Angular momentum}
\end{align*}


\ptitle{Evaluate Commutators}

For some operators it can be useful to evaluate the commutator on a test function first, and then remove the test function in the end, e.g.:
\noindent\begin{align*}
    \left[\frac{d^2}{dx^2}, \widehat{x}\right] f(x) & = \frac{d^2}{dx^2} xf(x) - x\frac{d^2}{dx^2}f(x) \\
                                                    & = 2\frac{d}{dx} f(x)                             \\
    \left[\frac{d^2}{dx^2}, \widehat{x}\right]      & = 2\frac{d}{dx}
\end{align*}

\textbf{Remark}:

It is important to calculate $\widehat{A}\widehat{B}f - \widehat{B}\widehat{A}f$ and not $(\widehat{A}\widehat{B} - \widehat{B}\widehat{A})f$ directly.

\subsection{Inner Product}\label{ssec:InnerProd}
\ptitle{Properties}
\noindent\begin{align*}
    \langle f|g \rangle         & := \int_{-\infty}^{\infty} f^* g\; d \mathbf{r} \in L_2 &  &                                   \\
    \langle \lambda f|g \rangle & =\lambda^* \langle f|g \rangle                          &  & \text{semilinear for scalar mul.} \\
    \langle f|\lambda g \rangle & =\lambda \langle f|g \rangle                                                                   \\
    \langle f+g|h \rangle       & =\langle f|h \rangle + \langle g|h \rangle              &  & \text{linear for addition}        \\
    \langle f|g+h \rangle       & =\langle f|g \rangle + \langle f|h \rangle                                                     \\
    \langle g|f \rangle         & = {\langle f|g \rangle}^*                               &  & \text{``hermitian''}              \\
    \langle f|f \rangle         & \ge 0                                                   &  & \text{positive definite}          \\
    \langle f|f \rangle         & = 0 \leftrightarrow f=0
\end{align*}

\ptitle{Schwarz-Inequality}
\begin{align*}
    |\langle f|g \rangle |^2                  & \leq \langle f|f \rangle \langle g|g \rangle                 \\
    \left|\int_{a}^{b}{f(x)}^{*}g(x)dx\right| & \leq \sqrt{\int_{a}^{b}|f(x)|^{2}dx\int_{a}^{b}|g(x)|^{2}dx}
\end{align*}

\subsection{Constants}
\noindent\begin{align*}
    h = 6.63 \cdot 10^{-34} \;\mathrm{J\cdot s}      &  &  & \text{Planck constant}    \\
    k_b =8.617 \cdot 10^{-5} \;\mathrm{\frac{eV}{K}} &  &  & \text{Boltzmann constant}
\end{align*}

\subsection{Important Laws from Classical Physics}
\textbf{Ideal Gas}\\
Ideal gas law:
\noindent\begin{align*}
    PV & =nRT=n k_B T                                             \\
    \text{with}                                                   \\
    n  & =\frac{N_{particles}}{N_A} \text{ ($n$: number of Mols)} \\
    R  & =N_A k_B \text{ universal gas constant}
\end{align*}
Distance between atoms:
\noindent\begin{align*}
    d & ={\left(\frac{V_{gas}}{N}\right)}^{1/3}
\end{align*}

\subsection{Useful Integrals}
\subsubsection{Exponentials}
\begin{footnotesize}
    \noindent\begin{align*}
         & \int e^{\lambda x}dx           &  & =\frac{1}{\lambda }e^{\lambda x}+C                                                 \\
         & \int a^{\lambda x}dx           &  & =\frac{1}{\lambda \cdot \ln(a)}a^{\lambda x}+C                                     \\
         & \int e^{\lambda x}\sin(ax+b)dx &  & =\frac{e^{\lambda x}}{a^2+\lambda ^2}\left(\lambda \sin(ax+b)-a\cos(ax+b)\right)+C \\
         & \int e^{\lambda x}\cos(ax+b)dx &  & =\frac{e^{\lambda x}}{a^2+\lambda ^2}\left(\lambda \cos(ax+b)+a\sin(ax+b)\right)+C \\
         & \int x \cdot e^{\lambda x}dx   &  & =(\frac{\lambda x-1}{\lambda ^2})\cdot e^{\lambda x}+C                             \\
         & \int x^2 \cdot e^{\lambda x}dx &  & =(\frac{\lambda ^2x^2-2\lambda x+2}{\lambda ^3})\cdot e^{\lambda x}                \\
         & \int x\cdot e^{x^2}dx          &  & =\frac{1}{2}\cdot e^{x^2}+C
    \end{align*}
\end{footnotesize}

\paragraph{Specific Intervals}
\begin{footnotesize}
    \noindent\begin{align*}
         & \int_0^{\infty} e^{-\lambda x}x^n dx                                               &  & =\frac{n!}{\lambda ^{n+1}},\quad \lambda >0                                       \\
         & \int_0^{\infty} e^{-\lambda x^2} dx                                                &  & =\frac{1}{2}\sqrt{\frac{\pi}{\lambda }},\quad \lambda >0                          \\
         & \int_{-\infty}^{\infty} e^{-\lambda x^2} dx                                        &  & =\sqrt{\frac{\pi}{\lambda }},\quad \lambda >0                                     \\
         & \int_{-\infty}^{\infty}e^{-ax^2}e^{-iwx} dx                                        &  & = \sqrt{\frac{\pi}{a}}e^{\frac{w^2}{4a}}                                          \\
         & \int_{-\infty}^{\infty}e^{-(ax^2+bx+c)}dx                                          &  & = \sqrt{\frac{\pi}{a}}e^{\frac{b^2}{4a}-c}                                        \\
         & \int_{-\infty}^{\infty}x^2e^{-\lambda x^2}dx                                       &  & = \frac{1}{2\lambda }\sqrt{\frac{\pi}{\lambda }},\quad \mathrm{Re}\{\lambda \} >0 \\
         & \int_0^\infty x^n e^{-ax}dx                                                        &  & =\frac{n!}{a^{n+1}},\quad n\in\mathbb{N}^+                                        \\
         & \int_{0}^{a} x\sin^2\left(\frac{n\pi}{x}x\right)\; dx                              &  & =\frac{a^2}{4}                                                                    \\
         & \int_0^a x^2\cdot\sin^2\left(\frac{n\pi}{a}x\right)dx                              &  & ={\left(\frac{a}{2\pi n}\right)}^3\left(\frac{4\pi^3n^3}{3}-2n\pi\right)          \\
         & \int_{0}^{a} x\sin\left(\frac{\pi}{x}x\right)\sin\left(\frac{2\pi}{x}x\right)\; dx &  & = -\frac{8a^2}{9\pi^2}
    \end{align*}
\end{footnotesize}
