\section{Appendix}
\subsection{Notation}
The following notation is based on the book \textit{Introduction to Quantum Mechanics} by David J. Darrel
\begin{align*}
     & \lvert \alpha \rangle                                             &  & \text{Vector}                             \\
     & \alpha_i^{b_j}                                                    &  & \text{Vector element $i$ wrt.\ base } b_j \\
     & \hat{T}                                                           &  & \text{Linear transformation/ Operator}    \\
     & \mathbf{T}                                                        &  & \text{Matrix}                             \\
     & \widetilde{\mathbf{T}}                                            &  & \text{Transpose}                          \\
     & \mathbf{T}^*                                                      &  & \text{Complex conjugate}                  \\
     & \mathbf{T}^\dagger = {\widetilde{\mathbf{T}}}^*                   &  & \text{Hermitian}                          \\
     & \mathbf{T}^{-1}                                                   &  & \text{Inverse}                            \\
     & \lbrack \mathbf{S},\mathbf{T} \rbrack = \mathbf{ST} - \mathbf{TS} &  & \text{Commutator}                         \\
\end{align*}

\subsection{Constants}
\noindent\begin{align*}
    h = 6.63 \cdot 10^{-34} \;\mathrm{J\cdot s}      &  &  & \text{Planck constant}    \\
    k_b =8.617 \cdot 10^{-5} \;\mathrm{\frac{eV}{K}} &  &  & \text{Boltzmann constant}
\end{align*}

\subsection{Important Laws from Classical Physics}
\textbf{Ideal Gas}\\
Ideal gas law:
\noindent\begin{align*}
    PV & =nRT=n k_B T                                             \\
    \text{with}                                                   \\
    n  & =\frac{N_{particles}}{N_A} \text{ ($n$: number of Mols)} \\
    R  & =N_A k_B \text{ universal gas constant}
\end{align*}
Distance between atoms:
\noindent\begin{align*}
    d & ={\left(\frac{V_{gas}}{N}\right)}^{1/3}
\end{align*}

\subsection{Useful Integrals}
\subsubsection{Exponentials}
\begin{footnotesize}
    \noindent\begin{align*}
         & \int e^{\lambda x}dx           &  & =\frac{1}{\lambda }e^{\lambda x}+C                                                 \\
         & \int a^{\lambda x}dx           &  & =\frac{1}{\lambda \cdot \ln(a)}a^{\lambda x}+C                                     \\
         & \int e^{\lambda x}\sin(ax+b)dx &  & =\frac{e^{\lambda x}}{a^2+\lambda ^2}\left(\lambda \sin(ax+b)-a\cos(ax+b)\right)+C \\
         & \int e^{\lambda x}\cos(ax+b)dx &  & =\frac{e^{\lambda x}}{a^2+\lambda ^2}\left(\lambda \cos(ax+b)+a\sin(ax+b)\right)+C \\
         & \int x \cdot e^{\lambda x}dx   &  & =(\frac{\lambda x-1}{\lambda ^2})\cdot e^{\lambda x}+C                             \\
         & \int x^2 \cdot e^{\lambda x}dx &  & =(\frac{\lambda ^2x^2-2\lambda x+2}{\lambda ^3})\cdot e^{\lambda x}                \\
         & \int x\cdot e^{x^2}dx          &  & =\frac{1}{2}\cdot e^{x^2}+C
    \end{align*}
\end{footnotesize}

\paragraph{Specific Intervals}
\begin{footnotesize}
    \noindent\begin{align*}
         & \int_0^{\infty} e^{-\lambda x}x^n dx                                               &  & =\frac{n!}{\lambda ^{n+1}},\quad \lambda >0                                       \\
         & \int_0^{\infty} e^{-\lambda x^2} dx                                                &  & =\frac{1}{2}\sqrt{\frac{\pi}{\lambda }},\quad \lambda >0                          \\
         & \int_{-\infty}^{\infty} e^{-\lambda x^2} dx                                        &  & =\sqrt{\frac{\pi}{\lambda }},\quad \lambda >0                                     \\
         & \int_{-\infty}^{\infty}e^{-ax^2}e^{-iwx} dx                                        &  & = \sqrt{\frac{\pi}{a}}e^{\frac{w^2}{4a}}                                          \\
         & \int_{-\infty}^{\infty}e^{-(ax^2+bx+c)}dx                                          &  & = \sqrt{\frac{\pi}{a}}e^{\frac{b^2}{4a}-c}                                        \\
         & \int_{-\infty}^{\infty}x^2e^{-\lambda x^2}dx                                       &  & = \frac{1}{2\lambda }\sqrt{\frac{\pi}{\lambda }},\quad \mathrm{Re}\{\lambda \} >0 \\
         & \int_0^\infty x^n e^{-ax}dx                                                        &  & =\frac{n!}{a^{n+1}},\quad n\in\mathbb{N}^+                                        \\
         & \int_{0}^{a} x\sin^2\left(\frac{n\pi}{x}x\right)\; dx                              &  & =\frac{a^2}{4}                                                                    \\
         & \int_0^a x^2\cdot\sin^2\left(\frac{n\pi}{a}x\right)dx                              &  & ={\left(\frac{a}{2\pi n}\right)}^3\left(\frac{4\pi^3n^3}{3}-2n\pi\right)          \\
         & \int_{0}^{a} x\sin\left(\frac{\pi}{x}x\right)\sin\left(\frac{2\pi}{x}x\right)\; dx &  & = -\frac{8a^2}{9\pi^2}
    \end{align*}
\end{footnotesize}
