\section{Multiple Particle Systems}
\subsection{Z-particle Systems}
\subsubsection{Hydrogenic States}
A hydrogenic atom with $Z$ protons and one electron has
\begin{equation*}
    \widetilde{E}_n = -\left[\frac{m}{2\hbar^2}{\left(\frac{Z e^2}{4\pi\epsilon_0}\right)}^2\right]\frac{1}{n^2} = \frac{Z^2 E_1}{n^2}, \quad n=1,2,\dots
\end{equation*}
\subsubsection{Multi-Electron Systems}
In a general multi-particle system (e.g.\ atom with $Z$ protons) the Hamiltonian is given by
    {\small
        \noindent\begin{equation*}
            \widehat{H} = \sum_{j=1}^{Z} \Biggl[ \underbrace{ -\frac{\hbar^{2}}{2m}\nabla_{j}^{2}-\left(\frac{1}{4\pi \epsilon_{0}}\right)\frac{Ze^{2}}{r_{j}}}_{T\text{ and interaction with nucleus}}  +              \underbrace{\frac{1}{2}\left(\frac{1}{4\pi \epsilon_{0}}\right)\sum_{j\neq k}^{Z}\frac{e^{2}}{\left|\mathbf{r}_{j}-\mathbf{r}_{k}\right|}}_{\text{repulsive int.\ of electrons}}\Biggr]
        \end{equation*}
    }

The ``repulsion'' term makes exact solving of the TISE, in general, impossible. An approximation and an exception are presented hereafter.

\subsection{Two-Particle Systems}
$\widehat{H}$ is given by
\begin{equation*}
    \widehat{H}=-\frac{\hbar^{2}}{2m_{1}}\nabla_{1}^{2}-\frac{\hbar^{2}}{2m_{2}}\nabla_{2}^{2}+V(\mathbf{r}_{1},\mathbf{r}_{2},t)
\end{equation*}
where $V$ includes any interactions between particles i.e.:
\begin{enumerate}
    \item interactions between  electrons
    \item interactions of electrons with the nucleus
\end{enumerate}

\subsection{Reducable Two-Particle Problems}
\subsubsection{Central Potential Systems}

Given a system where the particles interact only with each other via a potential that depends on their separation
\begin{equation*}
    V(\mathbf{r}_1,\mathbf{r}_2)\to V(|\mathbf{r}_1-\mathbf{r}_2|)
\end{equation*}
the two-particle TISE can be solved analytically by splitting the problem into
\noindent\begin{align*}
    \textbf{c.o.m. motion:}   &                                                                              \\
    \mathbf{R}\equiv\         & \frac{m_1\mathbf{r_1}+m_2\mathbf{r_2}}{m_1+m_2} &   & \text{ center of mass} \\
    M\equiv\                  & m_1 + m_2                                       &   & \text{ total mass}     \\
    \textbf{relative motion:} &                                                                              \\
    \mathbf{r}\equiv\         & \mathbf{r_1}-\mathbf{r_2}                       &                            \\
    m_r\equiv\                & \frac{m_1 m_2}{m_1+m_2}                         &   & \text{ reduced mass}
\end{align*}
In this case, the TISE separates to
\begin{equation*}
    -\frac{\hbar^{2}}{2M}\nabla_{\mathbf{R}}^{2}\psi-\frac{\hbar^{2}}{2m_r}\nabla_{\mathbf{r}}^{2}\psi+V(\mathbf{r})\psi=E\psi
\end{equation*}
and we can find a solution
\begin{equation*}
    \psi = \psi_\mathbf{R}(\mathbf{R})\cdot\psi_\mathbf{r}(\mathbf{r})
\end{equation*}

\subsubsection{Noninteracting Particles}
Ignore the interactions ($V$).
\newpar{}
\ptitle{Distinguishable Particles} (no Pauli)

The overall solutions are products
\begin{equation*}
    \psi(\mathbf{r}_{1},\mathbf{r}_{2} )=\psi_{a}(\mathbf{r}_{1})\psi_{b}(\mathbf{r}_{2}) \quad\text{or}\quad
    \psi(\mathbf{r}_{1},\mathbf{r}_{2}) =\psi_{b}(\mathbf{r}_{1})\psi_{a}(\mathbf{r}_{2})
\end{equation*}

\newpar{}
\ptitle{Indistinguishable Particles} (bosons, fermions)

The overall solutions are linear combinations (sym.\ or asym.):
\begin{align*}
    \psi_{+}(\mathbf{r}_{1},\mathbf{r}_{2}) & =C\left[\psi_{a}(\mathbf{r}_{1})\psi_{b}(\mathbf{r}_{2})+\psi_{b}(\mathbf{r}_{1})\psi_{a}(\mathbf{r}_{2})\right]\quad \mathrm{symmetric} \\
    \psi_{-}(\mathbf{r}_{1},\mathbf{r}_{2}) & =C\left[\psi_{a}(\mathbf{r}_{1})\psi_{b}(\mathbf{r}_{2})-\psi_{b}(\mathbf{r}_{1})\psi_{a}(\mathbf{r}_{2})\right]\quad \mathrm{antisymm.}
\end{align*}

\paragraph[Exchange Symmetry]{Exchange Symmetry of $\psi_\pm$}
\ptitle{Exchange Operator $\widehat{P}$}

\noindent\begin{equation*}
    \widehat{P}\psi(\mathbf{r_1},\mathbf{r_2})=\psi(\mathbf{r_2},\mathbf{r_1})
\end{equation*}
$\widehat{P}$, $\widehat{H}$ share the eigenfunctions $\psi_{\pm}$ (eigvals.\ = $\pm 1$):
\begin{equation*}
    \left[\widehat{P},\widehat{H}\right]=0 \quad \overset{\text{gen. Ehrenfest}}{\Rightarrow} \quad
    \frac{d}{dt}\langle\widehat{P}\rangle=0
\end{equation*}

For two \textbf{distinguishable} particles one finds that
\begin{equation*}
    \left\langle{\left(x_{1}-x_{2}\right)}^{2}\right\rangle_{d}=\left\langle x^{2}\right\rangle_{a}+\left\langle x^{2}\right\rangle_{b}-2\left\langle x\right\rangle_{a}\left\langle x\right\rangle_{b}
\end{equation*}
For two \textbf{undistinguishable} particles one has
\begin{align*}
    \left\langle{\left(x_{1}-x_{2}\right)}^{2}\right\rangle_{\pm}= & \left\langle x^{2}\right\rangle_{a}+\left\langle x^{2}\right\rangle_{b}-2\langle x\rangle_{a}\langle x\rangle_{b}\mp2|\langle x\rangle_{ab}|^{2} \\
                                                                   & =\left\langle{\left(\Delta x\right)}^{2}\right\rangle_{d}\mp2\left|\left\langle x\right\rangle_{ab}\right|^{2}
\end{align*}
meaning that particles in $\psi_{+}$ (symmetric w.r.t.\ exchange) tend to be closer together.

\textbf{Remarks}
\begin{itemize}
    \item Exchange symmetry or the overall 2-particle wave function influences location of particles w.r.t.\ each other
\end{itemize}

\paragraph{Bosons and Fermions}
\begin{itemize}
    \item \textbf{Bosons} are particles with \textbf{integer} spin (e.g.\ photons (s=1), gravitons (s=2))
    \item \textbf{Fermions} are particles with \textbf{half-integer} spin (e.g.\ electrons, protons, neutrons (s=$\frac{1}{2}$))
\end{itemize}

% \newpar{}
\ptitle{Exchange Symmetry Axiom}

The exchange symmetry of the overall wave function of
\begin{itemize}
    \item identical \textbf{bosons is symmetric} w.r.t.\ exchange
    \item identical \textbf{fermions is antisymmetric} w.r.t.\ exchange
\end{itemize}

\newpar{}
\ptitle{Exchange Symmetry of Hydrogen}

One has for $\chi(s)$ and two identical spins $s=\frac{1}{2}$ the spin states
\begin{align*}
     & \begin{rcases}
           \uparrow \uparrow                                                        \\
           \downarrow \downarrow                                                    \\
           \frac{1}{\sqrt{2}}\left(\uparrow \downarrow + \downarrow \uparrow\right) \\
       \end{rcases} \chi_{\mathrm{triplet}}(s)\text{: symmetric w.r.t.\ ex.\ } (S=1) \\
     & \begin{rcases}
           \frac{1}{\sqrt{2}}\left(\uparrow \downarrow - \downarrow \uparrow\right)
       \end{rcases} \chi_{\mathrm{singlet}}(s) \text{: antisymmetric w.r.t.\ ex.\ } (S=0)
\end{align*}
Because $\psi$ of fermions is antisymmetric:
\noindent\begin{align*}
    \psi & = \psi_{+}\cdot \chi_{\mathrm{singlet}}(s) &  & \text{stable chemical bond}        \\
    \psi & = \psi_{-}\cdot \chi_{\mathrm{triplet}}(s) &  & \text{higher energy (antibonding)}
\end{align*}

\newpar{}
\ptitle{Pauli Exclusion Principle}

One cannot have two fermions in exactly the same state (same QN) and still have antisymmetry w.r.t.\ exchange.
(Does not apply to bosons)
\subsection{Wavefunction for Fermions}
\noindent\begin{equation*}
    \psi_{n,\ell,m_\ell,m_s} = A\cdot R_{n\ell}(r)\cdot Y_\ell^{m_\ell}(\theta, \varphi) \cdot \left|s,m_s \right\rangle
\end{equation*}