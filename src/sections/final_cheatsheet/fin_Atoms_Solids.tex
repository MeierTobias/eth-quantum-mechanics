\section{Atoms}
\subsection{Hydrogen Orbitals}
\renewcommand{\arraystretch}{1.1}
\setlength{\oldtabcolsep}{\tabcolsep}\setlength\tabcolsep{2pt}
{\small     % Smaller font
    \begin{tabularx}{\linewidth}{@{}ll>{\footnotesize}lX@{}}
                 & \textbf{Name} & \textbf{(Range,Letter)}                             & \textbf{Meaning}           \\
        $n$      & princ. QN     & $(0,K),(1,L),(2,M)\dots$                            & shell (radius)             \\
        $\ell$   & azim. QN      & $(0,s),(1,p),(2,d)$                                 & subshell (orbital shape)   \\
                 &               & $(3,f),\ldots,(n-1,\cdot)$                          &                            \\
        $m_\ell$ & magn. QN      & $-\ell,-\ell+1,\dots,\ell-1,\ell$                   & \# of orbital orientations \\
        $m_s$    & spin QN       & $(+\frac{1}{2},\uparrow),(-\frac{1}{2},\downarrow)$ & spin orientation           \\
                 &               &                                                     & (w.r.t.\ z-axis)           \\
        $s$      & spin          & fermions: half-int                                  &                            \\
                 &               & bosons: int                                         &
    \end{tabularx}
}           % End smaller font
\renewcommand{\arraystretch}{1}
\setlength{\tabcolsep}{\oldtabcolsep}
\subsubsection{Orbital Table}
\renewcommand{\arraystretch}{0.8}
\setlength{\oldtabcolsep}{\tabcolsep}\setlength\tabcolsep{3pt}
{\small     % Smaller font
    \begin{tabularx}{\linewidth}{@{}llllll@{}}
            &                & \textbf{Subshell(s)}          & \# \textbf{Orbitals}              & \multicolumn{2}{c}{\# \textbf{of electrons}\dots}                      \\
        $n$ & \textbf{Shell} & $\ell=0\dots n-1$             & $m_{\ell}=-\ell,\dots \ell$       & \textbf{per Subshell}                             & \textbf{per Shell} \\
        1   & K              & s \quad{\color{teal}$\ell=0$} & 1                                 & 2                                                 & 2                  \\
        \cmidrule{1-6}
        2   & L              & s \quad{\color{teal}$\ell=0$} & 1 \qquad{\color{teal} $\cdot 2=$} & 2  \qquad{\color{teal} $\sum=$}                   & 8                  \\
            &                & p \quad{\color{teal}$\ell=1$} & 3 \qquad{\color{teal} $\cdot 2=$} & 6                                                 &                    \\
        \cmidrule{1-6}
        3   & M              & s \quad{\color{teal}$\ell=0$} & 1                                 & 2                                                 & 18                 \\
            &                & p \quad{\color{teal}$\ell=1$} & 3                                 & 6                                                 &                    \\
            &                & d \quad{\color{teal}$\ell=2$} & 5                                 & 10                                                &                    \\
        % \cmidrule{1-6}
        % 4   & N              & s                    & 1                                  & 2                                                 & 32                 \\
        %     &                & p                    & 3                                  & 6                                                 &                    \\
        %     &                & d                    & 5                                  & 10                                                &                    \\
        %     &                & f                    & 7                                  & 14                                                &
    \end{tabularx}
}           % End smaller font
\renewcommand{\arraystretch}{1}
\setlength{\tabcolsep}{\oldtabcolsep}

% \ptitle{Remarks}

% \begin{itemize}
%     \item The total number of orbitals is the number of possible states ($n,l,m_l$) that one electron can occupy.
% \end{itemize}
\subsection{Filling Orbitals with Electrons}
Pauli/spin: 2 electron s per orbital.

\ptitle{Filling Order}

\includegraphics[width=\linewidth]{periodic_table_el_configurations.png}

\noindent\begin{equation*}
    1s^2 2s^2 2p^6 3s^2 3p^6 4s^2 3d^{10}\dots^{\textsf{total nr.\ of electrons}}
\end{equation*}

\ptitle{Noble Gas Notation}
\noindent\begin{equation*}
    Fe(Z=26): 1s^2 2s^2 2p^6 3s^2 3p^6 4s^2 3d^6 = \left[Ar\right]4s^2 3d^6
\end{equation*}

\ptitle{Exceptions}: $Cr: \left[Ar\right]4s^1 3d^5 \quad Cu: \left[Ar\right]4s^1 3d^{10}$

\subsection{Full-State Description of Atoms}
\ptitle{Angular Momentum (AM)}

\renewcommand{\arraystretch}{1.3}
\setlength{\oldtabcolsep}{\tabcolsep}\setlength\tabcolsep{6pt}

\begin{tabularx}{\linewidth}{@{}llll@{}}
    *       & total * AM & projection on $z$                                  \\
    \cmidrule{1-3}
    orbital & $L$        & $M_L=\sum_{i}m_{\ell,i}$ & $M_L=0 \Rightarrow L=0$ \\
    spin    & $S$        & $M_S=\sum_{i}m_{s,i}$    & $M_S=0 \Rightarrow S=0$ \\
    total   & $J$        & $M_J=M_L+M_S$
\end{tabularx}

\renewcommand{\arraystretch}{1}
\setlength\tabcolsep{\oldtabcolsep}

\ptitle{Addition Rule}
\begin{equation*}
    j_{\mathrm{total}} = j_1 \oplus j_2 = (j_1+j_2),\dots,|j_1-j_2|
\end{equation*}

\ptitle{Term Symbol}
\noindent\begin{equation*}
    \prescript{\overbrace{2S+1}^{\textsf{\# of projections}}}{}{X_J}, \quad X=
        {\begin{cases}
                S,     & L=0 \\
                P,     & L=1 \\
                D,     & L=2 \\
                \cdots &
            \end{cases}}
\end{equation*}

\ptitle{Finding Ground State}
\begin{enumerate}
    \item Look only at partially filled subshells
    \item Use \textbf{addition rule} to find all possible combinations
          \noindent\begin{equation*}
              J = L\oplus S \quad\begin{cases}
                  L =\ell_1 \oplus\ell_2 \oplus \cdots \oplus\ell_n \\
                  S = s_1 \oplus s_2 \oplus \cdots \oplus s_n       \\
              \end{cases}
          \end{equation*}
    \item Consider exchange symmetry
          \begin{itemize}
              \item (fermions) keep only antisymmetric states
              \item (exclude states that violate Pauli excl.\ principle i.e. 2 electrons with exact same QN)
          \end{itemize}
    \item Find ground state with \textbf{Hund's rule}
          \begin{enumerate}
              \item State with largest $S$ is most stable
              \item For states with same $S$, largest $L$ is most stable
              \item For states with same $S$ and $L$ (but different $J$)
                    \begin{itemize}
                        \item Smallest $J$ most stable for subshells that are no more than half full
                        \item Larqest $J$ most stable for subshells that are more than half full
                    \end{itemize}
          \end{enumerate}
    \item (optional) Multiply quantum numbers by $\hbar$
\end{enumerate}


\section{Solids}
\subsection{Free-Electron Model}\label{ssec:FEM}
Neglecting spin and electron-electron interactions, an electrons wavefunction in a box with length $l_x,l_y,l_z$ is given by
\noindent\begin{align*}
    E_{n_x,n_y,n_z}    & = \frac{\hbar^2\pi^2}{2m}\left(\frac{n_x^2}{l_x^2}+\frac{n_y^2}{l_y^2}+\frac{n_z^2}{l_z^2}\right),\quad n_x,n_y,n_z \in \mathbb{N}\backslash 0 \\
    \psi_{n_x,n_y,n_z} & = \sqrt{\frac{8}{l_x l_y l_z}} \sin\left(\frac{n_x\pi}{l_x}x\right) \sin\left(\frac{n_y\pi}{l_y}y\right) \sin\left(\frac{n_z\pi}{l_z}z\right)
\end{align*}
with spacing ($\to$ bands). For $l_x=l_y=l_z=$ 1 cm one has:
\noindent\begin{equation*}
    \Delta E \leq 10^{-6} ~\mathrm{eV}
\end{equation*}

\subsubsection{k-Space}
\noindent\begin{equation*}
    E = \frac{\hbar^2k^2}{2m}, \quad \begin{cases}
        k^2 = k_x^2+k_y^2+k_z^2 \\
        k_x = \frac{n_x\pi}{l_x}, k_y = \frac{n_y\pi}{l_y}, k_z = \frac{n_z\pi}{l_z}
    \end{cases}
\end{equation*}
and each solution occupies $\frac{\pi^3}{l_x l_y l_z}=\frac{\pi^3}{V}$

\subsubsection{Fermi Energy}
For a solid with $N$ atoms with $q$ valence electrons each the total volume (3D)
\noindent\begin{equation*}
    \frac{1}{8}\cdot\underbrace{\frac{4}{3}\pi k_F^3}_{\textsf{vol. sphere}} = \underbrace{\frac{Nq}{2}}_{\frac{\textsf{\# electrons}}{\mathsf{spin}}}\cdot \underbrace{\frac{\pi^3}{V}}_{\textsf{vol. per state}}
\end{equation*}
is occupied in k-space until all electron states are occupied.
\noindent\begin{equation*}
    k_F  = \begin{cases}
        Nq\frac{\pi}{2L}                                  & \mathrm{1D} \\
        {\left(\frac{2Nq\pi}{A}\right)}^\frac{1}{2}       & \mathrm{2D} \\
        {\left(\frac{3\pi^2 N q}{V}\right)}^{\frac{1}{3}} & \mathrm{3D}
    \end{cases}, \quad
    E_F  = \begin{cases}
        \frac{1}{8}{\left(\hbar \pi \frac{Nq}{L}\right)}^2                    & \mathrm{1D} \\
        \frac{\hbar^2 \pi}{m} \frac{Nq}{A}                                  & \mathrm{2D} \\
        \frac{\hbar^2}{2m}{\left(\frac{3\pi^2 N q}{V}\right)}^{\frac{2}{3}} & \mathrm{3D}
    \end{cases}
\end{equation*}

\newpar{}
\ptitle{Density of States}

The \textit{density of states} - the number of one-electron levels per unit energy is given by
\noindent\begin{equation*}
    D(E)=\frac{\partial N(E)}{\partial E} \overset{\mathrm{3D}}{=} \frac{V}{2\pi^2}{\left(\frac{2m}{\hbar^2}\right)}^{\frac{3}{2}} E^{\frac{1}{2}} \quad \Rightarrow \Delta E = \frac{1}{D(E)}
\end{equation*}
and changes with the energy. It is denser for bottom states.

\subsection{Potential Models}
\subsubsection{Bloch's Theorem}
\noindent\begin{equation*}
    V(X) = V(x+a)
\end{equation*}
\noindent\begin{equation*}
    \psi(x+a) = e^{iKa} \psi(x), \quad K\in \mathbb{R}
\end{equation*}
In words: The wave function only changes in space by a phase factor.
\noindent\begin{equation*}
    {|\psi(x+a)|}^2={|\psi(x)|}^2
\end{equation*}

\subsubsection{Periodic Boundary Conditions}
\noindent\begin{align*}
    \psi(x) = \psi(x+Na) & = e^{iKNa}\psi(x)                                                       \\
    1                    & =e^{iKNa}         & \Rightarrow K=\frac{2\pi j}{Na}, \; j\in \mathbb{Z}
\end{align*}

\ptitle{Potential}

In its simples form, the potential can be modeled with a \textit{Dirac-comb}:
\noindent\begin{equation*}
    V(x)=\alpha\sum_{j=0}^{N-1}\delta(x-ja)
\end{equation*}
Solving the 1D TISE for this potential results in
\noindent\begin{equation*}
    \underbrace{\cos(Ka)}_{\in [-1,1]} = \underbrace{\cos(ka) + \frac{m\alpha}{\hbar^2 k}\sin(ka)}_{\textsf{exceeds }[-1,1]}, \quad k=\sqrt{\frac{2mE}{\hbar^2}}
\end{equation*}
and only certain bands of energy are allowed.

\subsection{Energy Bands and Band Gaps}
\begin{itemize}
    \item \textbf{Metals}: ``no'' band gap $E_g$
    \item \textbf{Semiconductor}: $E_g < 4eV$
    \item \textbf{Insulator}: $E_g > 4eV$
\end{itemize}

\subsection{Various Useful Stuff}
\noindent\begin{equation*}
    n_{\mathsf{electrons}} = N_i \exp\left(-\frac{E_g}{k_b T}\right)\quad \begin{cases}
        n:   & \mathrm{\frac{electrons}{cm^3}} \\
        N_i: & \mathrm{cm^{-3}}
    \end{cases}
\end{equation*}
\noindent\begin{equation*}
    E^{(\mathrm{eV})} = E^{(\mathrm{J})} \cdot 6.242 \cdot 10^{18}\, \mathrm{\frac{eV}{J}}
\end{equation*}