\section{Approximation Methods}

\ptitle{Perturbation theory}
\begin{itemize}
    \item Attempts to solve new problems given the solution of an existing, simpler one.
    \item Provides \textbf{corrections} to \textbf{all} energy levels of a certain \textbf{order}.
\end{itemize}
\ptitle{Variational principle}
\begin{itemize}
    \item Doesn't need previous knowledge of energies or states but, in turn, can also not use it to yield a broad range of results.
    \item Provides an \textbf{upper bound} for the ground state energy (any some information on the 1st excited state) using trial functions.
\end{itemize}

\subsection{Time-Independent Perturbation Theory}
\noindent\begin{equation*}
    H=H^{(0)}+\lambda H^{\prime}, \quad \to H\psi_n = E_n \psi_n
\end{equation*}

\subsubsection{Non-Degenerate Perturbation Theory}
\noindent\begin{align*}
    E_{n}    & =E_{n}^{(0)}+\lambda E_{n}^{(1)}+\lambda^{2}E_{n}^{(2)}+\cdots         \\
    \psi_{n} & =\psi_{n}^{(0)}+\lambda\psi_{n}^{(1)}+\lambda^{2}\psi_{n}^{(2)}+\cdots
\end{align*}

\ptitle{1st and 2nd Order Approximations}
\begin{equation*}
    E_{n}^{(1)}=\left\langle\psi_{n}^{(0)}\right| H^{\prime} \left.\psi_{n}^{(0)}\right\rangle
\end{equation*}

\begin{equation*}
    \psi_{n}^{(1)}=\sum_{m\neq n}\frac{\left\langle\psi_{m}^{(0)}\right|H^{\prime}\left.\psi_{n}^{(0)}\right\rangle}{\left(E_{n}^{(0)}-E_{m}^{(0)}\right)}\psi_{m}^{(0)}
\end{equation*}

\begin{equation*}
    E_{n}^{(2)}=\sum_{m\neq n}\frac{\left|\left\langle\psi_{m}^{(0)}\right|H^{\prime}\left.\psi_{n}^{(0)}\right\rangle\right|^{2}}{E_{n}^{(0)}-E_{m}^{(0)}}
\end{equation*}

\subsubsection{Degenerate Perturbation Theory}
\textbf{Two} different unpertubed states are degenerate if they share their eigenvalues ($E_1^{(0)} = E_2^{(0)}$).

\newpar{}
\ptitle{1st Order Approximation}
Perturbation ``splits degeneracy up'' into 2 non-degenerate states:
\noindent\begin{gather*}
    E_1 = E_1^{(0)} + E_{-}^{(1)} \quad < \quad E_2 = E_2^{(0)} + E_{+}^{(1)}\\
    E_{\pm}^{(1)}=\frac{1}{2}\left[W_{aa}+W_{bb}\pm\sqrt{{(W_{aa}-W_{bb})}^{2}+4 |W_{ab}|^{2}}\right]\\
    W_{ij}=\left\langle\psi_{i}^{(0)} \right| H^{\prime} \left.\psi_{j}^{(0)}\right\rangle
\end{gather*}

\paragraph{n-Fold Degeneracies}
% TODO: replace this with example??
\noindent\begin{equation*}
    \underbrace{
        \left(\begin{array}{cc}
            W_{aa} & W_{ab} \\
            W_{ba} & W_{bb}
        \end{array}\right)}_{\mathbf{W}}
    \left(\begin{array}{c}
            \alpha \\
            \beta
        \end{array}\right)
    =E^{(1)}\left(\begin{array}{c}
            \alpha \\
            \beta
        \end{array}\right)
\end{equation*}

If $\mathbf{W}^{n\times n}$ is diagonal
\begin{itemize}
    \item The $\psi_a^{(0)},\psi_b^{(0)},\psi_c^{(0)} \dots $ are already proper eigenstates
    \item One has for the 2-fold degenerate case
          \begin{align*}
              E_{+}^{(1)} & =W_{aa}=\left\langle\psi_{a}^{(0)}\right|\left.\widehat{H}^{\prime}\psi_{a}^{(0)}\right\rangle \\
              E_{-}^{(1)} & =W_{bb}=\left\langle\psi_{b}^{(0)}\right|\left.\widehat{H}^{\prime}\psi_{b}^{(0)}\right\rangle
          \end{align*}
\end{itemize}

% \begin{examplesection}[Matrix example]
%     \noindent\begin{equation*}
%         \widehat{\mathbf{H}} = \underbrace{V_0 
%         \begin{bmatrix}
%             1&0&0\\
%             0&1&0\\
%             0&0&2
%         \end{bmatrix}}_{\mathbf{H}^(0)}
%         + \underbrace{V_0 \varepsilon
%         \begin{bmatrix}
%             -1&0&0\\
%             0&0&1\\
%             0&1&0
%         \end{bmatrix}}_{\lambda\mathbf{H}^{\prime}}
%     \end{equation*}

%     since $\mathbf{H}^{(0)}$ is already diagonalized
%     \noindent\begin{gather*}
%         \underbrace{\lambda_1 = \lambda_2 = V_0}_{\textsf{degenerate}},\quad \underbrace{\lambda_3 = 2V_0}_{\textsf{non-degenerate}}\\
%         \psi_1^{(0)} = \begin{bmatrix}
%             1&0&0
%         \end{bmatrix}^{\mathsf{T}},
%         \psi_2^{(0)} = \begin{bmatrix}
%             0&1&0
%         \end{bmatrix}^{\mathsf{T}},
%         \psi_3^{(0)} = \begin{bmatrix}
%             0&0&1
%         \end{bmatrix}^{\mathsf{T}}
%     \end{gather*}

%     \newpar{}
%     \ptitle{Non-Degenerate (3)}
%     \noindent\begin{equation*}
%         E_3 = E_3^{(0)}+E_3^{(1)}+E_3^{(2)} = V_0(2+\varepsilon^2)
%     \end{equation*}

%     \ptitle{Degenerate (1\&2)}
%     \noindent\begin{align*}
%             E_1 &= E_1^{(0)} + E_{-}^{(1)} = \lambda_1+ E_{-}^{(1)} \\
%             E_2 &= E_2^{(0)} + E_{+}^{(1)} = \lambda_1+ E_{+}^{(1)}
%     \end{align*}
% \end{examplesection}

\subsection{Variational Principle}
The variational principle theorem then states:
\begin{equation*}
    E_{gs}\leq\left\langle\psi_{\mathrm{trial}}\right|\widehat{H} \psi_{\mathrm{trial}}\left.\right\rangle \equiv \left<H\right>
\end{equation*}

\ptitle{Steps}
\begin{enumerate}
    \item Set trial function e.g.: $\Psi_{\mathrm{trial}}={\left(\frac{\alpha}{\pi}\right)}^{1/4}\exp\left(\frac{-\alpha x^{2}}{2}\right)$\newline (1s orbital of HO)
    \item Calculate expectation: $\left\langle H\right\rangle= \left\langle\psi_{\mathrm{trial}}\right|\widehat{H} \psi_{\mathrm{trial}}\left.\right\rangle$
    \item Minimize: $\frac{d}{d \alpha} \left\langle H\right\rangle=0\;\to\; \alpha_{\min}$ 
    \item $E_{gs}\leq {\left\langle H \right\rangle}_{\alpha_{\min}}$
\end{enumerate}