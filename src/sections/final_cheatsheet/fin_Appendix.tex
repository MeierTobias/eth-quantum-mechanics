\section{Tables}
\subsection{Useful Integrals}
\subsubsection{Exponentials}
\begin{footnotesize}
    \noindent\begin{align*}
        %  & \int e^{\lambda x}dx           &  & =\frac{1}{\lambda }e^{\lambda x}+C                                                 \\
         & \int a^{\lambda x}dx           &  & =\frac{1}{\lambda \cdot \ln(a)}a^{\lambda x}+C                                     \\
         & \int e^{\lambda x}\sin(ax+b)dx &  & =\frac{e^{\lambda x}}{a^2+\lambda ^2}\left(\lambda \sin(ax+b)-a\cos(ax+b)\right)+C \\
         & \int e^{\lambda x}\cos(ax+b)dx &  & =\frac{e^{\lambda x}}{a^2+\lambda ^2}\left(\lambda \cos(ax+b)+a\sin(ax+b)\right)+C \\
         & \int x \cdot e^{\lambda x}dx   &  & =(\frac{\lambda x-1}{\lambda ^2})\cdot e^{\lambda x}+C                             \\
         & \int x^2 \cdot e^{\lambda x}dx &  & =(\frac{\lambda ^2x^2-2\lambda x+2}{\lambda ^3})\cdot e^{\lambda x}                \\
         & \int x\cdot e^{\lambda x^2}dx  &  & =\frac{1}{2\lambda}\cdot e^{\lambda x^2}+C
    \end{align*}
\end{footnotesize}

\paragraph{Specific Intervals}
more on \textit{Useful Info Sheet}
\begin{footnotesize}
    \noindent\begin{align*}
        %  & \int_0^{\infty} e^{-\lambda x}x^n dx                                               &  & =\frac{n!}{\lambda ^{n+1}},\quad \lambda >0                                       \\
        %  & \int_0^{\infty} e^{-\lambda x^2} dx                                                &  & =\frac{1}{2}\sqrt{\frac{\pi}{\lambda }},\quad \lambda >0                          \\
        %  & \int_{-\infty}^{\infty} e^{-\lambda x^2} dx                                        &  & =\sqrt{\frac{\pi}{\lambda }},\quad \lambda >0                                     \\
         & \int_{-\infty}^{\infty}e^{-ax^2}e^{-iwx} dx                                        &  & = \sqrt{\frac{\pi}{a}}e^{\frac{w^2}{4a}}                                          \\
        %  & \int_{-\infty}^{\infty}e^{-(ax^2+bx+c)}dx                                          &  & = \sqrt{\frac{\pi}{a}}e^{\frac{b^2}{4a}-c}                                        \\
         & \int_{-\infty}^{\infty}x^2e^{-\lambda x^2}dx                                       &  & = \frac{1}{2\lambda }\sqrt{\frac{\pi}{\lambda }},\quad \mathrm{Re}\{\lambda \} >0 \\
        %  & \int_0^\infty x^n e^{-ax}dx                                                        &  & =\frac{n!}{a^{n+1}},\quad n\in\mathbb{N}^+                                        \\
        %  & \int_{0}^{a} x\sin^2\left(\frac{n\pi}{x}x\right)\; dx                              &  & =\frac{a^2}{4}                                                                    \\
        %  & \int_0^a x^2\cdot\sin^2\left(\frac{n\pi}{a}x\right)dx                              &  & ={\left(\frac{a}{2\pi n}\right)}^3\left(\frac{4\pi^3n^3}{3}-2n\pi\right)          \\
         & \int_{0}^{a} x\sin\left(\frac{\pi}{x}x\right)\sin\left(\frac{2\pi}{x}x\right)\; dx &  & = -\frac{8a^2}{9\pi^2}
    \end{align*}
\end{footnotesize}


\subsection{Legendre Polynomials}
% \begin{footnotesize}
%     \noindent\begin{align*}
%         P_{0} & =1                     & P_{3} & =\frac{1}{2}(5x^{3}-3x)           \\
%         P_{1} & =x                     & P_{4} & =\frac{1}{8}(35x^{4}-30x^{2}+3)   \\
%         P_{2} & =\frac{1}{2}(3x^{2}-1) & P_{5} & =\frac{1}{8}(63x^{5}-70x^{3}+15x)
%     \end{align*}
% \end{footnotesize}
\begin{footnotesize}
    \noindent\begin{gather*}
        P_{0} =1, \quad P_{1}=x,\quad P_{2}=\frac{1}{2}(3x^{2}-1), \quad P_{3}  =\frac{1}{2}(5x^{3}-3x)\\
        P_{4} =\frac{1}{8}(35x^{4}-30x^{2}+3), \quad P_{5} =\frac{1}{8}(63x^{5}-70x^{3}+15x)
    \end{gather*}
\end{footnotesize}

\subsubsection[Associated Legendre Functions]{Associated Legendre Functions for $x=\cos(\theta)$}
% \begin{footnotesize}
%     \noindent\begin{align*}
%         P_{0}^{0}     & =1                       & P_{2}^{0}     & =\frac{1}{2}(3 \cos^{2}\theta-1)             \\
%         P_{1}^{\pm 1} & =-\sin\theta             & P_{3}^{\pm 3} & =-15 \sin\theta(1-\cos^{2}\theta)            \\
%         P_{1}^{0}     & =\cos\theta              & P_{3}^{\pm 2} & =15 \sin^{2}\theta \cos\theta                \\
%         P_{2}^{\pm 2} & =3 \sin^{2}\theta        & P_{3}^{\pm 1} & =-\frac{3}{2}\sin\theta (5 \cos^{2}\theta-1) \\
%         P_{2}^{\pm 1} & =-3 \sin\theta\cos\theta & P_{3}^{0}     & =\frac{1}{2} (5 \cos^{3}\theta-3 \cos\theta)
%     \end{align*}
% \end{footnotesize}

\begin{footnotesize}
    \noindent\begin{gather*}
        P_{0}^{0}=1, \quad P_{1}^{0}=\cos\theta, \quad P_{1}^{\pm 1} =-\sin\theta, \quad P_{2}^{0} =\frac{1}{2}(3 \cos^{2}\theta-1)\\
        P_{2}^{\pm 1} =-3 \sin\theta\cos\theta, \quad P_{2}^{\pm 2} =3 \sin^{2}\theta, \quad P_{3}^{0} =\frac{1}{2} (5 \cos^{3}\theta-3 \cos\theta) \\
        P_{3}^{\pm 1}  =-\frac{3}{2}\sin\theta (5 \cos^{2}\theta-1),\quad P_{3}^{\pm 2}  =15 \sin^{2}\theta \cos\theta,\\
        P_{3}^{\pm 3}  =-15 \sin\theta(1-\cos^{2}\theta)
    \end{gather*}
\end{footnotesize}