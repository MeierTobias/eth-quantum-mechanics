\section{Mathematical Tools}

\subsection{Operators}
\noindent\begin{align*}
    \hat{Q}(\hat{x},\hat{p}): &  & \widehat{x} & =x & \widehat{p}_x & = -i\hbar \frac{\partial}{\partial x} \\
\end{align*}

\subsection{Probabilities}
The probability of finding the particle between $a$ and $b$ at time $t$ is:
\begin{equation*}
    \int_a^b \underbrace{{\Psi(x,t)}^*\Psi(x,t)}_{|\Psi(x,t)|^2}\,dx
\end{equation*}
\begin{align*}
    \int_{-\infty}^{\infty} |C\cdot \Psi(x,t)|^2 dx                                 = \langle C\Psi|C\Psi\rangle & = 1,\;\Psi\in L_2, \; C\in\mathbb{C} \\
    \frac{d}{dt}\int_{-\infty}^{\infty} |\Psi(x,t)|^2 dx = \frac{d}{dt}\langle\Psi|\Psi\rangle                   & = 0
\end{align*}
\textbf{Remarks:}
\begin{enumerate}
    \item If $\Psi(x,t)$ is normalized at $t=0$ it will stay normalized $\forall t$.
    \item $\Psi \in L_2$ implies $\lim \limits_{x \to \infty}\Psi=0$
\end{enumerate}

\subsubsection{Expectation Value}
Observation of an ensemble of identical systems:
\noindent\begin{align*}
    \langle Q(\hat{x},\hat{p})\rangle & = \int_{-\infty}^{\infty}\Psi^*\hat{Q}(\hat{x},\hat{p})\Psi dx = \langle\Psi|\widehat{Q}\Psi\rangle
\end{align*}

\ptitle{Useful Expectation Values} (Ehrenfest Theorem)
\noindent\begin{align*}
    \langle p \rangle & = m \frac{d\langle x \rangle}{dt}
\end{align*}

\subsubsection{Variance}
\noindent\begin{align*}
    {\sigma_{\widehat{Q}}}^2 & = \langle {\widehat{Q}}^2 \rangle - {\langle \widehat{Q} \rangle }^2 =\left\langle{\left(Q-\langle Q\rangle\right)}^{2}\right\rangle                                                               \\
                             & =\left\langle\Psi\left|{\left(\widehat{Q}-q\right)}^{2}\Psi\right\rangle\stackrel{\text{hermitian}}{=}\left\langle\left(\widehat{Q}-q\right)\Psi\right|\left(\widehat{Q}-q\right)\Psi\right\rangle
\end{align*}


\subsection{Commutators}

\noindent\begin{equation*}
    \left[\widehat{A},\widehat{B}\right] = \widehat{A}\widehat{B} - \widehat{B}\widehat{A} = -\left[\widehat{B},\widehat{A}\right]
\end{equation*}
\noindent\begin{align*}
    \left[\widehat{A}\widehat{B},\widehat{C}\right]  & =\widehat{A}\left[\widehat{B},\widehat{C}\right]+\left[\widehat{A},\widehat{C}\right]\widehat{B}     \\
    \left[\widehat{A}+\widehat{B},\widehat{C}\right] & = \left[\widehat{A},\widehat{C}\right]+\left[\widehat{B},\widehat{C}\right]                          \\
    \left[\widehat{A},\widehat{B}^2\right]           & = \left[\widehat{A},\widehat{B}\right] \widehat{B} + \widehat{B}\left[\widehat{A},\widehat{B}\right]
\end{align*}

\ptitle{Canonical Commutation Relations}

\noindent\begin{align*}
    \left[\widehat{x},\widehat{p}_x\right]  & = i\hbar  & \left[\widehat{x}, \widehat{y}\right]     & = 0 \\
    \left[\widehat{p}_x, \widehat{x}\right] & = -i\hbar & \left[\widehat{p}_x, \widehat{p}_y\right] & = 0
\end{align*}
\ptitle{Useful Commutators}

\noindent\begin{align*}
    \left[c,\hat B\right]                        & =0                     & \left[\widehat{H},\widehat{x}\right]     & =-\frac{i\hbar}{m}\widehat{p} \\
    \left[\widehat{a}_{-},\widehat{a}_{+}\right] & = 1                    & \left[\widehat{H}, \widehat{V}\right]    & \neq 0                        \\
    \left[\widehat{x}^n,\widehat{p}\right]       & = i \hbar n x^{n-1}    & \left[\widehat{V},\widehat{x}\right]     & =0                            \\
    \left[\widehat{x},\widehat{p}^2\right]       & = 2i\hbar\widehat{p}   & \left[\widehat{L}_x,\widehat{L}_y\right] & =i\hbar\widehat{L}_z          \\
    \left[f(\widehat{x}),\widehat{p}\right]      & = i \hbar\frac{df}{dx} &                                          &
\end{align*}

\textbf{Remark}: Evaluate $\widehat{A}\widehat{B}f - \widehat{B}\widehat{A}f$ and not $(\widehat{A}\widehat{B} - \widehat{B}\widehat{A})f$.


\subsection{Dirac Notation}
Assuming the functions $f,g$ are square integrable i.e.\ in $L_2$ (\textbf{Hilbert space}) $\mathcal{H}$.

\noindent\begin{align*}
    \left|f\right\rangle  & := \begin{bmatrix}
                                   c_1 & c_2 & \cdots & c_n
                               \end{bmatrix}^T                               &  & \text{``ket''}              \\
    \left\langle f\right| & := \begin{bmatrix}
                                   {c_1}^* & {c_2}^* & \cdots & {c_n}^*
                               \end{bmatrix}                   &  & \text{``bra''}                            \\
    \langle f|g \rangle   & := \int_{-\infty}^{\infty} f^* g\; d \mathbf{r} \in L_2 &  & \text{inner product}
\end{align*}
\ptitle{ONB}
\noindent\begin{align*}
    \langle f_m|f_n \rangle                                       & = \delta_{mn}                        &  & \text{orthonormal}        \\
    f(x)                                                          & = \sum_{n=1}^{\infty} c_n f_n(x)     &  & \text{complete}           \\
    f^*(x)                                                        & = \sum_{n=1}^{\infty} c_n^* f_n^*(x) &  &                           \\
    c_n                                                           & = \langle f_n|f \rangle              &  & \text{projection on base} \\
    \sum_{j=1}^{n} \left|f_j\right\rangle \left\langle f_j\right| & = \mathbf{I}                         &  & \text{closure relation}
\end{align*}

\ptitle{Properties of Inner Product}
\noindent\begin{align*}
    \langle \lambda f|g \rangle & =\lambda^* \langle f|g \rangle               & \langle f|\lambda g \rangle & =\lambda \langle f|g \rangle \\
    \langle f+g|h \rangle       & =\langle f|h \rangle + \langle g|h \rangle   & \langle g|f \rangle         & = {\langle f|g \rangle}^*    \\
    \langle f|f \rangle         & \ge 0                                        & \langle f|f \rangle         & = 0 \leftrightarrow f=0      \\
    |\langle f|g \rangle |^2    & \leq \langle f|f \rangle \langle g|g \rangle
\end{align*}

\subsection{Observables}
Observables are represented by hermitian operators.
\begin{itemize}
    \item Eigenstate:\ \ref{midterm:det_states} \&\ \ref{midterm:eig_fun}
    \item Else:\ \ref{midterm:gen_stat}
\end{itemize}

\subsubsection{Hermitian Operators}
The expectation value of a observable is real and thus
\noindent\begin{align*}
    \langle Q\rangle                            & = {\langle Q\rangle}^*                                            \\
    \langle Q\rangle=\langle f|\hat{Q} g\rangle & = \langle \hat{Q}f|g\rangle={\langle Q\rangle}^*\quad \forall f,g
\end{align*}

These operators are \textbf{linear} if
\noindent\begin{equation*}
    \hat{Q}\left[af(x)+bg(x)\right]=a\hat{Q}f(x)+b\hat{Q}g(x),
\end{equation*}

\ptitle{Properties}

\noindent\begin{align*}
    \langle f|(\hat{Q} + \hat{R})g\rangle                                    & = \langle (\hat{Q} + \hat{R})f|g\rangle                                                               \\[0.75em]
    \langle f|\hat{Q}\hat{R}g\rangle       =\langle \hat{Q}f|\hat{R}g\rangle & =  \langle \hat{R}\hat{Q}f|g\rangle \overset{[\hat{Q},\hat{R}]=0}{=} \langle \hat{Q}\hat{R}f|g\rangle
\end{align*}

\subsubsection{Determinant States}\label{midterm:det_states}
If an ensemble is in an \textbf{determinant state}, every measurement yields the same result:
\noindent\begin{align*}
    \sigma^2     & = \langle \Psi|(\hat{Q} - q) \Psi\rangle = 0 \\
    \hat{Q} \Psi & = q \Psi
\end{align*}
These determinant states of $Q$ are \textit{eigenfunctions} of $\hat{Q}$ and the expectation $\langle Q\rangle = q$ the corresponding \textit{eigenvalue}.

\textbf{Remarks:}
\begin{itemize}
    \item If multiple eigenfunctions share their eigenvalue, they are called \textit{degenerated states}.
    \item The TISE is an example for a determinant state\newline
          $\hat{Q}: \hat{H},\; q:E$.
\end{itemize}

\subsubsection{Eigenfunctions of a Hermitian Operator}\label{midterm:eig_fun}
A set of eigenvalues $q$ for $\hat{Q}$ is called its \textit{spectrum}.

\newpar{}
\ptitle{Discrete Spectrum}

If $\psi_n$ are solutions to $\hat{Q}\psi_n=q\psi_n$ then
\noindent\begin{equation*}
    \Psi(x,0)     = \sum_{n=1}^{\infty} c_n \underbrace{\psi_n(x)}_{\textsf{eigenfunctions}} \;\Leftrightarrow\; |\Psi\rangle  = \sum_{n=1}^{\infty} c_n \underbrace{|\psi_n\rangle}_{\textsf{eigenvectors}}
\end{equation*}

\begin{itemize}
    \item Eigenvalues are real
    \item $\in \mathcal{H}$, represent \textbf{physical} states and form a complete set.
    \item Eigenfunctions with different eigenvalues are orthogonal
          \noindent\begin{equation*}
              \langle f_m|f_n\rangle=\delta_{mn}
          \end{equation*}
    \item Eigenvalue equation for \textbf{position}:
          \noindent\begin{equation*}
              \widehat{x}g_{x'}= x'g_{x'}\;\Leftrightarrow\; g_{x'}=\delta(x-x')
          \end{equation*}
\end{itemize}

\newpar{}
\ptitle{Continuous Spectrum}
\begin{itemize}
    \item Eigenfunctions with a continuous spectrum are not normalizable ($\notin \mathcal{H}$)
    \item These eigenfunctions with real eigenvalues form a \textbf{complete set} and are \textit{Dirac-orthonormalizable}:
          \noindent\begin{equation*}
              \langle f_{x'}|f_{x''}\rangle=\delta(x''-x')
          \end{equation*}
    \item Eigenvalue equation for \textbf{momentum}:
          \noindent\begin{equation*}
              \widehat{p}g_{p'}= p'g_{p'}\;\Leftrightarrow\; g_{p'}=\frac{1}{\sqrt{2\pi\hbar}} \exp\left(\frac{ip'x}{\hbar}\right), \quad \lambda = \frac{2\pi\hbar}{p'}
          \end{equation*}
\end{itemize}

\subsubsection{Generalized Statistical Interpretation}\label{midterm:gen_stat}
If the spectrum of $\hat{Q}$ is discrete and the state
\begin{equation*}
    \Psi=\sum_n c_n \psi_n
\end{equation*}
is \textbf{not a eigenfunction} of $\hat{Q}$ (but $\psi_n$ is):
\noindent\begin{equation*}
    \underbrace{|c_n|^2}_{\textsf{Probability}}, \quad c_n = \underbrace{\langle \psi_n|\Psi\rangle}_{\textsf{Projection}} =\int_{-\infty}^{\infty} {\psi_n}^* \Psi\; d^3 \mathbf{r}
\end{equation*}
As a result:
\noindent\begin{equation*}
    \langle Q\rangle  = \sum_{n=1}^{\infty} |c_n|^2 q_n, \quad  \sum_{n=1}^{\infty} |c_n|^2 = 1, \sigma^2          \neq 0
\end{equation*}

\textbf{Wave Function Collapse}

After a measurement of an observable, the wave function \textbf{collapses} into the measured eigenfunction $\psi_n$ with eigenvalue $q_n$ and probability $|c_n|^2$. This collapse reflects the transition from a superposition of possible states to a \textbf{definitive state}.

\subsection{Postulates}

\begin{enumerate}
    \item The state of a QM system is described by $\Psi(\mathbf{r},t)$, and $|\Psi|^2\; d^3 \mathbf{r}$ is the probability of finding the particle in the volume $d^3 \mathbf{r} = dx\,dy\,dz$.
    \item Every observable quantity in classical mechanics is represented by a \textbf{linear hermitian operator}, $\hat{Q}$, such that the mean value of the observable from an ensemble is
          \noindent\begin{equation*}
              \langle Q\rangle=\int\Psi^{*}\hat{Q}\Psi d^{3} \mathbf{r}= \langle\Psi|\hat{Q}\Psi\rangle
          \end{equation*}\newline
          \textbf{Restated}:\newline
          If the system is in a state that is an eigenstate of $\hat{Q}$, a measurement on a QM ensemble will always yield the eigenvalue $q$ of $\hat{Q}$ (e.g.\ determinant state of the infinite square well).\newline
          Else, a single measurement will yield one of the eigenvalues $q_n$ of $\hat{Q}$ with probability $|c_n|^2$ (e.g.\ general state of infinite square well).
    \item $\Psi(\mathbf{r},t)$ evolves in time according to the TDSE:
          \noindent\begin{align*}
              i\hbar \frac{\partial \Psi}{\partial t} & =\hat{H}\Psi                     \\
              \hat{H}                                 & = \frac{\hat{p}^2}{2m} + \hat{V}
          \end{align*}
\end{enumerate}

\subsection{Fundamental QM Relations}

\ptitle{Black Body Radiation}

\begin{equation*}
    k_B T_{low} \ll h\nu_{high}
\end{equation*}

\ptitle{De Broglie}

\begin{equation*}
    p=\frac{h}{\lambda}=\frac{2\pi\hbar}{\lambda}=\hbar\cdot k
\end{equation*}
\textbf{Remarks:}
\begin{enumerate}
    \item Connects the domains of waves ($\lambda$) and particles ($p$).
    \item QM becomes relevant if $\lambda>d$ (item's characteristic size).
\end{enumerate}

\subsection{Generalized Uncertainty Principle}
\subsubsection{Compatible Observables}

\begin{itemize}
    \item $[\widehat{A}, \widehat{B}] = 0$, same eigenfunctions
    \item Can be determined simultaneously:
          \begin{enumerate}
              \item Measuring $A$ yields eigenvalue $a_n$ with probability $|c_{n,A}|^2$ and $\sigma_A=0$
              \item Wave function collapses into the corresponding eigenfunction $\psi'$
              \item Subsequent measurements for $A$ or $B$ will yield eigenvalues $a_n, b_m$, $\sigma_A=0, \sigma_B=0$ $\Rightarrow$ \textbf{deterministic}
          \end{enumerate}
\end{itemize}

\subsubsection{Incompatible Observables}

\begin{itemize}
    \item $[\widehat{A}, \widehat{B}] \neq 0$, different eigenfunctions
    \item Can't be determined simultaneously without any uncertainty:
          \begin{enumerate}
              \item Measuring $A$ yields eigenvalue $a_n$ with probability $|c_{n,A}|^2$ and $\sigma_A=0$
              \item Wave function collapses into the corresponding eigenfunction $\psi'$
              \item Subsequent measurements
                    \begin{itemize}
                        \item for $A$ will yield eigenvalues $a_n$, $\sigma_A=0$ $\Rightarrow$ \textbf{deterministic}
                        \item for $B$ will yield eigenvalue $b_m$ with probability $|c_{m,B}|^2$ (eigenfunction is not shared) $\Rightarrow$ \textbf{probabilistic}
                    \end{itemize}
          \end{enumerate}
\end{itemize}

\textbf{Remarks}:
\begin{itemize}
    \item The previous statements hold for two subsequent measurements, if for example, a third measurement is made, the compatability of the second and third have to be evaulated.
    \item $\langle[\hat{A},\hat{B}]\rangle=\langle\Psi|\hat{A}\hat{B}\Psi\rangle-\langle\Psi|\hat{B}\hat{A}\Psi\rangle$
\end{itemize}

\newpar{}
\ptitle{General Uncertainty Principle (GUP)}

For any Observables $A$, $B$:
\begin{equation*}
    \sigma_A^2\cdot\sigma_B^2\geqslant{\underbrace{\left(\frac1{2i}\langle[\hat{A},\hat{B}]\rangle\right)}_{\textsf{real}}}^2
\end{equation*}

Therefore the position and momentum operator yield the following uncertainty relation:
\begin{equation*}
    \sigma_x\sigma_p \geq \frac{\hbar}{2}
\end{equation*}

\subsubsection{Time-Energy Uncertainty}
Plugging the \textbf{generalized Ehrenfest theorem}
\begin{equation*}
    \frac{d}{dt}\left\langle Q\right\rangle=\frac{i}{\hbar}\left\langle\left[\hat{H},\hat{Q}\right]\right\rangle+\left\langle\frac{\partial\hat{Q}}{\partial t}\right\rangle
\end{equation*}
into the GUP yields, assuming $\hat{Q}$ is time-independent, the \textbf{time-energy uncertainty}:
\begin{equation*}
    \Delta t\cdot\Delta E\geqslant\frac\hbar2 \qquad \left|\begin{matrix}
        \Delta t & \equiv \frac{\sigma_Q}{|d\langle Q\rangle/dt|} \\
        \Delta E & \equiv \sigma H
    \end{matrix}\right.
\end{equation*}
where

\ptitle{Remarks}

\begin{itemize}
    \item $\Delta t$ describes how long it takes the expectation value of $Q$ to change by $\sigma$.
    \item If they commute, $Q$ is a conserved quantity.
\end{itemize}

\subsection{Properties of Even and Odd Functions}
For functions $f_1, f_2$ (even) and $g_1, g_2$ (odd):
\newpar{}
\begin{tabularx}{\linewidth}{@{}XXX@{}}
    Multiplication         & Addition          & Differentiation \\
    \midrule{}
    $f_1\cdot f_2$ is even & $f_1+f_2$ is even & $f'$ is odd     \\
    $g_1\cdot g_2$ is even & $g_1+g_2$ is odd  & $g'$ is even    \\
    $f\cdot g$ is odd      &                   &
\end{tabularx}


\newcol{}
\section{Schroedinger Equation}
\noindent\begin{equation*}
    i\hbar \frac{\partial \Psi(x,t)}{\partial t}           = \underbrace{- \frac{\hbar^2}{2m} \frac{\partial^2 \Psi(x,t)}{\partial x^2}}_{E_{kin}} + \underbrace{V(x,t)\Psi(x,t)}_{E_{pot}}
\end{equation*}

\subsection{Stationary States}
\noindent\begin{align*}
    \Psi_n(x,t)                                                        & = \psi_n(x)\varphi_n(t)                                                                                         \\
    \underbrace{i\hbar\frac1\varphi\frac{d\varphi}{dt}}_{\varphi_n(t)} & =\underbrace{-\frac{\hbar^2}{2m}\frac1\psi\frac{d^2\psi}{dx^2}+V}_{\psi_n (x)} = \underbrace{E}_{\text{const.}}
\end{align*}

\ptitle{Phase Factor}
\noindent\begin{equation*}
    \varphi_n(t) =\exp\left[\frac{-iE_n t}{\hbar}\right], \qquad E_n = \frac{\hbar^2 k_n^2}{2m}
\end{equation*}

\ptitle{TISE}
\noindent\begin{equation*}
    \widehat{H}\psi  = \Bigl[\overbrace{\underbrace{-\frac{\hbar^2}{2m}\frac{d^2}{dx^2}}_{\frac{1}{2m}\hat{p}^2}}^{E_{kin}}+ \overbrace{V(x)}^{E_{pot}}\Bigr]\psi = E\psi
\end{equation*}

\newpar{}
Assuming $V(x) = 0$:
\noindent\begin{align*}
    \psi(x) & =A\sin(kx)+B\cos(kx)                                               &  & \text{standing wave (ISW)} \\
    \psi(x) & =Ce^{ikx}+De^{-ikx}                                                &  & \text{free particle}       \\
    k       & =\sqrt{\frac{2mE}{\hbar^{2}}}=\frac{p}{\hbar}=\frac{2\pi}{\lambda} &  & \text{wave number}
\end{align*}

\ptitle{Hamiltonian}

$\widehat{H}$ is the Hamiltonian operator representing \textbf{total energy}:
\noindent\begin{equation*}
    \langle H\rangle = E,\quad{\sigma_H}^2 = 0, \quad \langle p\rangle = 0
\end{equation*}

\newpar{}

\textbf{Remarks:}
\begin{itemize}
    \item $|\Psi(x,t)|^2 = |\psi(x)|^2$ (prob.\ density and expectation values are time-independent):
          \noindent\begin{equation*}
              \langle Q\rangle=\int_{-\infty}^\infty\psi^*(x)\widehat{Q}\psi(x)dx
          \end{equation*}
    \item The energies $E_n$ can be obtained with diagonalization
          \noindent\begin{equation*}
              \det\left(\mathbf{H}-E_n \mathbf{I}\right) = \mathbf{0} \qquad \mathbf{H}_{jk} = \left\langle f_j\right|\widehat{H}\left|f_k \right\rangle
          \end{equation*}
\end{itemize}

\subsubsection{Combining the Separable Solutions}
A particle/wave in a general state $\Psi$ is in a superposition of stationary states $\psi_n$ with energy $E_n$ and probability $|c_n|^2$:
\noindent\begin{equation*}
    \Psi(x,t) =\sum_{n=1}^\infty c_n\psi_n(x) \underbrace{\exp\left[\frac{-iE_n t}{\hbar}\right]}_{\varphi_n(t)}=\sum_{n=1}^\infty c_n\Psi_n(x,t)
\end{equation*}

\noindent\begin{equation*}
    \langle H\rangle=\sum_{n=1}^\infty|c_n|^2\cdot \underbrace{E_n}_{\textsf{Eigenvalue}} \qquad \sum_{n=1}^\infty|c_n|^2 =1
\end{equation*}

\ptitle{Remarks:}
\begin{itemize}
    \item A Measurement will \textbf{collapse} $\Psi$ into one of the $\Psi_n$'s.
    \item Probabilities and expectations of the general solution are in general \textbf{not} time-independent.
    \item As $c_n$ are independent of time, the probability to get a certain energy and $\langle H\rangle$ are constant in time (energy conservation).
\end{itemize}

\subsection{Free Particle}
A \textit{free particle} propagates in space as a \textbf{wave packet} and its wavefunction is determined by
\noindent\begin{equation*}
    \Psi_{wp}(x,t)=\frac{1}{\sqrt{2\pi}}\int_{-\infty}^{\infty}g(k)\exp\Biggl[i\Biggl(kx- \underbrace{\frac{\hbar k^{2}}{2m}}_{\omega}t\Biggr)\Biggr]dk
\end{equation*}

\noindent\begin{equation*}
    g(k)=\frac{1}{\sqrt{2\pi}}\int_{-\infty}^{\infty}\Psi_{wp}(x,0)e^{-ikx}dx
\end{equation*}
\begin{equation*}
    k  = \frac{2\pi}{\lambda}\qquad p  = \hbar k
\end{equation*}

\textbf{Remarks:}

\begin{itemize}
    \item A free particle can have any positive energy $E$.
    \item There is no free particle with a definite energy.
          % \item $\frac{1}{\sqrt{2\pi}}$ plays the role of $c_n$ from the discrete superposition case.
    \item Separable solutions do not represent physical states, therefore $\notin$ Hilbert space.
    \item The particle carries a range of $k$
\end{itemize}

\ptitle{Group Velocity}

A wavepacket consisting of infinitely many waves has a different \textit{group velocity} than the \textit{phase velocity} of the individual waves:
\noindent\begin{equation*}
    v_{group} = v_{classical} = 2v_{phase}
\end{equation*}

\begin{center}
    \includegraphics[width = 0.4\linewidth]{group_vel.png}
\end{center}

\subsection{ISW}
\noindent\begin{equation*}
    V(x)=\begin{cases}0,&0\le x\le a\\\infty,&\text{otherwise}\end{cases}
\end{equation*}
\noindent\begin{align*}
    \psi_{n}(x) & =\sqrt{\frac{2}{a}}\sin\left(\frac{n\pi}{a}x\right)                      &  & \text{standing wave} \\
    E_{n}       & =\frac{\hbar^{2}k_{n}^{2}}{2m}=\frac{n^{2}\pi^{2}\hbar^{2}}{2ma^{2}} > 0 &  & \text{quantized}
\end{align*}

\begin{center}
    \includegraphics[width = 0.4\linewidth]{ISW.png}
\end{center}

\textbf{Remarks:}
\begin{itemize}
    \item Solutions alternate between \textit{odd} and \textit{even} starting from $\psi_1$: \textit{odd}
    \item $\psi_{n+1}$ has one node (zero-crossing) more than $\psi_n$
    \item $\psi_n$ are eigenfunctions/vectors and $E_n$ are the corresponding eigenvalues
    \item $\psi_n$ are orthonormal base functions (complete):
          \noindent\begin{equation*}
              \int_{-\infty}^{\infty} \psi_m^*\psi_n\; dx =\langle \psi_m|\psi_n\rangle= \delta_{mn}
          \end{equation*}
          % \item \textbf{Dirichlet's theorem}: any function can be described by a fourier series
\end{itemize}

\ptitle{General Solution}
\noindent\begin{align*}
    \Psi(x,t) & =\sum_{n=1}^{\infty}c_{n} \overbrace{\underbrace{\sqrt{\frac{2}{a}}\sin\left(\frac{n\pi}{a}x\right)}_{\psi_n} \underbrace{\exp\left[-i\frac{n^{2}\pi^{2}\hbar}{2ma^{2}}t\right]}_{\varphi_n}}^{\Psi_n} \\
    c_n       & =\langle \psi_n|\Psi(x,0)\rangle = \sqrt{\frac{2}{a}}\int_0^a\sin\left(\frac{n\pi}{a}x\right)\Psi(x,0)dx
\end{align*}

\ptitle{Expectation Values}
\noindent\begin{align*}
    \langle \varphi_n|x^2\varphi_m\rangle & = \frac{2a^2}{\pi^2}\left(\frac{{(-1)}^{n-m}}{{(n-m)}^2}-\frac{{(-1)}^{n+m}}{{(n+m)}^2}\right) \\
    \langle \varphi_n|x^2\varphi_n\rangle & = a^2\left(\frac{1}{3} -\frac{1}{2{(n\pi)}^2}\right)
\end{align*}

\subsection{Harmonic Oscillator}
% Analogon to mechanical frictionless spring and mass system.
\begin{equation*}
    V(x) = \frac{1}{2}k x^2 = \frac{1}{2}m \omega^2 x^2 \qquad \text{with} \quad \omega = \sqrt{\frac{k}{m}}
\end{equation*}

\ptitle{TISE}
\noindent\begin{equation*}
    \overbrace{\underbrace{\frac{-\hbar^2}{2m}\frac{d^2\psi}{dx^2}}_{\frac{1}{2m}\hat{p}^2\psi} + \frac{1}{2}m \omega^2 x^2 \psi}^{\widehat{H}\psi}  = E\psi
\end{equation*}

\ptitle{Ladder Operators}
\begin{align*}
    \widehat{a}_{+} & = \frac{1}{\sqrt{2\hbar m \omega}}\left(-i\widehat{p}+m\omega\widehat{x}\right) &  & \text{raising operator}  \\
    \widehat{a}_{-} & = \frac{1}{\sqrt{2\hbar m \omega}}\left(+i\widehat{p}+m\omega\widehat{x}\right) &  & \text{lowering operator}
\end{align*}
\noindent\begin{gather*}
    \widehat{a}_{-}\widehat{a}_{+}=\frac{1}{\hbar\omega}\widehat{H}+\frac{1}{2} \quad\quad \widehat{a}_{+}\widehat{a}_{-}=\frac{1}{\hbar\omega}\widehat{H}-\frac{1}{2} \\
    \widehat{a}_{+} + \widehat{a}_{-} = \sqrt{\frac{2m\omega}{\hbar}} \widehat{x} \\
    \left[\widehat{a}_{-},\widehat{a}_{+}\right] = 1
\end{gather*}

\ptitle{Ladders and TISE}
\begin{align*}
    \widehat{H}(\widehat{a}_{+}\psi_n) & = (E_n+\hbar\omega)\widehat{a}_{+}\psi_n \\
    \widehat{H}(\widehat{a}_{-}\psi_n) & = (E_n-\hbar\omega)\widehat{a}_{-}\psi_n
\end{align*}

\ptitle{Solution to TISE}
\noindent\begin{align*}
    \psi_0 & = {\left(\frac{m\omega}{\pi\hbar}\right)}^{\frac{1}{4}}\exp\left(\frac{-m\omega}{2\hbar}x^2\right) & E_0 & = \frac{1}{2}\hbar\omega     \\
    \psi_n & = \frac{1}{\sqrt{n!}}{\left(\widehat{a}_{+}\right)}^n \psi_0                                       & E_n & = (n+\frac{1}{2})\hbar\omega
\end{align*}
\begin{align*}
    \widehat{a}_{-}\psi_0                & = 0\psi_0 = 0                                      \\
    \widehat{a}_{+}\psi_n                & = \sqrt{n+1}\:\psi_{n+1}                           \\
    \widehat{a}_{-}\psi_n                & = \sqrt{n}\:\psi_{n-1}                             \\
    \widehat{a}_{+}\widehat{a}_{+}\psi_n & = \sqrt{n+1}\sqrt{n+2}\:\psi_{n+2}                 \\
    \widehat{a}_{-}\widehat{a}_{-}\psi_n & = \sqrt{n-1}\sqrt{n}\:\psi_{n-2}                   \\
    \widehat{a}_{+}\widehat{a}_{-}\psi_n & = \sqrt{n}\sqrt{n}\:\psi_{n} = n \:\psi_{n}        \\
    \widehat{a}_{-}\widehat{a}_{+}\psi_n & = \sqrt{n+1}\sqrt{n+1}\:\psi_{n} = (n+1)\:\psi_{n}
\end{align*}

\begin{center}
    \includegraphics[width=.8\linewidth]{harmonic_oscillator.png}
\end{center}

\ptitle{Hermite Polynomial Form} (\textit{Useful Info Sheet})

\noindent\begin{align*}
    H_n & = {(-1)}^n e^{x^2} \frac{d^n}{dx^n} e^{-x^2}                                           \\
    H_1 & = 2x                                         & H_3 & = 8x^3-12x       &  & \text{odd}  \\
    H_2 & = 4x^2-2                                     & H_4 & = 16x^4-48x^2+12 &  & \text{even}
\end{align*}

\ptitle{Remarks:}
\begin{itemize}
    \item The energy is quantized.
    \item $n$ is the number of energy quanta in the oscillator
    \item $\widehat{N} \equiv \widehat{a}_{+}\widehat{a}_{-} = $ number operator because $\left<N\right> = n$ for $\psi_n$
    \item The lowest state $\psi_0$ has the \textbf{zero point energy}. Even if all the energy is removed from the system (zero Kelvin).
    \item Solutions alter between even and odd because the potential is \textbf{symmetric}.
    \item $\psi_{n+1}$ has one node more than $\psi_n$
    \item $\psi_n$ are mutually orthogonal + complete (form a basis).
    \item Soft boundary conditions - finite potential at the boundary/barrier.
    \item Negative kinetic energy outside the potential parabola.
    \item The spacing between two energy levels spreads out for large $k$ as $\omega=\sqrt{\frac{k}{m}}$.
\end{itemize}

\ptitle{Specific Expectation Values}
\noindent\begin{align*}
    \left\langle x \right\rangle _n & = 0 & \left\langle x^2 \right\rangle _n & = \left(n+\frac{1}{2}\right)\frac{\hbar}{m\omega} & \forall n \in \mathbb{N} \\
    \left\langle p \right\rangle _n & = 0 & \left\langle p^2 \right\rangle _n & = \left(n+\frac{1}{2}\right)\hbar m\omega         & \forall n \in \mathbb{N}
\end{align*}
\noindent\begin{align*}
    \left\langle T \right\rangle _n & = \frac{\left\langle p^2 \right\rangle}{2m} =  \frac{\hbar \omega}{2}\left(n+\frac{1}{2}\right)        & \langle E\rangle_n & =  \langle T\rangle_n +\langle V\rangle_n \\
    \left\langle V \right\rangle _n & = \frac{m\omega^2}{2}\left\langle x^2 \right\rangle = \frac{\hbar \omega}{2}\left(n+\frac{1}{2}\right) & \langle T\rangle_n & =  \langle V\rangle_n
\end{align*}

\subsection{Finite Potentials}
\noindent\begin{equation*}
    \begin{cases}
        E > V(+\infty)\text{ or } E > V(-\infty)  & \text{scattering state} \\
        E < V(+\infty)\text{ and } E < V(-\infty) & \text{bound state}
    \end{cases}
\end{equation*}
\textbf{Remarks}:

\begin{itemize}
    \item The total energy of the particle is conserved i.e.\ if the potential $V$ increases, the wavelength $\lambda$ (and so the velocity!) decreases and vice versa.
          \noindent\begin{equation*}
              k=\sqrt{\frac{2m(E-V_0)}{\hbar^2}} = \frac{2\pi}{\lambda}
          \end{equation*}
    \item If a particle penetrates into a potential higher than its own energy level it will take negative kinetic energy to conserve energy.
\end{itemize}

\subsubsection{Solving TISE}
\begin{enumerate}
    \item Split up into regions
    \item Solve TISE for regions
          \begin{itemize}
              \item $V=V_i$:
                    \noindent\begin{equation*}
                        -\frac{\hbar^2}{2m}\frac{d^2}{dx^2} \psi + V_i\psi=E\psi
                    \end{equation*}
              \item $V=\infty$: no solution
          \end{itemize}
    \item match solutions at interfaces between regions using boundary conditions:
          \begin{itemize}
              \item $\psi(x)$ continous and finite
              \item $\frac{d\psi}{dx}$ continous
          \end{itemize}
\end{enumerate}

\subsubsection{Tunneling}
Particle with $0<E<V_0$. Transmission coefficient $T$:
\noindent\begin{equation*}
    T\approx\frac{16E(V_0-E)}{V_0^2}\exp\Biggl[-4 \underbrace{\frac{\sqrt{2m(V_0-E)}}{\hbar}}_{k} a\Biggr]
\end{equation*}

\section{Tables}
\subsection{Useful Integrals}
\subsubsection{Exponentials}
\begin{footnotesize}
    \noindent\begin{align*}
        %  & \int e^{\lambda x}dx           &  & =\frac{1}{\lambda }e^{\lambda x}+C                                                 \\
         & \int a^{\lambda x}dx           &  & =\frac{1}{\lambda \cdot \ln(a)}a^{\lambda x}+C                                     \\
         & \int e^{\lambda x}\sin(ax+b)dx &  & =\frac{e^{\lambda x}}{a^2+\lambda ^2}\left(\lambda \sin(ax+b)-a\cos(ax+b)\right)+C \\
         & \int e^{\lambda x}\cos(ax+b)dx &  & =\frac{e^{\lambda x}}{a^2+\lambda ^2}\left(\lambda \cos(ax+b)+a\sin(ax+b)\right)+C \\
         & \int x \cdot e^{\lambda x}dx   &  & =(\frac{\lambda x-1}{\lambda ^2})\cdot e^{\lambda x}+C                             \\
         & \int x^2 \cdot e^{\lambda x}dx &  & =(\frac{\lambda ^2x^2-2\lambda x+2}{\lambda ^3})\cdot e^{\lambda x}                \\
         & \int x\cdot e^{x^2}dx          &  & =\frac{1}{2}\cdot e^{x^2}+C
    \end{align*}
\end{footnotesize}

\paragraph{Specific Intervals}
more on \textit{Useful Info Sheet}
\begin{footnotesize}
    \noindent\begin{align*}
        %  & \int_0^{\infty} e^{-\lambda x}x^n dx                                               &  & =\frac{n!}{\lambda ^{n+1}},\quad \lambda >0                                       \\
        %  & \int_0^{\infty} e^{-\lambda x^2} dx                                                &  & =\frac{1}{2}\sqrt{\frac{\pi}{\lambda }},\quad \lambda >0                          \\
        %  & \int_{-\infty}^{\infty} e^{-\lambda x^2} dx                                        &  & =\sqrt{\frac{\pi}{\lambda }},\quad \lambda >0                                     \\
         & \int_{-\infty}^{\infty}e^{-ax^2}e^{-iwx} dx                                        &  & = \sqrt{\frac{\pi}{a}}e^{\frac{w^2}{4a}}                                          \\
        %  & \int_{-\infty}^{\infty}e^{-(ax^2+bx+c)}dx                                          &  & = \sqrt{\frac{\pi}{a}}e^{\frac{b^2}{4a}-c}                                        \\
         & \int_{-\infty}^{\infty}x^2e^{-\lambda x^2}dx                                       &  & = \frac{1}{2\lambda }\sqrt{\frac{\pi}{\lambda }},\quad \mathrm{Re}\{\lambda \} >0 \\
        %  & \int_0^\infty x^n e^{-ax}dx                                                        &  & =\frac{n!}{a^{n+1}},\quad n\in\mathbb{N}^+                                        \\
        %  & \int_{0}^{a} x\sin^2\left(\frac{n\pi}{x}x\right)\; dx                              &  & =\frac{a^2}{4}                                                                    \\
        %  & \int_0^a x^2\cdot\sin^2\left(\frac{n\pi}{a}x\right)dx                              &  & ={\left(\frac{a}{2\pi n}\right)}^3\left(\frac{4\pi^3n^3}{3}-2n\pi\right)          \\
         & \int_{0}^{a} x\sin\left(\frac{\pi}{x}x\right)\sin\left(\frac{2\pi}{x}x\right)\; dx &  & = -\frac{8a^2}{9\pi^2}
    \end{align*}
\end{footnotesize}

%% not needed in Midterm %%

% \subsection{Legendre Polynomials}
% \noindent\begin{align*}
%     P_{0} & =1                     & P_{3} & =\frac{1}{2}(5x^{3}-3x)           \\
%     P_{1} & =x                     & P_{4} & =\frac{1}{8}(35x^{4}-30x^{2}+3)   \\
%     P_{2} & =\frac{1}{2}(3x^{2}-1) & P_{5} & =\frac{1}{8}(63x^{5}-70x^{3}+15x)
% \end{align*}
% \noindent\begin{align*}
%     P_{0}^{0} & =1                       & P_{2}^{0} & =\frac{1}{2}(3 \cos^{2}\theta-1)             \\
%     P_{1}^{1} & =-\sin\theta             & P_{3}^{3} & =-15 \sin\theta(1-\cos^{2}\theta)            \\
%     P_{1}^{0} & =\cos\theta              & P_{3}^{2} & =15 \sin^{2}\theta \cos\theta                \\
%     P_{2}^{2} & =3 \sin^{2}\theta        & P_{3}^{1} & =-\frac{3}{2}\sin\theta (5 \cos^{2}\theta-1) \\
%     P_{2}^{1} & =-3 \sin\theta\cos\theta & P_{3}^{0} & =\frac{1}{2} (5 \cos^{3}\theta-3 \cos\theta)
% \end{align*}