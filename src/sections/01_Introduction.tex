\section{Introduction}
\subsection{Statistics}
Given a probability density $\rho$ the probability of obtaining $x$ in the interval $x+dx$ is given by $\rho dx$ and hence the probability of finding it in [$a,b$] is given by
\begin{equation*}
    P_{ab}=\int_{a}^{b} \rho(x)\, dx
\end{equation*}
The probability density $\rho$ is called \textbf{normalized} if
\begin{equation*}
    \int_{-\infty}^{\infty}\rho(x)dx = 1
\end{equation*}
The \textbf{expectation value} $\langle x \rangle$ can be calculated with
\begin{equation*}
    \langle x \rangle = \int_{-\infty}^{\infty} x\rho(x) dx
\end{equation*}
The \textbf{variance} ${\sigma_x}^2$ and \textbf{standard deviation} $\sigma_x$ is given by
\begin{equation*}
    {\sigma_x}^2 = \langle x^2 \rangle - {\langle x \rangle }^2
\end{equation*}

\subsection{Schrödinger Equation for Matter}
\begin{align*}
               &  & i\hbar \frac{\partial \Psi(\mathbf{r},t)}{\partial t} & = \underbrace{- \frac{\hbar^2}{2m} \nabla^2 \Psi(\mathbf{r},t)}_{E_{kin}} + \underbrace{V(\mathbf{r},t)\Psi(\mathbf{r},t)}_{E_{pot}} \\
    \text{1D:} &  & i\hbar \frac{\partial \Psi(x,t)}{\partial t}          & = \underbrace{- \frac{\hbar^2}{2m} \frac{\partial^2 \Psi(x,t)}{\partial x^2}}_{E_{kin}} + \underbrace{V(x,t)\Psi(x,t)}_{E_{pot}}
\end{align*}

\renewcommand{\arraystretch}{1.3}
\setlength\tabcolsep{6pt} % default value: 6pt
\begin{tabularx}{\linewidth}{@{}ll@{}}
    $\hbar = \frac{h}{2\pi}$ & $h$: Planck constant          \\
    $m,t,x$                  & Mass, time, position          \\
    $V(x,t)$                 & Potential energy function     \\
    $\Psi$                   & Complex particle wavefunction \\
                             & ("probability amplitude")     \\
\end{tabularx}
\renewcommand{\arraystretch}{1}
\setlength\tabcolsep{6pt} % default value: 6pt

\subsection{Planck}
When describing black body radiation with classical physics, the problem of \textit{infinite radiation at
    high frequencies at any temperature} occurs, becuase each of the infinite modes need a small amount of energy.
In quantum mechancs, each mode can only be filled in discrete amounts of energy $h \nu$:
\begin{equation*}
    k_B T_{low} \ll h\nu_{high}
\end{equation*}
which means that there is not enough thermal energy for high frequency modes to radiate.
\subsubsection{De Broglie}
The \textit{de Briogle relation} connects the wavelength $\lambda$ with the momentum $p$ for both light and matter
\begin{equation*}
    p=\frac{h}{\lambda}=\frac{2\pi\hbar}{\lambda}=\hbar\cdot k
\end{equation*}
\textbf{Remarks:}
\begin{enumerate}
    \item As $\lambda$ is a property of waves and $p$ one of particles respectively, this relation connects the two domains.
    \item QM becomes relevant if $\lambda$ is greater than the item's characteristic size $d$.
\end{enumerate}

\subsubsection{Probabilities}
The probability of finding the particle between $a$ and $b$ at time $t$ is:
\begin{equation*}
    \int_a^b \underbrace{{\Psi(x,t)}^*\Psi(x,t)}_{|\Psi(x,t)|^2}\,dx
\end{equation*}
The probability density of the is given by
\begin{equation*}
    \rho(x,t) = |\Psi(x,t)|^2
\end{equation*}
To be physically meaningful, $\Psi$ has to be in $L_2$ (square integrable) and be \textit{normalized}
\begin{align*}
    \int_{-\infty}^{\infty} |C\cdot \Psi(x,t)|^2 dx = 1, \; C\in\mathbb{C} \\
    \frac{d}{dt}\int_{-\infty}^{\infty} |\Psi(x,t)|^2 dx = 0
\end{align*}
\textbf{Remarks:}
\begin{enumerate}
    \item If $\Psi(x,t)$ is normalized at $t=0$ it will stay normalized $\forall t$.
    \item $\Psi \in L_2$ implies $\lim \limits_{x \to \infty}\Psi=0$
\end{enumerate}


\subsection{Position, Momentum and Uncertainty}
Any observable quantity $Q$ has an operator $\hat{Q}$ which can be formulated in terms of the position and momentum operators $\hat{x}, \hat{p}$
\begin{equation*}
    \hat{Q}(\hat{x},\hat{p}), \hspace*{10pt} \hat{x},\; \hat{p}=-i\hbar \frac{\partial}{\partial x}
\end{equation*}


\subsubsection{Expectation Value}
The \textbf{expectation value} of both position and momentum can be optained by
\begin{align*}
    \langle x \rangle & = \int_{-\infty}^{\infty} \hat{x} |\Psi(x,t)|^2 \, dx                                                                                                         \\
    \langle p \rangle & = \int_{-\infty}^{\infty} \Psi^* \underbrace{\left(-i\hbar \frac{\partial}{\partial x}\right)}_{\hat{p}} \Psi \, dx & = m \frac{d\langle \hat{x} \rangle}{dt}
\end{align*}
In general the expectation value can be obtained by measuring the \textbf{observable quantity} on a \textit{quantum mechanical ensemble} (ensemble of identically prepared systems):
\begin{align*}
    \langle Q(x,p)\rangle & = \int_{-\infty}^{\infty}\Psi^*\hat{Q}(\hat{x},\hat{p})\Psi dx
\end{align*}
if valid (e.g. $\hat{Q}$ contains no $\partial{x}$ or $\partial{t}$) this can be simplified to:\\
\begin{align*}
    \int_{-\infty}^{\infty}\hat{Q}(\hat{x},\hat{p}) \underbrace{|\Psi|^2}_{\rho} dx
\end{align*}
For example, the expectation value of the kinetic energy
\begin{align*}
    T                 & =\frac{mv^2}{2}=\frac{p^2}{2m}                                                             \\
    \text{is given by}                                                                                             \\
    \langle T \rangle & = -\frac{\hbar^2}{2m}\int_{-\infty}^{\infty}\Psi^*\frac{\partial^2\Psi}{\partial x^2} \,dx
\end{align*}

\textbf{Remarks:}
\begin{enumerate}
    \item An operator $\hat{Q}$ $\textit{represents}$ its observable $Q$ which means that $\hat{Q}$ yields $Q$ when "sandwiching" it and integrating over $\mathbb{R}$.
    \item $\textit{Ehrenfest's Theorem}$ states that expectation values in QM follow classical laws (example: $\langle p \rangle$).
\end{enumerate}

\subsubsection{Heisenberg Uncertainty Principle}
By gaining certainty on the observation of either position or momentum, the uncertainty of the other one increases:
\begin{equation*}
    \underbrace{\sigma_x\sigma_p}_{\text{std.\ dev.}} \geq \frac{\hbar}{2}
\end{equation*}


