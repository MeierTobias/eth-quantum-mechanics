\section{Time Independent SE}

\subsection{Stationary States}
The Schrödinger equation can be solved by separation of variables:
\noindent\begin{align*}
    \Psi_n(x,t)                                                        & = \psi_n(x)\varphi_n(t)                                                                                          \\
    \underbrace{i\hbar\frac1\varphi\frac{d\varphi}{dt}}_{\varphi_n(t)} & =\underbrace{-\frac{\hbar^2}{2m}\frac1\psi\frac{d^2\psi}{dx^2}+V}_{\psi_n (x)} = \underbrace{E}_{\text{const.}}
\end{align*}

\ptitle{LHS:}

The \textbf{time-dependent} part is a first order ODE that can be solved by:
\noindent\begin{align*}
    \frac{d\varphi}{d\iota} & =-\frac{iE}{\hbar}\varphi \\
    \varphi_n(t)            & =e^{-iE_n t/\hbar}
\end{align*}

\ptitle{RHS:}

The \textbf{time-independent} part is a second order ODE, whoose solution depends on the potential $V(x)$ (i.e.\ a measure of the environment of the particle).
\noindent\begin{equation*}
    \underbrace{\Bigl[\overbrace{-\frac{\hbar^2}{2m}\frac{d^2}{dx^2}}^{E_{kin}}+ \overbrace{V(x)}^{E_{pot}}\Bigr]}_{\widehat{H}}\psi = E\psi
\end{equation*}

\newpar{}
Assuming that the particle has only kinetic energy ($V(x) = 0$), two general solutions are possible:
\noindent\begin{align*}
    \psi(x) & =A\sin(kx)+B\cos(kx)                                               &  & \text{free particle} \\
    \psi(x) & =Ce^{ikx}+De^{-ikx}                                                &  & \text{ISW}           \\
    k       & =\sqrt{\frac{2mE}{\hbar^{2}}}=\frac{p}{\hbar}=\frac{2\pi}{\lambda} &  & \text{wave number}
\end{align*}

\ptitle{Hamiltonian}

$\widehat{H}$ is the Hamiltonian operator representing \textbf{total energy}:
\noindent\begin{equation*}
    \langle H\rangle = E,\quad{\sigma_H}^2 = 0
\end{equation*}

\textbf{Remarks:}
\begin{itemize}
    \item $\widehat{H}\psi = E\psi$ is an \textit{eigenvalue equation}
    \item every expectation value is constant in time
\end{itemize}




\subsubsection{Combining the Separable Solutions}
The \textbf{separable solutions}
\noindent\begin{equation*}
    \Psi_n(x,t)=\psi_n(x)e^{-iE_n t/\hbar}
\end{equation*}
are \textbf{stationary states} that can be superpositioned to obtain $\Psi(x,t)$ (\textit{Fourier series}):
\noindent\begin{equation*}
    \Psi(x,t) =\sum_{n=1}^\infty c_n\psi_n(x) \underbrace{e^{-iE_n t/\hbar}}_{\varphi(t)}=\sum_{n=1}^\infty c_n\Psi_n(x,t)
\end{equation*}
where $|c_n|^2$ is the \textit{probability that a measurement of the energy would return} $E_n$. Therfore $|c_n|^2$ is normalized and energy is conserved:
\noindent\begin{align*}
    \sum_{n=1}^\infty|c_n|^2 & =1                           \\
    \langle H\rangle         & =\sum_{n=1}^\infty|c_n|^2E_n
\end{align*}

\subsubsection{Free Particle}

\subsubsection{Infinite Square Well (ISW)}

\subsection{Harmonic Oscillator}