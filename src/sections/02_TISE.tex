\section{Time Independent SE (TISE)}

\subsection{Stationary States}
The Schrödinger equation can be solved by separation of variables:
\noindent\begin{align*}
    \Psi_n(x,t)                                                        & = \psi_n(x)\varphi_n(t)                                                                                         \\
    \underbrace{i\hbar\frac1\varphi\frac{d\varphi}{dt}}_{\varphi_n(t)} & =\underbrace{-\frac{\hbar^2}{2m}\frac1\psi\frac{d^2\psi}{dx^2}+V}_{\psi_n (x)} = \underbrace{E}_{\text{const.}}
\end{align*}

\ptitle{LHS:}

The \textbf{time-dependent} part is a first order ODE that can be solved by:
\noindent\begin{align*}
    \frac{d\varphi}{d\iota} & =-\frac{iE}{\hbar}\varphi                                               \\
    \varphi_n(t)            & =\exp\left[\frac{-iE_n t}{\hbar}\right],\quad E_{n}=\frac{n^{2}\pi^{2}\hbar^{2}}{2ma^{2}}
\end{align*}

\ptitle{RHS:}

The \textbf{time-independent} part is a second order ODE, whoose solution depends on the potential $V(x)$ (i.e.\ a measure of the environment of the particle).
\noindent\begin{equation*}
    \underbrace{\Bigl[\overbrace{-\frac{\hbar^2}{2m}\frac{d^2}{dx^2}}^{E_{kin}}+ \overbrace{V(x)}^{E_{pot}}\Bigr]}_{\widehat{H}}\psi = E\psi
\end{equation*}

\newpar{}
Assuming that the particle has only kinetic energy ($V(x) = 0$), two general solutions are possible:
\noindent\begin{align*}
    \psi(x) & =A\sin(kx)+B\cos(kx)                                               &  & \text{standing wave (ISW)}           \\
    \psi(x) & =Ce^{ikx}+De^{-ikx}                                                &  & \text{free particle} \\
    k       & =\sqrt{\frac{2mE}{\hbar^{2}}}=\frac{p}{\hbar}=\frac{2\pi}{\lambda} &  & \text{wave number}
\end{align*}

\textbf{Remark:}
\begin{itemize}
    \item $|\Psi(x,t)|^2 = |\psi(x)|^2$ (prob.\ density is time-independent):
    \noindent\begin{equation*}
        \langle Q\rangle=\int_{-\infty}^\infty\psi^*(x)\hat{Q}\psi(x)dx
    \end{equation*}
\end{itemize}

\newpar{}
\ptitle{Hamiltonian}

$\widehat{H}$ is the Hamiltonian operator representing \textbf{total energy}:
\noindent\begin{equation*}
    \langle H\rangle = E,\quad{\sigma_H}^2 = 0
\end{equation*}

\textbf{Remarks:}
\begin{itemize}
    \item $\widehat{H}\psi = E\psi$ is an \textit{eigenvalue equation}
    \item every expectation value is constant in time
\end{itemize}

\subsubsection{Combining the Separable Solutions}
The \textbf{separable solutions}
\noindent\begin{equation*}
    \Psi_n(x,t)=\psi_n(x)e^{-iE_n t/\hbar}
\end{equation*}
are \textbf{stationary states} that can be superpositioned to obtain $\Psi(x,t)$ (\textit{Fourier series}):
\noindent\begin{equation*}
    \Psi(x,t) =\sum_{n=1}^\infty c_n\psi_n(x) \underbrace{\exp\left[\frac{-iE_n t}{\hbar}\right]}_{\varphi_n(t)}=\sum_{n=1}^\infty c_n\Psi_n(x,t)
\end{equation*}
where $|c_n|^2$ is the \textit{probability that a measurement of the energy would return} $E_n$. Therfore $|c_n|^2$ is normalized and energy is conserved:
\noindent\begin{align*}
    \sum_{n=1}^\infty|c_n|^2 & =1                           \\
    \langle H\rangle         & =\sum_{n=1}^\infty|c_n|^2E_n
\end{align*}


\subsubsection{Infinite Square Well (ISW)}
A particle constrained ``walls'' behaves differently from a free particle. Given the potential i.e.\ constraint
\noindent\begin{equation*}
    V(x)=\begin{cases}0,&0\le x\le a\\\infty,&\text{otherwise}\end{cases}
\end{equation*}
the particle will manifest as infinitely many standing waves
\noindent\begin{align*}
    \psi_{n}(x)=\sqrt{\frac{2}{a}}\sin\left(\frac{n\pi}{a}x\right)
\end{align*}
each with a corresponding energy (quantized)
\noindent\begin{equation*}
    E_{n}=\frac{\hbar^{2}k_{n}^{2}}{2m}=\frac{n^{2}\pi^{2}\hbar^{2}}{2ma^{2}}
\end{equation*}

\begin{center}
    \includegraphics[width = 0.4\linewidth]{ISW.png}
\end{center}

\textbf{Remarks:}
\begin{itemize}
    \item Solutions alternate between \textit{odd} and \textit{even} starting from \textit{odd}
    \item $\psi_{n+1}$ has one node (zero-crossing) more than $\psi_n$
    \item $\psi_n$ are eigenfunctions/vectors and $E_n$ are the corresponding eigenvalues
    \item $\psi_n$ are orthonormal base functions (complete). This implies that any function can be  described by this \textit{Fourier series} (\textbf{Dirichlet's theorem}):
    \noindent\begin{equation*}
        \int_{0}^{a} \psi_m^*\psi_n\; dx = \delta_{mn}= \begin{cases}
            1 & m=n\\
            0 & m\neq n
        \end{cases}
    \end{equation*}
\end{itemize}

\ptitle{General Solution}

These standing waves can be combined to form
\noindent\begin{equation*}
    \Psi(x,t)=\sum_{n=1}^{\infty}c_{n} \underbrace{\sqrt{\frac{2}{a}}\sin\left(\frac{n\pi}{a}x\right)}_{\psi_n} \underbrace{\exp\left[-i\frac{n^{2}\pi^{2}\hbar}{2ma^{2}}t\right]}_{\varphi_n}.
\end{equation*}
With the initial condition $\Psi(x,0)$ the weights $c_n$ can be determined:
\noindent\begin{equation*}
    c_n=\sqrt{\frac{2}{a}}\int_0^a\sin\left(\frac{n\pi}{a}x\right)\Psi(x,0)dx
\end{equation*}

\subsubsection{Free Particle}
A \textit{free particle} propagates in space as a \textbf{wave packet} and its wavefunction is determined by
\noindent\begin{equation*}
    \Psi_{wp}(x_{i}t)=\frac{1}{\sqrt{2\pi}}\int_{-\infty}^{\infty}g(k)\exp\left[i\left(kx- \underbrace{\frac{\hbar k^{2}}{2m}}_{\omega}t\right)\right]dk
\end{equation*}
the corresponding \textbf{shape function} can be determined from the initial condition $\Psi(x,0)$ i.e.\ setting $t=0$ (inverse FT):
\noindent\begin{equation*}
    g(k)=\frac{1}{\sqrt{2\pi}}\int_{-\infty}^{\infty}\Psi_{wp}(x,0)e^{-ikx}dx
\end{equation*}

\textbf{Remark:}

\begin{itemize}
    \item A free particle can have any positive energy $E$.
\end{itemize}

\ptitle{Group Velocity}

A wavepacket consisting of infinitely many waves has a different \textit{group velocity} than the \textit{phase velocity} of the individual waves:
\noindent\begin{equation*}
    v_{group} = v_{classical} = 2v_{phase}
\end{equation*}

\begin{center}
    \includegraphics[width = 0.4\linewidth]{group_vel.png}
\end{center}


\subsection{Harmonic Oscillator}