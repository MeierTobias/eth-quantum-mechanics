\section{Time Dependent SE (TDSE)}

\subsection{Stationary States}
The Schrödinger equation can be solved by separation of variables:
\noindent\begin{align*}
    \Psi_n(x,t)                                                        & = \psi_n(x)\varphi_n(t)                                                                                         \\
    \underbrace{i\hbar\frac1\varphi\frac{d\varphi}{dt}}_{\varphi_n(t)} & =\underbrace{-\frac{\hbar^2}{2m}\frac1\psi\frac{d^2\psi}{dx^2}+V}_{\psi_n (x)} = \underbrace{E}_{\text{const.}}
\end{align*}

\ptitle{Left hand side:}

The \textbf{time-dependent} part is a first order ODE that can be solved by:
\noindent\begin{align*}
    \frac{d\varphi}{dt} & =-\frac{iE}{\hbar}\varphi \quad\Leftrightarrow                                 \\
    \varphi_n(t)        & =\exp\left[\frac{-iE_n t}{\hbar}\right], \qquad E_n = \frac{\hbar^2 k_n^2}{2m}
\end{align*}

Where the solution $\varphi_n(t)$ is called the \textbf{phase factor}.
\newpar{}

\ptitle{Right hand side:}

The \textbf{time-independent} part is a second order ODE, whoose solution depends on the potential $V(x)$ (i.e.\ a measure of the environment of the particle).
\noindent\begin{align*}
    \Bigl[\overbrace{-\frac{\hbar^2}{2m}\frac{d^2}{dx^2}}^{E_{kin}}+ \overbrace{V(x)}^{E_{pot}}\Bigr]\psi & = E\psi \\
    \widehat{H}\psi                                                                                       & = E\psi
\end{align*}
Which is also called the \textbf{time independent} SE (\textbf{TISE}).

\newpar{}
Assuming that the particle has only kinetic energy ($V(x) = 0$), two general solutions are possible:
\noindent\begin{align*}
    \psi(x) & =A\sin(kx)+B\cos(kx)                                               &  & \text{standing wave (ISW)} \\
    \psi(x) & =Ce^{ikx}+De^{-ikx}                                                &  & \text{free particle}       \\
    k       & =\sqrt{\frac{2mE}{\hbar^{2}}}=\frac{p}{\hbar}=\frac{2\pi}{\lambda} &  & \text{wave number}
\end{align*}

\ptitle{Hamiltonian}

$\widehat{H}$ is the Hamiltonian operator representing \textbf{total energy}:
\noindent\begin{equation*}
    \langle H\rangle = E,\quad{\sigma_H}^2 = 0
\end{equation*}

\newpar{}

\textbf{Remarks:}
\begin{itemize}
    \item $|\Psi(x,t)|^2 = |\psi(x)|^2$ (prob.\ density is time-independent):
          \noindent\begin{equation*}
              \langle Q\rangle=\int_{-\infty}^\infty\psi^*(x)\hat{Q}\psi(x)dx
          \end{equation*}
    \item $\widehat{H}\psi = E\psi$ is an \textit{eigenvalue equation}
    \item every expectation value is constant in time
\end{itemize}

\subsubsection{Combining the Separable Solutions}
The \textbf{separable solutions}
\noindent\begin{equation*}
    \Psi_n(x,t)=\psi_n(x)\underbrace{e^{-iE_n t/\hbar}}_{\varphi_n(t)}
\end{equation*}
are \textbf{stationary states} that can be superpositioned to obtain $\Psi(x,t)$ (\textit{Fourier series}):
\noindent\begin{equation*}
    \Psi(x,t) =\sum_{n=1}^\infty c_n\psi_n(x) \underbrace{\exp\left[\frac{-iE_n t}{\hbar}\right]}_{\varphi_n(t)}=\sum_{n=1}^\infty c_n\Psi_n(x,t)
\end{equation*}
where $|c_n|^2$ is the \textit{probability that an energy measurement would result in} $E_n$. Therefore $|c_n|^2$ is normalized and energy is conserved:
\noindent\begin{equation*}
    \sum_{n=1}^\infty|c_n|^2 =1
\end{equation*}
The expectation value of the energy of a particle in a general state is given by:
\noindent\begin{equation*}
    \langle H\rangle=\sum_{n=1}^\infty|c_n|^2E_n
\end{equation*}

\ptitle{Remark:}
In other words, $\Psi_n$ \textit{says} where a particle most likely \textit{is} when it has the energy $E_n$. So $\Psi$ tells us that a particle is in serval $\Psi_n$'s with several $E_n$'s at the same time. A Measurement will \textbf{collapse} $\Psi$ into one of the $\Psi_n$'s.

\subsubsection{Free Particle}
A \textit{free particle} propagates in space as a \textbf{wave packet} and its wavefunction is determined by
\noindent\begin{equation*}
    \Psi_{wp}(x_{i}t)=\frac{1}{\sqrt{2\pi}}\int_{-\infty}^{\infty}g(k)\exp\left[i\left(kx- \underbrace{\frac{\hbar k^{2}}{2m}}_{\omega}t\right)\right]dk
\end{equation*}
the corresponding \textbf{shape function} can be determined from the initial condition $\Psi(x,0)$ i.e.\ setting $t=0$ (inverse FT):
\noindent\begin{equation*}
    g(k)=\frac{1}{\sqrt{2\pi}}\int_{-\infty}^{\infty}\Psi_{wp}(x,0)e^{-ikx}dx
\end{equation*}

with:

\begin{align*}
    k & = \frac{2\pi}{\lambda} \\
    p & = \hbar k
\end{align*}

\textbf{Remark:}

\begin{itemize}
    \item A free particle can have any positive energy $E$.
    \item Separable solutions do not represent physical states, therefore $\notin$ Hilbert space.
\end{itemize}

\ptitle{Group Velocity}

A wavepacket consisting of infinitely many waves has a different \textit{group velocity} than the \textit{phase velocity} of the individual waves:
\noindent\begin{equation*}
    v_{group} = v_{classical} = 2v_{phase}
\end{equation*}

\begin{center}
    \includegraphics[width = 0.4\linewidth]{group_vel.png}
\end{center}


\subsubsection{Infinite Square Well (ISW)}
A particle constrained ``walls'' behaves differently from a free particle. Given the potential i.e.\ constraint
\noindent\begin{equation*}
    V(x)=\begin{cases}0,&0\le x\le a\\\infty,&\text{otherwise}\end{cases}
\end{equation*}
the particle will manifest as infinitely many standing waves
\noindent\begin{align*}
    \psi_{n}(x)=\sqrt{\frac{2}{a}}\sin\left(\frac{n\pi}{a}x\right)
\end{align*}
each with a corresponding energy (\textbf{quantized})
\noindent\begin{equation*}
    E_{n}=\frac{\hbar^{2}k_{n}^{2}}{2m}=\frac{n^{2}\pi^{2}\hbar^{2}}{2ma^{2}} > 0
\end{equation*}

\begin{center}
    \includegraphics[width = 0.4\linewidth]{ISW.png}
\end{center}

\textbf{Remarks:}
\begin{itemize}
    \item Solutions alternate between \textit{odd} and \textit{even} starting from $\psi_1$: \textit{odd}
    \item $\psi_{n+1}$ has one node (zero-crossing) more than $\psi_n$
    \item $\psi_n$ are eigenfunctions/vectors and $E_n$ are the corresponding eigenvalues
    \item $\psi_n$ are orthonormal base functions (complete). This implies that any function can be  described by this \textit{Fourier series} (\textbf{Dirichlet's theorem}):
          \noindent\begin{equation*}
              \int_{-\infty}^{\infty} \psi_m^*\psi_n\; dx = \delta_{mn}= \begin{cases}
                  1 & m=n     \\
                  0 & m\neq n
              \end{cases}
          \end{equation*}
\end{itemize}

\ptitle{Stationary states}
\begin{equation*}
    \Psi_n(x,t)=\underbrace{\sqrt{\frac{2}{a}}\sin\left(\frac{n\pi}{a}x\right)}_{\psi_n} \underbrace{\exp\left[-i\frac{n^{2}\pi^{2}\hbar}{2ma^{2}}t\right]}_{\varphi_n}.
\end{equation*}

\ptitle{General Solution}

These stationary states can be combined to form
\noindent\begin{equation*}
    \Psi(x,t)=\sum_{n=1}^{\infty}c_{n} \underbrace{\sqrt{\frac{2}{a}}\sin\left(\frac{n\pi}{a}x\right)}_{\psi_n} \underbrace{\exp\left[-i\frac{n^{2}\pi^{2}\hbar}{2ma^{2}}t\right]}_{\varphi_n}.
\end{equation*}
With the initial condition $\Psi(x,0)$ the weights $c_n$ can be determined:
\noindent\begin{equation*}
    c_n=\sqrt{\frac{2}{a}}\int_0^a\sin\left(\frac{n\pi}{a}x\right)\Psi(x,0)dx
\end{equation*}


\subsection{Harmonic Oscillator}

The Harmonic Oscillator differs from the Infinite Square Well by the Potential Energy function $V(x,t)$ which is given by the following quadratic function:

\begin{equation*}
    \widehat{V}(x) = \frac{1}{2}k_s x^2 = \frac{1}{2}m \omega^2 x^2 \qquad \text{with} \quad \omega = \sqrt{\frac{k_s}{m}}
\end{equation*}
The mechanical equivalent would be a frictionless spring and mass system with spring constant $k_s$.

\newpar{}

The TISE becomes
\noindent\begin{align*}
    \widehat{H}\psi                                                          & = E\psi \qquad \Leftrightarrow \\
    \Bigl[-\frac{\hbar^2}{2m}\frac{d^2}{dx^2}+\widehat{V}(x)\Bigr]\psi       & = E\psi \qquad \Leftrightarrow \\
    \frac{-\hbar^2}{2m}\frac{d^2\psi}{dx^2} + \frac{1}{2}m \omega^2 x^2 \psi & = E\psi
\end{align*}

\ptitle{Ladder Operators}

To solve this TISE one can use the \textbf{rising and lowering operators}: (for details on operators see~\ref{operators})
\begin{align*}
    \hat{a}_{+} & = \frac{1}{\sqrt{2\hbar m \omega}}\left(-i\hat{p}+m\omega\hat{x}\right) &  & \text{raising operator}  \\
    \hat{a}_{-} & = \frac{1}{\sqrt{2\hbar m \omega}}\left(+i\hat{p}+m\omega\hat{x}\right) &  & \text{lowering operator}
\end{align*}

\ptitle{Hamiltonian}

The Hamiltonian Operator can now be written as
\noindent\begin{align*}
    \hat{H} & =\hbar\omega\left(\hat{a}_{-}\hat{a}_{+}-\frac{1}{2}\right) \\
            & =\hbar\omega\left(\hat{a}_{+}\hat{a}_{-}+\frac{1}{2}\right)
\end{align*}
from which
\noindent\begin{gather*}
    \hat{a}_{-}\hat{a}_{+}=\frac{1}{\hbar\omega}\hat{H}+\frac{1}{2} \quad\quad \hat{a}_{+}\hat{a}_{-}=\frac{1}{\hbar\omega}\hat{H}-\frac{1}{2}\\
    \left[\hat{a}_{-},\hat{a}_{+}\right] = 1
\end{gather*}

\ptitle{TISE}

Applied to the TISE we get
\begin{align*}
    \widehat{H}(\hat{a}_{+}\psi_n) & = (E_n+\hbar\omega)\hat{a}_{+}\psi_n \\
    \widehat{H}(\hat{a}_{-}\psi_n) & = (E_n-\hbar\omega)\hat{a}_{-}\psi_n
\end{align*}
This implies, if $\psi_n$ are solutions to the TISE with energy $E_n$, then $\hat{a}_{+}\psi_n$/$\hat{a}_{-}\psi_n$ are also solutions but with one more/less quantum of energy.
\newpar{}
At the bottom of the energy potential function the steady state is given by
\begin{equation*}
    \hat{a}_{+}\psi_0 = 0\psi_0 = 0
\end{equation*}
Which is a simple solvable differential equation that results in
\begin{equation*}
    \psi_0 = {\left(\frac{m\omega}{\pi\hbar}\right)}^{\frac{1}{4}}\exp\left(\frac{-m\omega}{2\hbar}x^2\right)
\end{equation*}
To get $\psi_n$, $\hat{a}_{+}$ can be applied $n$-times.
\begin{equation*}
    \psi_n = \frac{1}{\sqrt{n!}}{\left(\hat{a}_{+}\right)}^n \psi_0
\end{equation*}
with
\begin{align*}
    E_0 & = \frac{1}{2}\hbar\omega     \\
    E_n & = (n+\frac{1}{2})\hbar\omega
\end{align*}

For an arbitrary state $n$ the next state is given by
\begin{align*}
    \hat{a}_{+}\psi_n & = \sqrt{n+1}\psi_{n+1} \\
    \hat{a}_{-}\psi_n & = \sqrt{n}\psi_{n-1}
\end{align*}

\includegraphics[width=\linewidth]{harmonic_oscillator.png}

\ptitle{Remarks:}
\begin{itemize}
    \item The energy is quantized.
    \item $n$ is the number of energy quanta in the oscillator
    \item $\widehat{N} \equiv \hat{a}_{+}\hat{a}_{-} = $ number operator because $\left<N\right> = n$ for $\psi_n$
    \item The lowest state $\psi_0$ has the \textbf{zero point energy}. Even if all the energy is removed from the system (zero Kelvin) the oscillator still has this zero point energy.
    \item Solutions alter between even and odd.
    \item $\psi_n$ form a complete basis.
    \item Soft boundary conditions - finite potential at the boundary/barrier.
\end{itemize}

\ptitle{Specific Expectation Values}
\noindent\begin{align*}
    \left\langle x \right\rangle _n   & = 0                                                                                                    & \forall n \in \mathbb{N} \\
    \left\langle x^2 \right\rangle _n & = \left(n+\frac{1}{2}\right)\frac{\hbar}{m\omega}                                                      & \forall n \in \mathbb{N} \\
    \left\langle p \right\rangle _n   & = 0                                                                                                    & \forall n \in \mathbb{N} \\
    \left\langle p^2 \right\rangle _n & = \left(n+\frac{1}{2}\right)\hbar m\omega                                                              & \forall n \in \mathbb{N} \\
    \left\langle T \right\rangle _n   & = \frac{\left\langle p^2 \right\rangle}{2m} =  \frac{\hbar \omega}{2}\left(n+\frac{1}{2}\right)                                   \\
    \left\langle V \right\rangle _n   & = \frac{m\omega^2}{2}\left\langle x^2 \right\rangle = \frac{\hbar \omega}{2}\left(n+\frac{1}{2}\right)
\end{align*}
Note that the energy $E_n$ is split up equally into kinetic and potential energy $T,V$

\subsection{Commutator}
\noindent\begin{align*}
    \left[\hat A,\hat B\right]           & \equiv\hat A\hat B-\hat B\hat A                    \\
    \left[c,\hat B\right]                & =0                              & c \in \mathbb{R} \\
    \left[\hat{x},\hat{p}\right]         & =i\hbar                                            \\
    \left[\hat{a}_{-},\hat{a}_{+}\right] & = 1
\end{align*}

\textbf{Remark}

For some operators it can be useful to evaluate the commutator on a test function first, and then remove the test function in the end, e.g.:
\noindent\begin{align*}
    \left[\frac{d^2}{dx^2}, \hat{x}\right] f(x) & = \frac{d^2}{dx^2} xf(x) - x\frac{d^2}{dx^2}f(x) \\
                                                & = 2\frac{d}{dx} f(x)                             \\
    \left[\frac{d^2}{dx^2}, \hat{x}\right]      & = 2\frac{d}{dx}
\end{align*}