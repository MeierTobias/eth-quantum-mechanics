\section{Formalism}

\subsection{Dirac Notation and Hilbert Spaces}
Assuming the functions $\alpha, \beta$ are square integrable i.e.\ in $L_2$ (\textbf{Hilbert space}) $\mathcal{H}$,
Dirac notation is used to simplify the notation of vectors, inner product (scalar product), expectation values etc. For the functions
\begin{align*}
    f   & = \sum_n c_n f_n     \\
    f^* & = \sum_n c_n^* f_n^*
\end{align*}
we define:
\noindent\begin{align*}
    |f\rangle           & := \begin{bmatrix}
                                 c_1 & c_2 & \cdots & c_n
                             \end{bmatrix}^T                               &  & \text{``ket''}              \\
    \langle f|          & := \begin{bmatrix}
                                 {c_1}^* & {c_2}^* & \cdots & {c_n}^*
                             \end{bmatrix}                   &  & \text{``bra''}                            \\
    \langle f|g \rangle & := \int_{-\infty}^{\infty} f^* g\; d \mathbf{r} \in L_2 &  & \text{inner product}
\end{align*}

These functions can be used as a complete orthonormal base (ONB) if
\noindent\begin{align*}
    \langle f_m|f_n \rangle & = \delta_{mn}                    &  & \text{orthonormal}        \\
    f(x)                    & = \sum_{n=1}^{\infty} c_n f_n(x) &  & \text{complete}           \\
    c_n                     & = \langle f_n|f \rangle          &  & \text{projection on base}
\end{align*}


\subsubsection{Properties of the Inner Product}

\noindent\begin{align*}
    \langle \lambda f|g \rangle & =\lambda^* \langle f|g \rangle             &  & \text{semilinear for scalar mul.} \\
    \langle f|\lambda g \rangle & =\lambda \langle f|g \rangle                                                      \\
    \langle f+g|h \rangle       & =\langle f|h \rangle + \langle g|h \rangle &  & \text{linear for addition}        \\
    \langle f|g+h \rangle       & =\langle f|g \rangle + \langle f|h \rangle                                        \\
    \langle g|f \rangle         & = {\langle f|g \rangle}^*                  &  & \text{``hermitian''}              \\
    \langle f|f \rangle         & \ge 0                                      &  & \text{positive definite}          \\
    \langle f|f \rangle         & = 0 \leftrightarrow f=0                                                         
\end{align*}

\subsubsection{Properties of QM Hilbert Space}
\begin{itemize}
    \item Describes physical solutions to QM systems
    \item The stationary states $\psi_n$ form an ONB and are complete
\end{itemize}

\subsection{Observables}
Observables are represented by hermitian operators.
\subsubsection{Hermitian Operators}
The expectation value of a observable is real and thus
\noindent\begin{align*}
    \langle Q\rangle           & = {\langle Q\rangle}^*                       \\
    \langle f|\hat{Q} g\rangle & = \langle \hat{Q}f|g\rangle\quad \forall f,g
\end{align*}
Such an operator is called a hermitian operator (applying the operator to one or the other member of an inner product yields the same result). These operators are \textbf{linear} if
\noindent\begin{equation*}
    \hat{Q}\left[af(x)+bg(x)\right]=a\hat{Q}f(x)+b\hat{Q}g(x),
\end{equation*}

\ptitle{Properties}

\noindent\begin{align*}
    \langle f|(\hat{Q} + \hat{R})g\rangle                                    & = \langle (\hat{Q} + \hat{R})f|g\rangle                                                               \\[0.75em]
    \langle f|\hat{Q}\hat{R}g\rangle       =\langle \hat{Q}f|\hat{R}g\rangle & =  \langle \hat{R}\hat{Q}f|g\rangle \overset{[\hat{Q},\hat{R}]=0}{=} \langle \hat{Q}\hat{R}f|g\rangle
\end{align*}

\subsubsection{Determinant States}
A state $\Psi$ that would return the same result for every measurement of an ensemble is called a \textbf{determinant state} of the observable $Q$.
In other words, there would be no variance
\noindent\begin{align*}
    \sigma^2     & = \langle \Psi|(\hat{Q} - q) \Psi\rangle = 0 \\
    \hat{Q} \Psi & = q \Psi
\end{align*}
These determinant states of $Q$ are \textit{eigenfunctions} of $\hat{Q}$ and the expectation $\langle Q\rangle = q$ the corresponding \textit{eigenvalue}.
\newpar{}
In short: If a state fulfils the eigenvalue equation, every measurement will yield $q$.

\textbf{Remarks:}
\begin{itemize}
    \item A set of eigenvalues $q$ for $\hat{Q}$ is called its \textit{spectrum}.
    \item If multiple eigenfunctions share their eigenvalue, they are called \textit{degenerated states}.
    \item Eigenfunctions (of a Hermitian operator) with different eigenvalues are orthogonal and form a complete set.
    \item The TISE is an example for a determinant state\newline
          $\hat{Q}: \hat{H},\; q:E$.
\end{itemize}

\subsection{Eigenfunctions of a Hermitian Operator}
Eigenfunctions of hermitian operators can either have a discrete or continous spectrum.

\newpar{}
If $\psi_n$ are solutions to $\hat{Q}\psi_n=q\psi_n$ then
\noindent\begin{align*}
    \Psi(x,0)    & = \sum_{n=1}^{\infty} c_n \underbrace{\psi_n(x)}_{\textsf{eigenfunctions}}    \\
    |\Psi\rangle & = \sum_{n=1}^{\infty} c_n \underbrace{|\psi_n\rangle}_{\textsf{eigenvectors}}
\end{align*}

\ptitle{Discrete Spectrum}

\begin{itemize}
    \item Eigenvalues are real
    \item Eigenfunctions with different eigenvalues are orthogonal
    \item Eigenfunctions with a discrete spectrum are within the Hilbert space $\mathcal{H}$, represent physical states and form a complete set.
\end{itemize}

\ptitle{Continous Spectrum}

\begin{itemize}
    \item Eigenfunctions with a continous spectrum are not normalizable
    \item These eigenfuntions with real eigenvalues are \textit{Dirac-orthonormalizable} and form a complete set.
\end{itemize}

\subsection{Generalized Statistical Interpretation}
If the spectrum of $\hat{Q}$ is discrete and the state
\begin{equation*}
    \Psi=\sum_n c_n f_n
\end{equation*}
is \textbf{not a eigenfunction} of $\hat{Q}$ (but $f_n$ is), the probability of measuring the particular eigenvalue $q_n$ corresponding to the eigenfunction $f_n(x)$ is
\noindent\begin{equation*}
    |c_n|^2, \quad c_n = \langle f_n|\Psi\rangle
\end{equation*}
It follows that
\noindent\begin{align*}
    \langle Q\rangle & = \sum_{n=1}^{\infty} |c_n|^2 q_n                     \\
    1                & =\sum_{n=1}^{\infty} |c_n|^2      & \text{normalized} \\
    \sigma^2         & \neq 0
\end{align*}
As mentioned in Section\ \ref{ssec:ISW}, these coefficients can be obtained by projecting $\Psi$ onto $\psi_n$:
\noindent\begin{equation*}
    c_n = \int_{-\infty}^{\infty} {\psi_n}^* \Psi\; d^3 \mathbf{r} = \langle \psi_n |\Psi \rangle
\end{equation*}

\textbf{Important Remark}

After a measurement of an observable, the wave function \textbf{collapses} into the measured eigenfunction $\psi_n$ with eigenvalue $q_n$ and probability $|c_n|^2$. This collapse reflects the transition from a superposition of possible states to a \textbf{definitive state}.

\subsection{Postulates of Quantum Mechanics}

\begin{enumerate}
    \item The state of a QM system is described by $\Psi(\mathbf{r},t)$, and $|\Psi|^2\; d^3 \mathbf{r}$ is the probability of finding the particle in the volume $d^3 \mathbf{r} = dx\,dy\,dz$.
    \item Every observable quantity in classical mechanics is represented by a \textbf{linear hermitian operator}, $\hat{Q}$, such that the mean value of the observable from an ensemble is
          \noindent\begin{equation*}
              \langle Q\rangle=\int\Psi^{*}\hat{Q}\Psi d^{3} \mathbf{r}= \langle\Psi|\hat{Q}\Psi\rangle
          \end{equation*}\newline
          \textbf{Restated}:
          If the system is in a eigenstate of $\hat{Q}$, identical measurements will always yield the eigenvalue $q$ of $\hat{Q}$.
          Else, a single measurement will yield one of the eigenvalues $q_n$ of $\hat{Q}$ with probability $|c_n|^2$.
\end{enumerate}