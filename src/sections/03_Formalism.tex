\section{Formalism}

\subsection{Dirac Notation and Hilbert Spaces}
Assuming the functions $f,g$ are square integrable i.e.\ in $L_2$ (\textbf{Hilbert space}) $\mathcal{H}$,
Dirac notation is used to simplify the notation of vectors, inner product (scalar product), expectation values etc. For the functions
\begin{align*}
    f   & = \sum_n c_n f_n     \\
    f^* & = \sum_n c_n^* f_n^*
\end{align*}
we define:
\noindent\begin{align*}
    \left|f\right\rangle  & := \begin{bmatrix}
                                   c_1 & c_2 & \cdots & c_n
                               \end{bmatrix}^T                               &  & \text{``ket''}              \\
    \left\langle f\right| & := \begin{bmatrix}
                                   {c_1}^* & {c_2}^* & \cdots & {c_n}^*
                               \end{bmatrix}                   &  & \text{``bra''}                            \\
    \langle f|g \rangle   & := \int_{-\infty}^{\infty} f^* g\; d \mathbf{r} \in L_2 &  & \text{inner product}
\end{align*}
where ``ket'' and ``bra'' are \textbf{vectors} that represent the function $f$.

These functions can be used as a complete orthonormal base (ONB) if
\noindent\begin{align*}
    \langle f_m|f_n \rangle & = \delta_{mn}                    &  & \text{orthonormal}        \\
    f(x)                    & = \sum_{n=1}^{\infty} c_n f_n(x) &  & \text{complete}           \\
    c_n                     & = \langle f_n|f \rangle          &  & \text{projection on base}
\end{align*}

\textbf{Remarks}
\begin{itemize}
    \item $\sum_{j=1}^{n} \left|f_j\right\rangle \left\langle f_j\right| = \mathbf{I}$ (Closure relation)
    \item More properites of the inner product can be found in Appendix\ \ref{ssec:InnerProd}
\end{itemize}

\subsubsection{Properties of QM Hilbert Space}
\begin{itemize}
    \item Describes \textbf{physical} solutions to QM systems
    \item The stationary states $\psi_n$ form an ONB and are complete
\end{itemize}

\subsection{Observables}
Observables are represented by hermitian operators.
\subsubsection{Hermitian Operators}
The expectation value of a observable is real and thus
\noindent\begin{align*}
    \langle Q\rangle                            & = {\langle Q\rangle}^*                                            \\
    \langle Q\rangle=\langle f|\hat{Q} g\rangle & = \langle \hat{Q}f|g\rangle={\langle Q\rangle}^*\quad \forall f,g
\end{align*}
Such an operator is called a hermitian operator (applying the operator to one or the other member of an inner product yields the same result). An operator \textbf{must} be hermitian for its observable to become real. These operators are \textbf{linear} if
\noindent\begin{equation*}
    \hat{Q}\left[af(x)+bg(x)\right]=a\hat{Q}f(x)+b\hat{Q}g(x),
\end{equation*}

\ptitle{Properties}

\noindent\begin{align*}
    \langle f|(\hat{Q} + \hat{R})g\rangle                                    & = \langle (\hat{Q} + \hat{R})f|g\rangle                                                               \\[0.75em]
    \langle f|\hat{Q}\hat{R}g\rangle       =\langle \hat{Q}f|\hat{R}g\rangle & =  \langle \hat{R}\hat{Q}f|g\rangle \overset{[\hat{Q},\hat{R}]=0}{=} \langle \hat{Q}\hat{R}f|g\rangle
\end{align*}

\ptitle{Remark}

Given hermitian operators $\hat{A}$, $\hat{B}$ and $\hat{A}\hat{B}$ then, in integral notation one has for operators and functions
\begin{equation*}
    \int_{-\infty}^{\infty} {(\hat{A}\hat{B}f)}^* g\; d \mathbf{r}=\int_{-\infty}^{\infty} {(\hat{A}\hat{B})}^*f^* g\; d \mathbf{r}
\end{equation*}
but for operators and operators
\begin{equation*}
    \int_{-\infty}^{\infty} {(\hat{A}\hat{B})}^*f^* g\; d \mathbf{r}=\int_{-\infty}^{\infty} \hat{B}^*\hat{A}^*f^* g\; d \mathbf{r}
\end{equation*}

\subsubsection{Determinant States}
A state $\Psi$ that would return the same result for every measurement of an ensemble is called a \textbf{determinant state} of the observable $Q$.
In other words, there would be no variance
\noindent\begin{align*}
    \sigma^2     & = \langle \Psi|(\hat{Q} - q) \Psi\rangle = 0 \\
    \hat{Q} \Psi & = q \Psi
\end{align*}
These determinant states of $Q$ are \textit{eigenfunctions} of $\hat{Q}$ and the expectation $\langle Q\rangle = q$ the corresponding \textit{eigenvalue}.
\newpar{}
In short: If a state fulfils the eigenvalue equation, every measurement will yield $q$.

\textbf{Remarks:}
\begin{itemize}
    \item A set of eigenvalues $q$ for $\hat{Q}$ is called its \textit{spectrum}.
    \item If multiple eigenfunctions share their eigenvalue, they are called \textit{degenerated states}.
    \item Eigenfunctions (of a Hermitian operator) with different eigenvalues are orthogonal and form a complete set.
    \item The TISE is an example for a determinant state\newline
          $\hat{Q}: \hat{H},\; q:E$.
\end{itemize}

\subsubsection{Eigenfunctions of a Hermitian Operator}
Eigenfunctions of hermitian operators can either have a discrete or continuous spectrum.

\newpar{}
\ptitle{Discrete Spectrum}

If $\psi_n$ are solutions to $\hat{Q}\psi_n=q\psi_n$ then
\noindent\begin{align*}
    \Psi(x,0)    & = \sum_{n=1}^{\infty} c_n \underbrace{\psi_n(x)}_{\textsf{eigenfunctions}}    \\
    |\Psi\rangle & = \sum_{n=1}^{\infty} c_n \underbrace{|\psi_n\rangle}_{\textsf{eigenvectors}}
\end{align*}

\begin{itemize}
    \item Eigenvalues are real
    \item The eigenfunctions form a countable set (e.g.\ harmonic oscillator)
    \item Eigenfunctions with different eigenvalues are orthogonal
    \item Eigenfunctions with a discrete spectrum are within the Hilbert space $\mathcal{H}$, represent \textbf{physical} states and form a complete set.
    \item $|\psi_n\rangle$ can be thought of as basis vectors:
        \noindent\begin{equation*}
            \langle f_m|f_n\rangle=\delta_{mn}
        \end{equation*}
\end{itemize}

\ptitle{Continuous Spectrum}

\begin{itemize}
    \item Eigenfunctions with a continuous spectrum are not normalizable ($\notin \mathcal{H}$)
    \item The eigenfunctions appear as continuous set (e.g.\ free particle)
    \item These eigenfunctions with real eigenvalues are \textit{Dirac-orthonormalizable} and form a \textbf{complete set}.
    \item In contrast to the discrete spectrum case where we had $\langle f_m|f_n\rangle=\delta_{mn}$ we now have
          \noindent\begin{equation*}
              \langle f_{x'}|f_{x''}\rangle=\delta(x''-x')
          \end{equation*}
\end{itemize}

\subsubsection{Generalized Statistical Interpretation}
If the spectrum of $\hat{Q}$ is discrete and the state
\begin{equation*}
    \Psi=\sum_n c_n f_n
\end{equation*}
is \textbf{not a eigenfunction} of $\hat{Q}$ (but $f_n$ is), the probability of measuring the particular eigenvalue $q_n$ corresponding to the eigenfunction $f_n(x)$ is
\noindent\begin{equation*}
    |c_n|^2, \quad c_n = \langle f_n|\Psi\rangle
\end{equation*}
It follows that
\noindent\begin{align*}
    \langle Q\rangle & = \langle\Psi_{general}|\hat{Q}\Psi_{general}\rangle                                                                 \\
                     & =\left(\sum_{m}c_{m}^{*}\langle\Psi_{m}|\right)\hat{Q}\left(\sum_{n}c_{n}|\Psi_{n}\rangle\right)                     \\
                     & = \sum_{n=1}^{\infty} |c_n|^2 q_n                                                                                    \\
    1                & =\sum_{n=1}^{\infty} |c_n|^2                                                                     & \text{normalized} \\
    \sigma^2         & \neq 0
\end{align*}
As mentioned in Section\ \ref{ssec:ISW}, these coefficients can be obtained by projecting $\Psi$ onto $\psi_n$:
\noindent\begin{equation*}
    c_n = \int_{-\infty}^{\infty} {\psi_n}^* \Psi\; d^3 \mathbf{r} = \langle \psi_n |\Psi \rangle
\end{equation*}

\textbf{Important Remark}

After a measurement of an observable, the wave function \textbf{collapses} into the measured eigenfunction $\psi_n$ with eigenvalue $q_n$ and probability $|c_n|^2$. This collapse reflects the transition from a superposition of possible states to a \textbf{definitive state}.

\subsection{Postulates of Quantum Mechanics}

\begin{enumerate}
    \item The state of a QM system is described by $\Psi(\mathbf{r},t)$, and $|\Psi|^2\; d^3 \mathbf{r}$ is the probability of finding the particle in the volume $d^3 \mathbf{r} = dx\,dy\,dz$.
    \item Every observable quantity in classical mechanics is represented by a \textbf{linear hermitian operator}, $\hat{Q}$, such that the mean value of the observable from an ensemble is
          \noindent\begin{equation*}
              \langle Q\rangle=\int\Psi^{*}\hat{Q}\Psi d^{3} \mathbf{r}= \langle\Psi|\hat{Q}\Psi\rangle
          \end{equation*}\newline
          \textbf{Restated}:\newline
          If the system is in a state that is an eigenstate of $\hat{Q}$, a measurement on a QM ensemble will always yield the eigenvalue $q$ of $\hat{Q}$ (e.g.\ determinant state of the infinite square well).\newline
          Else, a single measurement will yield one of the eigenvalues $q_n$ of $\hat{Q}$ with probability $|c_n|^2$ (e.g.\ general state of infinite square well).
    \item $\Psi(\mathbf{r},t)$ evolves in time according to the TDSE:
          \noindent\begin{align*}
              i\hbar \frac{\partial \Psi}{\partial t} & =\hat{H}\Psi                     \\
              \hat{H}                                 & = \frac{\hat{p}^2}{2m} + \hat{V}
          \end{align*}
\end{enumerate}

\subsection{Operators with Continuous Sets of Eigenfunctions}
Remember: these eigenfunctions are not normalizable (not in Hilbert space) but those of them with real eigenvalues are Dirac-orthonormalizable and form a complete set.

\subsubsection{Continuous Set of Eigenfunctions for Position}

\ptitle{Eigenvalue Equation}

An eigenfunction of the position operator must fulfil the following eigenvalue equation for a specific $x'$:
\begin{equation*}
    \hat{x}g_{x'} =x'g_{x'}
\end{equation*}
and hence,
\begin{equation*}
    x \cdot g_{x'} =x' \cdot g_{x'}
\end{equation*}

\ptitle{Eigenfunctions}

The eigenfunction
\begin{equation*}
    g_{x'}=\delta(x-x')
\end{equation*}
satisfies the stated eigenvalue equation as
\begin{equation*}
    x\cdot \delta(x-x')=x'\cdot \delta(x-x')
\end{equation*}
which means that measuring the position of a particle in state $g_{x'}$ (not physical) would always yield position $x'$.

\ptitle{Dirac-Orthonormality}

One can easily show that
\begin{equation*}
    \langle g_{x'}|g_{x''}\rangle=\delta(x''-x')
\end{equation*}

\ptitle{Completeness}

Any function $f$ can be written (as an infinite linear combination of eigenfunctions) in terms of $g_{x'}$:
\begin{equation*}
    \int_{-\infty}^{\infty}f(x^{\prime})g_{x^{\prime}}(x)dx^{\prime}=\int_{-\infty}^{\infty}f(x^{\prime})\delta(x-x^{\prime})dx^{\prime}=f(x)
\end{equation*}
where the coefficients are just given by the values of $f$ at the desired location i.e. $f(x')$.

\ptitle{Remarks}

\begin{itemize}
    \item In the discrete case the integral was a sum, $f(x')$ was $c_n$
\end{itemize}

\subsubsection{Continuous Set of Eigenfunctions for Momentum}

\ptitle{Eigenvalue Equation}

An eigenfunction of the momentum operator must fulfil the following eigenvalue equation for a specific $p'$:
\begin{equation*}
    \hat{p}f_{p^{\prime}}=p^{\prime}f_{p^{\prime}}
\end{equation*}

\ptitle{Eigenfunctions}

Plugging $\hat{p}=-i\hbar \frac{d}{dx}$ into the eigenvalue equation and solving the ODE yields waves with
$k=\frac{2 \pi}{\lambda}=\frac{p'}{\hbar}$ and wavelength $\lambda=\frac{2\pi\hbar}{p^{\prime}}=\frac{h}{p^{\prime}}$
(de Broglie formula!):
\begin{equation*}
    f_{p^{\prime}}(x)=\frac{1}{\sqrt{2\pi\hbar}}\exp\left[\frac{ip^{\prime}x}{\hbar}\right]
\end{equation*}
which means that measuring the momentum of a particle in state $f_{p^{\prime}}(x)$ will always yield momentum $p'$.

\ptitle{Dirac-Orthonormality and Completeness}

As for $\hat{x}$ one can also show for $\hat{p}$ that
\begin{itemize}
    \item $\langle f_{p'}|f_{p''}\rangle=\delta(p''-p')$
    \item and $f_{p'}$ form a complete set
\end{itemize}


\subsection{The Generalized Uncertainty Principle}

\subsubsection{Compatible Observables}

\begin{itemize}
    \item The operators do commute: $[\widehat{A}, \widehat{B}] = 0$
    \item The operators have the same eigenfunctions
    \item Two compatible observables can be determined simultaneously:
          \begin{enumerate}
              \item Measuring $A$ yields eigenvalue $a_n$ with probability $|c_{n,A}|^2$ and $\sigma_A=0$
              \item Wave function collapses into the corresponding eigenfunction $\psi'$
              \item Subsequent measurements for $A,B$ will yield eigenvalues $a_n, b_m$, $\sigma_A=0, \sigma_B=0$
          \end{enumerate}
    \item Both measurements ($A$ and subsequent $A$ or $B$) are \textbf{deterministic}
\end{itemize}

\subsubsection{Incompatible Observables}

\begin{itemize}
    \item The operators don't commute: $[\widehat{A}, \widehat{B}] \neq 0$
    \item The operators don't have the same eigenfunctions
    \item The two observables can't be determined simultaneously without any uncertainty:
          \begin{enumerate}
              \item Measuring $A$ yields eigenvalue $a_n$ with probability $|c_{n,A}|^2$ and $\sigma_A=0$
              \item Wave function collapses into the corresponding eigenfunction $\psi'$
              \item Subsequent measurements
                    \begin{itemize}
                        \item for $A$ will yield eigenvalues $a_n$, $\sigma_A=0$
                        \item for $B$ will yield eigenvalue $b_m$ with probability $|c_{m,B}|^2$ (eigenfunction is not shared)
                    \end{itemize}
          \end{enumerate}
    \item Measurement $A$ and subsequent $A$ is \textbf{deterministic} but subsequent $B$ is \textbf{probabilistic}.

\end{itemize}

\textbf{Remarks}:
\begin{itemize}
    \item The previous statements hold for two subsequent measurements, if for example, a third measurement is made, the compatability of the second and third have to be evaulated
    \item Check the table of commutators in Appendix\ \ref{comm}
\end{itemize}

\newpar{}
\ptitle{General Uncertainty Principle (GUP)}

For any Observables $A$, $B$:
\begin{equation*}
    \sigma_A^2\cdot\sigma_B^2\geqslant{\left(\frac1{2i}\langle[\hat{A},\hat{B}]\rangle\right)}^2
\end{equation*}

\ptitle{Remarks}

\begin{itemize}
    \item The quantity $\left(\frac1{2i}\langle[\hat{A},\hat{B}]\rangle\right)$ is real (and it's square hence positive).
    \item $\langle[\hat{A},\hat{B}]\rangle=\langle\Psi|\hat{A}\hat{B}\Psi\rangle-\langle\Psi|\hat{B}\hat{A}\Psi\rangle$
\end{itemize}

\begin{examplesection}[Example: Position and Momentum]
    For $\hat{A}=x$, $\hat{B}=-i\hbar d/dx$ we have
    \begin{align*}
        \left[\hat{x},\hat{p}\right] & =i\hbar                                                                      \\
        \sigma_{x}^{2}\sigma_{p}^{2} & \geq{\left(\frac{1}{2i}i\hbar\right)}^{2}={\left(\frac{\hbar}{2}\right)}^{2} \\                                                           \\
        \sigma_{x}\sigma_{p}         & \geq\frac{\hbar}{2}
    \end{align*}
\end{examplesection}

\subsubsection{Time-Energy Uncertainty}
Plugging the \textbf{generalized Ehrenfest theorem}
\begin{equation*}
    \frac{d}{dt}\left\langle Q\right\rangle=\frac{i}{\hbar}\left\langle\left[\hat{H},\hat{Q}\right]\right\rangle+\left\langle\frac{\partial\hat{Q}}{\partial t}\right\rangle
\end{equation*}
into the GUP yields, assuming $\hat{Q}$ is time-independent, the \textbf{time-energy uncertainty}:
\begin{equation*}
    \Delta t\cdot\Delta E\geqslant\frac\hbar2
\end{equation*}
where
\begin{align*}
    \Delta t & \equiv \frac{\sigma_Q}{|d\langle Q\rangle/dt|} \\
    \Delta E & \equiv \sigma H
\end{align*}

\ptitle{Remarks}

\begin{itemize}
    \item $\Delta t$ describes how long it takes the expectation value of $Q$ to change by $\sigma$.
    \item The generalized Ehrenfest theorem states that the rate of change of $\langle Q \rangle$ is determined by the commutator of the operator and the Hamiltonian.
    \item If they commute, $Q$ is a conserved quantity.
\end{itemize}